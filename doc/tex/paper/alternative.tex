

%\subsection{Alternative to DGLAP DIS models}
Different approaches that are alternative to the DGLAP formalism can be used to analyse DIS data in \fitter .
These include several different dipole models and the use of 
transverse momentum dependent, or unintegrated PDFs, uPDFs.
These approaches are discussed below.

\subsection{DIPOLE models}

The dipole picture provides an alternative approach to virtual photon-proton
 scattering at low $x$ which allows the description of both inclusive and 
diffractive processes.
 In this approach, the virtual photon fluctuates into a $q\bar q$ (or $q\bar q g$) 
 dipole which interacts with the proton~\cite{NNZ:91}.  
The dipoles can be viewed as quasi-stable quantum mechanical states, which have very long 
life time $\propto 1/m_p x\;$ and a size which is not changed by scattering.
%A schematic view of dipole factorisation at small $x$ in DIS is illustrated in figure~\ref{fig:dipole}.
The dynamics of the interaction are embedded in the dipole scattering amplitude.

%\begin{figure}
%\begin{center}
%\includegraphics[width=0.5\linewidth]{figures/dipole.pdf}
%\end{center}
%\caption{Schematic diagram of dipole factorisation for the inclusive cross section in DIS.}
%\label{fig:dipole}
%\end{figure}

Several dipole models which assume different behavior of the dipole-proton 
cross sections are implemented in \fitter\ :
%\begin{itemize}
%\item the original Golec-Biernat-W\"usthoff (GBW)~\cite{Golec-Biernat:1998js} 
the Golec-Biernat-W\"usthoff (GBW)
dipole saturation model~\cite{Golec-Biernat:1998js},
the colour glass condensate approach to the high parton density 
regime called the Iancu-Itakura-Munier (IIM) dipole model~\cite{Iancu:2003ge} and 
a modified GBW model which takes into account the effects of  
DGLAP evolution called the Bartels-Golec-Kowalski (BGK) dipole model~\cite{Bartels:2002cj}.
%\end{itemize}

\begin{description}
\item \bf {GBW model:} \rm
In the GBW model the dipole-proton cross section $\sigma_{\text{dip}}$ is given by
\begin{equation}
\label{eGBW}
   \sigma_{\text{dip}}(x,r^{2}) = \sigma_{0} \left(1 - \exp \left[-\frac{r^{2}}{4R_{0}^{2}(x)} \right]\right),
\end{equation}
here $r$ corresponds to the transverse separation between the quark and the antiquark, and $R_{0}^{2}$
 is 
%an $x$ dependent scale parameter, having the form $R_{0}^{2}(x)=\left(x/x_{0}\right)^{\lambda}$.
an $x$-dependent scale parameter which represents the spacing of the gluons in the proton. $R_{0}^{2}(x)=\left(x/x_{0}\right)^{\lambda}$ is called the saturation radius.
The fitted parameters are the cross-section normalisation $\sigma_{0}$ and $x_{0}$ and $\lambda$. This model gives exact Bjorken scaling when the dipole size $r$ is small.
%%%%

\vspace{0.1cm}
\item \bf {IIM model:} \rm
The IIM model assumes an improved expression for the dipole cross section which is based on the 
Balitsky-Kovchegov equation~\cite{Balitsky:1995ub}. The explicit formula for $\sigma_{\text{dip}}$ 
can be found in~\cite{Iancu:2003ge}. The fitted parameters are an alternative scale parameter $\tilde{R}$, $x_{0}$ and $\lambda$.
%%%%

\vspace{0.1cm}
\item \bf {BGK model:} \rm
The BGK model modifies the GBW model by taking into account the  DGLAP evolution
of the gluon density. 
The dipole cross section is given by
\begin{equation}
\begin{array}{lcl}
   \sigma_{\text{dip}}(x,r^{2})  =  \sigma_{0} 
\left(1 - \exp \left[-\frac{\pi^{2} r^{2} \alpha_{s}(\mu^{2}) xg(x,\mu^{2})}{3 \sigma_{0}} \right]\right).
\end{array}
\label{eBGK}
\end{equation}
The factorization scale $\mu^{2}$ has the form $\mu^{2} = C_{bgk}/r^{2}+\mu^{2}_{0}$.
%This model uses the following gluon density at the starting scale $Q_{0}^{2}=1\mbox{ GeV}^{2}$.
This model relates to the GBW model using the idea that the spacing $R_0$ is inverse to the gluon density.
The gluon density parametrized at some starting scale $Q_{0}^{2}$ by
$ xg(x) = A_{g} x^{-\lambda_{g}}(1-x)^{C_{g}} $
is evolved to larger scales using LO or NLO DGLAP evolution.
The fitted parameters for this model are $\sigma_{0}$, $\mu^{2}_{0}$ and three parameters for the gluon density: $A_{g}$, $\lambda_{g}$, $C_{g}$. The parameter $C_{bgk}$ is kept fixed: $C_{bgk} = 4.0$. 
%%%%

\vspace{0.1cm}
\item \bf {BGK model with valence quarks:} \rm

The dipole models are valid in the low-$x$ region only, where the valence quark contribution is small, of the order of 5\%. The new HERA $F_2$ data have a precision which is  better than 2 \%. Therefore, in \fitter\ the contribution of the valence quarks is taken from the PDF fits and added to the original 
BGK model, this is uniquely possible within the \fitter\ framework.
% The quality of the fits of the BGK dipole model with valence quarks and without 
%valence quarks are the same.
\end{description}

\subsection{Transverse Momentum Dependent (unintegrated) PDFs with CCFM}

Here another alternative approach to collinear DGLAP evolution is presented.
In high energy factorization \cite{Catani:1990eg} the measured cross section is written
 as a convolution of the partonic cross section $\hat{\sigma}(É \kt),$ which depends on the transverse 
momentum $\kt$ of the incoming parton, with the $\kt$-dependent parton distribution function 
${\cal \tilde A}\left(x,\kt,\Pmax\right)$ (transverse momentum dependent (TMD) or unintegrated uPDF):
\begin{equation}
 \sigma  = \int 
\frac{dz}{z} d^2k_t \hat{\sigma}(\frac{x}{z},k_t)  {\cal \tilde A}\left(x,\kt,\Pmax\right)
\label{kt-factorisation}
\end{equation}
{\bf would probably be good to explain how the unintegrated relates to the integrated here}
Generally, the evolution of ${\cal \tilde A}\left(x,\kt,\Pmax\right)$ 
can proceed via the BFKL{\bf you need a BFKL reference}, DGLAP or via the CCFM evolution equations.
In \fitter\, an extension of the CCFM~\cite{\CCFM} evolution has been implemented.
Since the evolution cannot be easily obtained in  a closed form, 
 first a kernel $ {\cal \tilde A}\left(x'',\kt,\Pmax\right) $ 
is determined from the MC solution of the CCFM evolution equation, 
and is then folded with a non-perturbative starting distribution 
${\cal A}_0 (x)$~\cite{Jung:2012hy}:
\begin{eqnarray}
%\begin{align}
 x {\cal A}(x,\kt,\Pmax) & = & 
   x\int dx' \int dx'' {\cal A}_0 (x) {\cal \tilde A}\left(x'',\kt,\Pmax\right)  \delta(x' \cdot x'' - x) \nonumber \\  
 & = & \int dx' \int dx'' {\cal A}_0 (x) {\cal \tilde A}\left(x'',\kt,\Pmax\right) \frac{x}{x'} \delta(x'' - \frac{x}{x'}) \nonumber \\ 
 & = & \int dx' {{\cal A}_0 (x') }  \cdot \frac{x}{x'}{ {\cal \tilde A}\left(\frac{x}{x'},\kt,\Pmax\right). } 
%\end{align}
\end{eqnarray}
%An intrinsic $\kt$ dependence is included in the kernel ${\cal \tilde A}$
%\begin{eqnarray}
%{\cal \tilde A} & = & {\cal \tilde A'} \cdot f(k_{t\;0}) = {\cal \tilde A'} \cdot  \exp\left[ 
%-\frac{(\mu-k_{t\;0})^2}{\sigma^2}\right]
%\end{eqnarray}
The kernel  ${\cal \tilde A}$ includes all the dynamics of the evolution,
 Sudakov form factors and splitting functions and is determined in 
a grid of $50\otimes50\otimes50$ bins in $x,\kt,\Pmax$.  

The calculation of the cross section according to Eq.(\ref{kt-factorisation})
 involves a multidimensional Monte Carlo integration which is time consuming
 and suffers from numerical fluctuations, and therefore cannot be used directly in a fit
 procedure.
% involving the calculation of numerical derivatives in the search for a minimum. 
Instead the following procedure is applied:
\begin{eqnarray}
\nonumber
\sigma_r(x,Q^2) & = & \int_x^1 d x_g {\cal A}(x_g,\kt,\Pmax) \hat{ \sigma}(x,x_g,Q^2) \\
  & = & \int_x^1 dx' {\cal A}_0 (x') \cdot \tilde{ \sigma}(x/x',Q^2). 
  \label{final-convolution}
\end{eqnarray}

The kernel ${\cal \tilde A}$ has to be provided separately and is not
 calculable within the program. A starting distribution  ${\cal A}_0$, 
 at the starting scale $Q_0$, of the following form is used:
\begin{eqnarray}
x{\cal A}_0(x,\kt) &=& N x^{-B_g} \cdot (1 -x)^{C_g}\left( 1 -D_g x\right) 
\label{a0}
\end{eqnarray}
with free parameters $N,\, B_g,\, C_g,\, D_g$. 
%In the present version, only the transverse momentum dependent gluon distribution 
%can be obtained from the fit. 

The calculation of the $ep$ cross section follows eq.(\ref{kt-factorisation}), 
with the off-shell matrix element including quark masses taken from \cite{Catani:1990eg} 
in its implementation in {\tt CASCADE} \cite{Jung:2010si}.
In addition to the boson gluon fusion process, valence quark initiated 
$\gamma q\to q$ processes are included, with the valence quarks taken from~\cite{Deak:2010gk}.


\subsection{Diffractive PDFs}

\newcommand{\asotp}{\ensuremath{\frac{\alpha_{\rm s}}{2\pi}}}
\newcommand{\Sgl}[1]{\ensuremath{\tilde f_{#1+}}}
\newcommand{\Pom}{{I\!P}}
\newcommand{\Reg}{{I\!R}}
\newcommand{\xpom}{$x_{I\!P}$}


Similarly to standard DIS, diffractive parton distributions (DPDFs) 
can be derived from QCD fits to diffractive cross sections.
%In this section the diffractive process is briefly described.
At HERA about 10\% of deep inelastic interactions are diffractive leading to
events in which the interacting proton stays intact ($ep\to eXp$). 
In the diffractive process the proton appears well separated from the 
rest of the hadronic final state by a large rapidity gap  
and this is interpreted as the diffractive dissociation 
of the exchanged virtual photon to produce a hadronic system $X$ with mass much 
smaller than $W$ and the same net quantum numbers as the exchanged photon.
%Figure~\ref{fig:diff} illustrates the kinematic variables used to describe
%the inclusive diffractive DIS process. 
For such processes, the proton vertex factorisation approach
is assumed where diffractive DIS is mediated by the exchange of hard Pomeron 
or a
secondary Reggeon. 
The factorisable pomeron picture has proved remarkably successful in the description of most of these data.
%
%\begin{figure}[!ht]
%\begin{center}
%\includegraphics[width=0.5\linewidth]{figures/diffraction.pdf}
%\end{center}
%\caption{Schematic diagram of the kinematic variables used to
% describe the inclusive diffractive DIS process.}
%\label{fig:diff}
%\end{figure}

In addition to the usual variables $x$, $Q^2$, one must consider the squared four-momentum transfer $t$
(the undetected momentum transfer to the proton system) and
the mass $M_X$ of the diffractively produced final state. 
In practice, the variable $M_X$ 
is often replaced by $\beta=\frac{Q^2}{M_X^2+Q^2-t}$.
%
In models based on a factorisable Pomeron, $\beta$ may be viewed as the fraction of the
pomeron longitudinal momentum which is carried by the struck parton, $x=\beta x_{\Pom}$.
%The diffractive parton distribution functions (DPDFs) are interpreted as probabilities for 
%finding a parton with a small fraction of the proton momentum $x=\beta\Pom$

For the inclusive case, the diffractive cross-section can be expressed as:
\begin{equation}
\begin{array}{lcl}
  \frac{d\sigma}{d\beta\,dQ^2dx_{\Pom}\,dt}
=
  \frac{2\pi\alpha^2}{\beta Q^4}\,
    \left( 1 +  (1-y)^2 \right) \ensuremath{\overline\sigma}^{D(4)}(\beta,Q^2,x_{\Pom},t)
\label{Dxs}
\end{array}
\end{equation}
where the ``reduced cross-section'' , $\overline\sigma$, is defined as
\begin{equation}
\begin{array}{lcl}
\overline\sigma^{D(4)}
 = F_2^{D(4)} - \frac{y^2}{1 +  (1-y)^2}\, F_L^{D(4)}
 = F_T^{D(4)} + \frac{2(1-y)}{1 +  (1-y)^2}\, F_L^{D(4)}.
\label{eq:sigred}
\end{array}
\end{equation}
%The dimension of $F_k^{D(4)}(\beta,Q^2,x_{\Pom},t)$
%is $GeV^{-2}$ and thus quantities integrated over $t$.
%\begin{equation}
%F_k^{D(3)}(\beta,Q^2,x_{\Pom})
%\equiv
%\int_{t_{\rm min}}^{t_{\rm max}} dt
%F_k^{D(4)}(\beta,Q^2,x_{\Pom},t)
%\end{equation}
%are dimensionless. The maximum kinematically allowed value of $t$ is given by
%\begin{equation}
%t_{\rm MAX} 
%=
%-\frac{x_{\Pom}^2 m_p^2 + p_\perp^2}{1-x_{\Pom}}
%\approx 
%-\frac{x_{\Pom}^2}{1-x_{\Pom}} m_p^2
%\end{equation}
%where $m_p$ is the proton mass.
With $x = x_{\Pom}\beta$ we can relate this to the standard DIS formula.
%\begin{equation}
%\begin{array}{lcl}
%\frac{d\sigma}{d\beta\,dQ^2\,dx_{\Pom}\,dt} =
%  \frac{2\pi\alpha^2}{x\, Q^4}\,
%    \left( 1 +  (1-y)^2 \right) x_{\Pom}\ensuremath{\overline\sigma}^{D(4)}(\beta,Q^2,x_{\Pom},t)
%\end{array}
%\end{equation}
%which upon integration over $t$ reads
%\begin{equation}
%\begin{array}{lcl}
%\label{Dxs3}
%  \frac{d\sigma}{d\beta\,dQ^2\,dx_{\Pom}}
%=  
%  \frac{2\pi\alpha^2}{x Q^4}\,
%    \left( 1 +  (1-y)^2 \right) \,x_{\Pom}\ensuremath{\overline\sigma}^{D(3)}(\beta,Q^2,x_{\Pom}).
%\end{array}
%\end{equation}
%%The H1 and ZEUS data files typically contain $x_{\Pom}\ensuremath{\overline\sigma}^{D(3)}$.
The diffractive structure functions can be expressed as convolutions of the
calculable coefficient functions with diffractive quark and gluon distribution functions,
 which in general depend on all of \xpom, $Q^2$, $\beta$, $t$.
\\
\\
%==========================================
%{\bf Regge factorization} 
%Needed? \\
The diffractive PDFs in \fitter are implemented following the prescription of ZEUS
publication~\cite{zeus:diff2009} and can be used to reproduce the main results.
%For a  better description of data, a contribution from a secondary Reggeon, $\Reg$, is included, hence
%\begin{equation}
%F_k^{D(4)}(\beta,Q^2,x_{\Pom},t) = 
%\sum_{\mathcal{X} =\Pom,\Reg}
%\phi_\mathcal{X}(x_{\Pom},t)\, F^\mathcal{X}_k(\beta,Q^2)
%\end{equation}
%or
%\begin{equation}
%\label{eq:FD3}
%F_k^{D(3)}(\beta,Q^2,x_{\Pom}) = 
%\sum_{\mathcal{X} =\Pom,\Reg}
%\Phi_\mathcal{X}(x_{\Pom})\, F^\mathcal{X}_k(\beta,Q^2)
%\end{equation}
%where
%\begin{equation}
%\label{eq:intFlux}
%\Phi_{\mathcal{X}}(x_{\Pom}) =
%\int\limits_{t_{\rm min}}^{t_{\rm max}} dt\, \phi_\mathcal{X}(x_{\Pom},t)
%\,.
%\end{equation}
%The fluxes are parametrized as
%\begin{subequations}
%\label{eq:flux}
%\begin{equation}
%\phi_\mathcal{X}(x_{\Pom},t) = 
%\frac {A_\mathcal{X}\, e^{b_\mathcal{X} t}} {x_{\Pom}^{2\alpha_\mathcal{X}(t) -1}}
%\end{equation}
%where
%\begin{equation}
%\alpha_\mathcal{X}(t) = \alpha_\mathcal{X}(0) + \alpha_\mathcal{X}' t
%\,.
%\end{equation}
%\end{subequations}
%The function $F^\Reg_k(\beta,Q^2)$  is taken to be that of the pion.
%