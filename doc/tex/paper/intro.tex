
\label{sec:intro}
Deep inelastic lepton-nucleon scattering (DIS) plays a pivotal role in
determining the structure of the proton. 
%
The electron-proton collider HERA
covers a wide range of squared four-momentum transfers, $Q^2$, and 
Bjorken $x$.
%
Previous measurements
of the DIS cross section, performed 
by the H1~\cite{H1:2009bp,Aaron:2009kv,h1alphas,Adloff:1999ah,Adloff:2000qj,Adloff:2003uh} 
and ZEUS~\cite{Breitweg:1997hz,Breitweg:2000yn,Breitweg:1998dz,Chekanov:2001qu,zeuscc97,
Chekanov:2002ej,Chekanov:2002zs,Chekanov:2003yv,Chekanov:2003vw}
 experiments, using data at proton beam energies of
$E_p=820$~GeV and $E_p=920$~GeV and a lepton beam energy of 
$E_e=27.5$~GeV,
as well as the combination of their analyses~\cite{h1zeus:2009wt},
have enabled studies of 
perturbative Quantum Chromodynamics (QCD) with unprecedented precision. 
%
These measurements has been complemented with data including the 
data taken at $E_p=460$~GeV and $E_p=575$~GeV (LINK FL - Dec 2010).
%

This paper provides a comprehensive description of the \fitter\ package,
which is designed for analysis of the High Energy Physics data.
The package has been developed by members of the H1 collaboration
with an exclusive support of different theoretical groups.
The main purpose of the \fitter\ package is analysis of the information
obtained from the deep inelastic scattering experiments
and the determination of the parton density functions.
The broad range of data taken from the $e^{\pm}p$, $p\bar{p}$, and $pp$ collider experiments can be
studied by the package. The inclusive cross section and jet data,
rapidity and asymmetry data can be studied using the package.
In this paper the reference fit results of
the HERA 1.0 data~\cite{h1zeus:2009wt} are presented.

The outline of this paper is as follows.
\emph{The manulal begins with a brief discussion of the theoretical calculation
used in the program (section~\ref{sec:theory}) followed by description of the
PDF parameterisation (section~\ref{sec:pdf}) and the $\chi^2$ function used in the
minimisation (section~\ref{sec:chi2}). The installation instructions are given in
section~(link). A description of the program steering cards and
the output options is given in section~(link).}

