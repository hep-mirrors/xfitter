
\label{sec:summary}
\fitter is the first open-source platform designed for studies of the structure of the proton.
It provides a unique and flexible framework with a wide variety of QCD tools to 
facilitate analyses of the experimental data and theoretical calculations. 
%\fitter allows for direct comparisons of various theoretical approaches under the same settings,
%different methodologies in treating the experimental and model uncertainties can be used for benchmarking studies.

The \fitter code, in version $1.1.0$, has sufficient options to reproduce the different 
theoretical choices made in MSTW, CTEQ and ABM fits. This will potentially make it a  
valuable tool for benchmarking and understanding differences between PDF fits. 
Such a study would however need to consider a range of further questions, such as the choices of
data sets, treatments of uncertainties, input parameter values, $\chi^2$ definitions, etc. 
\\
The further progress of \fitter is driven by the latest QCD advances in theoretical calculations 
and in the precision of experimental data.


%We look forward to studying these questions in future work.


