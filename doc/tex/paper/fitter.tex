\documentclass[12pt,a4paper,dvips]{article}
\usepackage{graphicx,epsfig}
\usepackage{hhline}

\usepackage{amsmath,amssymb}
\usepackage{times}
\usepackage[varg]{txfonts}
\DeclareMathAlphabet{\mathbold}{OML}{txr}{b}{it}

\usepackage{array,multirow,dcolumn}
\usepackage[mathlines,displaymath]{lineno}
\usepackage{rotating}

 
% we use natbib instead of cite to work with hyperref
%\usepackage{cite}
\usepackage[numbers,square,comma,sort&compress]{natbib}
\usepackage{hypernat}
\usepackage{textcomp}
%\bibliographystyle{unsrt}
%\bibliographystyle{unsrtnat}
%\bibliographystyle{apsrev}
 \bibliographystyle{fitter}
%\bibliographystyle{JHEP-2}



%\usepackage{caption2}
%\renewcommand{\captionfont}{\normalfont\slshape\normalsize}
%\renewcommand{\captionlabelfont}{\normalfont\bfseries\normalsize}
\newcommand{\tablecaption}{%
%\setlength{\abovecaptionskip}{0pt}
%\setlength{\belowcaptionskip}{10pt}
\caption}

\newcommand{\D}{\displaystyle}
\newcolumntype{.}{D{.}{.}{-1}}
\newcolumntype{-}{D{-}{-}{-1}}

\usepackage[dvipsnames]{color}
\definecolor{rltred}{rgb}{0.75,0,0}
\definecolor{rltgreen}{rgb}{0,0.5,0}
\definecolor{rltblue}{rgb}{0,0,0.5}

\newcounter{pdfadd}    % need for correct PDF hyperlinks and bookmarks :-(
\usepackage[hyperindex,bookmarks,bookmarksnumbered,breaklinks,a4paper,unicode]{hyperref}
\hypersetup{%
  pdftitle        = {Measurement of the inclusive ep Scattering Cross Section
 at low Q2 and x at HERA},
  urlcolor        = rltblue,       % \href{...}{...} external (URL)
  urlbordercolor  = 0 0 0.5,
  filecolor       = rltblue,       % \href{...} local file
  filebordercolor = 0 0 0.5,
  linkcolor       = rltred,        % \ref{...}
  linkbordercolor = 0.75 0 0,
  citecolor       = rltgreen,      % \cite{...}
  citebordercolor = 0 0.5 0,
  pagecolor       = rltgreen,      % \pageref{...}
  pagebordercolor = 0 0.5 0,
  menucolor       = rltgreen,      % Acrobat menu items
  menubordercolor = 0 0.5 0,
  colorlinks    = true,
  pdfauthor     = {H1 Collaboration},
  pdfsubject    = { },
  pdfkeywords   = {High-Energy Physics, Particle Physics, Proton Structure, DIS}
}

\renewcommand{\topfraction}{1.0}
\renewcommand{\bottomfraction}{1.0}
\renewcommand{\textfraction}{0.0}
\newlength{\dinwidth}
\newlength{\dinmargin}
\setlength{\dinwidth}{21.0cm}
\textheight24cm \textwidth16.0cm
\setlength{\dinmargin}{\dinwidth}
\setlength{\unitlength}{1mm}
\addtolength{\dinmargin}{-\textwidth}
\setlength{\dinmargin}{0.5\dinmargin}
\oddsidemargin -1.0in
\addtolength{\oddsidemargin}{\dinmargin}
\setlength{\evensidemargin}{\oddsidemargin}
\setlength{\marginparwidth}{0.9\dinmargin}
\marginparsep 8pt \marginparpush 5pt
\topmargin -42pt
\headheight 12pt
\headsep 30pt \footskip 32pt
\parskip 3mm plus 2mm minus 2mm

%
% Definitions for F2 low Q2 paper
%
\newcommand{\empz}{\mbox{$E$$-$$P_z$}}
\newcommand{\qqe}{\mbox{$Q^2_e$}}
\newcommand{\qqs}{\mbox{$Q^2_\Sigma$}}
\newcommand{\ys}{\mbox{$y_\Sigma$}}
\newcommand{\xs}{\mbox{$x_\Sigma$}}
\newcommand{\ee}{\mbox{$E_e^{\prime}$}}
\newcommand{\thetae}{\mbox{$\theta_e$}}
\newcommand{\thetah}{\mbox{$\theta_h$}}
\newcommand{\gp}{\mbox{$\gamma p$}}
\newcommand{\pt}{\mbox{$P_\perp$}}
\newcommand{\piz}{\mbox{$\pi^0$}}
\newcommand{\Pth}{\mbox{$P_{\perp}^h$}}

\newcommand{\electron}{\mbox{positron}}

\newcommand{\Fig}{\mbox{figure}}
\newcommand{\Tab}{\mbox{table}}
\newcommand{\Eq}{\mbox{equation}}
\newcommand{\Sec}{\mbox{section}}

\newcommand{\FFig}{\mbox{Figure}}
\newcommand{\TTab}{\mbox{Table}}
\newcommand{\EEq}{\mbox{Equation}}
\newcommand{\SSec}{\mbox{Section}}

\def\ytrans{0.56}

\newcommand{\Figs}{\mbox{figures}}
\newcommand{\Tabs}{\mbox{tables}}
\newcommand{\Eqs}{\mbox{equations}}
\newcommand{\Secs}{\mbox{sections}}

\newcommand{\FFigs}{\mbox{Figures}}
\newcommand{\TTabs}{\mbox{Tables}}
\newcommand{\EEqs}{\mbox{Equations}}
\newcommand{\SSecs}{\mbox{Sections}}


\renewcommand{\perp}{{\rm T}}

\newcommand{\MB}{\mbox{NVX}}
\newcommand{\SVX}{\mbox{SVX}}

\newcommand{\thetamaxsvx}{178^{\circ}}
\newcommand{\thetamaxmb}{176.5^{\circ}}

\newcommand{\permil}{\textperthousand}

\def\cov{\mathop{\rm cov}\nolimits}
\def\dof{\mathop{n_{\rm dof}}\nolimits}

\def\DeltaM{\mathop{ w_{i,e}}\nolimits}
\def\DeltaMK{\mathop{ w_{k,e}}\nolimits}

\newcommand\T{\rule{0pt}{2.3ex}}
\newcommand\B{\rule[-1.ex]{0pt}{0pt}}

\newcommand\hqcjc{\mbox{medium $Q^2$ CJC}}
\newcommand\lqbst{\mbox{low $Q^2$ BST}}
\newcommand\ler{\mbox{reduced $E_p=460$~GeV}}
\newcommand\mer{\mbox{reduced $E_p=575$~GeV}}
\newcommand\lermer{reduced $E_p=460$~GeV and $E_p=575$~GeV}
\newcommand\ner{nominal $E_p=920$~GeV}

\newcommand\Hqcjc{\mbox{Medium $Q^2$ CJC}}
\newcommand\Lqbst{\mbox{Low $Q^2$ BST}}
\newcommand\Ler{\mbox{Reduced $E_p=460$~GeV}}
\newcommand\Mer{\mbox{Reduced $E_p=575$~GeV}}
\newcommand\Lermer{Reduced $E_p=460$~GeV and $E_p=575$~GeV}


\newcommand\lumicjcplus{\mbox{$53.2$}}
\newcommand\lumicjcminus{\mbox{$44.4$}}
\newcommand\lumibstplus{\mbox{$3.4$}}
\newcommand\lumibstminus{\mbox{$2.5$}}

\newcommand\lunit{\mbox{pb$^{-1}$}}
\newcommand\lumiler{\mbox{$12.2$}}
\newcommand\lumimer{\mbox{$5.9$}}
\newcommand\lumicjc{\mbox{$97.6$}}
\newcommand\lumibst{\mbox{$5.9$}}
\newcommand\gbw{GBW}
\newcommand\gbwdglap{GBW+DGLAP$_{\rm valence}$}
\newcommand\iim{IIM}
\newcommand\iimdglap{IIM+DGLAP$_{\rm valence}$}
\newcommand\bsat{B-SAT}
\newcommand\bsatdglap{B-SAT+DGLAP$_{\rm valence}$}

% acot/rt etc

\newcommand\ndfdefone{782}
\newcommand\chacotone{722.7}
\newcommand\chrtone{773.2}
\newcommand\ndfdef{781}
\newcommand\chacot{715.2}
\newcommand\chrt{764.5}

\newcommand\chrtqcut{288.8}
\newcommand\chacotqcut{248.3}
\newcommand\ndfdglapqcut{249}

% comes automatically ...
%\newcommand\iimchsq{$397.6/352$}
%\newcommand\gbwchsq{$718.8/352$}
%\newcommand\bsatchsq{$424.9/352$}

%\newcommand\iimqchsq{$259.4/252$}
%\newcommand\gbwqchsq{$559.7/252$}
%\newcommand\bsatqchsq{$xxx/252$}

%\newcommand\iimqdvchsq{$287.6/252$}
%\newcommand\gbwqdvchsq{$739.4/252$}
%\newcommand\bsatqdvchsq{$xxx/252$}

%%%%%%%%%%%%%%%%%%%%%%%%%%%%%%%
\section{$\chi^2$ Definitions}
\label{sec:chi2}
%%%%%%%%%%%

For a single data set with diagonal statistical uncertainties, 
 the $\chi^2$ function can be defined as~\cite{H1:2009bp}
%
\begin{equation}
 \chi^2_{\rm exp}\left(\boldsymbol{m},\boldsymbol{b}\right) = %\\
%~~~=
 \sum_i
 \frac{\left[m^i
- \sum_j \gamma^i_j m^i b_j  - {\mu^i} \right]^2}
{ \textstyle \delta^2_{i,{\rm stat}}\left(m^i -  \sum_j \gamma^i_j m^i b_j\right)+
\left(\delta_{i,{\rm uncor}}\,  m^i\right)^2}
 + \sum_j b^2_j.
\label{eq:ave}\end{equation}
%
Here ${\mu^i}$ is the  measured central value  at a point $i$ 
with  relative statistical $\delta_{i,stat}$ 
and relative uncorrelated systematic uncertainty $\delta_{i,unc}$.
Further, $b_j$ denotes a nuisance parameter for
 a correlated systematic error  source of type $j$ with an uncertainty
 while
$\gamma^i_j$ 
quantifies the sensitivity of the
measurement ${\mu^i}$ at the point $i$ to the systematic source $j$. 
The function $\chi^2_{\rm exp}$ depends on the predicted values $m^i$ 
(denoted as the vector $\boldsymbol{m}$) and 
 the set of systematic uncertainties $b_j$ ($\boldsymbol{b}$).
The predicted values $m^i$ depend on the PDFs as well as other input
paremeter (e.g value of $\alpha_S$),  $m^i( \boldsymbol{p})$. 
In the following, absolute (relative) values of uncertainties are given
by capital (small) Greek symbols, e.g. $\Delta^i_{\rm stat}$ ( $\delta^i_{\rm stat} )$. 

This definition of the $\chi^2$ function assumes that
systematic uncertainties are proportional to the central values 
(multiplicative errors), whereas the statistical errors scale 
with the square roots of the expected number of events. 
Other scaling properties for the statistical and uncorrelated
systematic uncertainties are discussed later.
% available as described in appendix~\ref{sec:herafitter}.
%%%%

In the case of off-diagonal statistical uncertainties, the $\chi^2$ function
is
\begin{equation} \label{eq:chi2gen}
\chi^2_{\rm exp} (\boldsymbol{m},\boldsymbol{b}) = \sum_{ij} \left ( m^i - \sum_l \Gamma^i_l(m^i)b_l - \mu^i \right)
  C^{-1}_{{\rm stat.}~ij}(m^i,m^j) \left(  m^j - \sum_l \Gamma^j_l(m^j)b_l - \mu^j \right) + 
\sum_l b^2_l \,.
\end{equation}
Here the scaling properties of the correlated systematic uncertainties 
$\Gamma^i_j$ and
of the covariance matrix $C_{{\rm stat.}~ij}$ are expresses as a dependence
on $m_i$ and the dependence of $\Delta_{\rm stat}$ on $b_j$ is ignored.

Eq.~\ref{eq:chi2gen} allows for two methods for fast determination
of the minimum, without need to include the formal nuisance parameters
corresponding to the systematic error sources into the minuit minimisation.
In the first method, the minimisation vs. $b_j$ is used to define covariance
matrix for the systematic uncertainties which is determined as
\begin{equation}
 C_{{\rm syst}~ij}= \sum_l \Gamma^i_l \Gamma^j_l \,.
\end{equation}
The total covariance matrix is given by the sum of the statistical and
systamtic covariance matrices
\begin{equation} 
C_{{\rm tot}~ij} = C_{{\rm stat}~ij} + C_{{\rm syst}~ij}\,,
\end{equation}
and the $\chi^2$ function takes a form
\begin{equation}
  \chi^2( \boldsymbol{m}) = \sum_{ij} ( m^i - \mu^i) C^{-1}_{{\rm tot}~ij} 
( m^j - \mu^j)\,.
\end{equation}

The second methods is used to determine optimal shifts of the nuisance
parameters at each iteration. The shifts are given by minimising 
Eq.~\ref{eq:chi2gen} vs. $b_j$ which leads to a system of  linear equations 
\begin{equation}
 \sum_k \sum_{ij} C^{-1}_{{\rm stat}~ij} \Gamma^i_l \Gamma^j_k \cdot b_k = \sum_{ij} C^{-1}_{{\rm stat}~ij} \Gamma^i_l (m_i - \mu_i)\,,
\end{equation}
where $1\le l \le N_{\rm syst}$, the total number of correlated systematic uncertainties.

Finally the nuisance parameters $\boldsymbol{b}$ can be excluded from the $\chi^2$ minimisation.  
In this case, which is referred to as an Offset method, the minimum is determined for their values set to zero
while uncertainties on the parameters $\boldsymbol{p}$ are determined by shifting each nuisance parameter $b_l$
by $\pm 1$. The total covariance matrix for parameters $p^i$ is determined as 
\begin{equation}
  C^{\rm offset}_{ {\rm par}~ ij} = \sum_{l=1}^{N_{syst}} \Delta p^i_l \Delta p^j_l \,,
\end{equation}
where $ \Delta p^i_l = 0.5 ( p^i( b_l = +1 ) - p^i(b_l = -1))$ and the quality of the fit is estimated by 
fixing $\boldsymbol{p}$ to the value at the minimum and minimising with respect to $\boldsymbol{b}$

Finally, all three approaches can be combined together. For example, only some of the systematic uncertainties
can be treated using the matrix method while others can be treated using the hessian method. In this case, the
covariance matrix  $C_{\rm syst}$ is build using the corresponding sub-set of systematic sources and $C_{\rm stat}$ 
is replaced by $C_{\rm stat}+C_{\rm syst}$ in Eq.~\ref{eq:chi2gen}. Similarly, some of the systematic uncertainties
can be treated using offset method and then $C^{\rm total}_{ {\rm par}} = C^{\rm hessian}_{\rm par} + C^{\rm offset}_{\rm par}$
where offset and hessian covariance matrices are calculated using corresponding systematic error sources.

\subsection{Bias corrections}

The correlated and uncorrelated systematic uncertainties can be treated as additive,  $\Gamma^i_l(m^i) = \gamma^i_l \mu^i$
or multiplicative, $\Gamma^i_l(m^i) = \gamma^i_l m^i$. The LogNormal treatment in which 
$ \mu^i + \sum_l \Gamma^i_j b_l$ is replaced by $ \mu^i \prod_l \exp( \gamma^i_j b_l) $ is forseen for the
next release of the {\tt HERAFitter}. 

The statistical uncertainties can be treated as additive, $\Delta^i(m^i) = \delta^i \mu^i$  and as Poisson,
$\Delta^i(m^i) = \delta^i \sqrt{\mu^i m^i}$. More complex scaling from Eq.~\ref{eq:ave}, 
which depends on shifts of $b_j$, is implemented using an iterative approach: for the first iteration $b_l =0$ 
 is used to determine values of $b_l$ which are then applied in the second iteration. Statistical covariance
matrix is scaled in a similar manner. In this case the correlation matrix is assumed to be fixed, the diagonal
ellements are updated using the prescription describe above and the covariance matrix is rescaled accordingly.

The modifications of the covariance matrix at each iteration of the minuit minimisation may lead to systematic
biases. There are two approaches to avoid these biases. In the first approach the covariance matrix is calculated
using the expected values at the first iteration of the minimisation and kept fixed to these values for further
iterations. This method requires several repetitions of the minimisation, to ensure that values close to optimal
are obtained already at the first iteration. The second method modifies the $\chi^2$ function by adding a term
corresponding to non-constant value of the covariance matrix:
\begin{equation}
 \chi^2_{\rm log} = 2 \log \frac{\Delta^i(m^i)}{\Delta^i(\mu^i)} 
\end{equation}  

\subsection{HERAFitter implementation}

%%%%%%%%%%%
% \subsection{Using Covariance Matrix}
%%%%%%%%%%%%%%%%%%%%%%%%%%%%%



%\documentclass[prd,superscriptaddress,unsortedaddress,twocolumn,showpacs,floatfix]{revtex4}

%\documentclass[preprint,superscriptaddress,unsortedaddress,showpacs,preprintnumbers,floatfix]{revtex4}

%\usepackage{epsfig}
%\usepackage{dcolumn}
%\usepackage{amsmath}


\linenumbers           % Remove for final publication!!!!!
\newcommand\fitter{ \mbox{\tt HERA Fitter} }

\begin{document}

% change natbib spacing between numbers in citations
\makeatletter \def\NAT@space{} \makeatother


\begin{titlepage}
 
\noindent
DESY 12-XXX \hfill ISSN 0418-9833 \\
 Month 2012 \\

\begin{flushleft}
Editors: S.~Glazov, V.~Radescu., P.~Belov and all, all, all.\\
Referees: \\

Version 0.0.1 \\
 13 December 2011
\end{flushleft}


\vspace*{3.5cm}

\begin{center}
\begin{Large}

{\bfseries
\fitter\ - PDF Fitting package
}

\vspace*{2cm}

H1 Collaboration

\end{Large}
\end{center}

\vspace*{2cm}

\begin{abstract} \noindent
We present the \fitter\ package for extracting the parton density functions 
(PDFs) using the data from deep inelastic scattering (DIS) and other processes. 
The package provides broad possibilities for the interpetation
of the $e^{\pm}p$ and $pp$ data and can be interesting for the LHC community.
The inclusive cross section data collected by the H1 experiment have been analysed using \fitter\
and the HERA1.0 PDF set has been obtained.
%
%The \fitter\ program has been used to analyse the inclusive data collected by the H1 experiment
%and determine the HERA1.0 PDF set.%\cite{h1zeus:2009wt}.
%
%
%The PDFs are important for the calculation of the cross sections
%for the $ep$ and $pp$ colliders and thus required for interpetation
%of the data collected at the LHC. The \fitter\ program were used to
%determine the HERA1.0 PDF set~\cite{h1zeus:2009wt}.
%A measurement is presented of the inclusive neutral current
%$e^\pm p$ scattering cross section 
%using data collected by the H1 experiment at HERA
%during the years  2003-2007 
%with proton beam energies $E_p$ of $920$, $575$, and $460$~GeV.
%$E_p=920$~GeV, $E_p=575$~GeV 
%and $E_p=460$~GeV is presented.
%
%The kinematic range of the measurement covers low squared
%four-momentum transfers, $1.5$~GeV$^2<Q^2<120$~GeV$^2$,  small 
%values of Bjorken $x$, $2.9 \cdot 10^{-5}<x<0.01$,
%and extends to high inelasticity up to $y=0.85$.
%
%A determination of 
%The structure function $F_L$
%is measured by combining the new results with previously published H1 data 
%by the H1
%collaboration measured
% at $E_p=920$~GeV.
%
%The new measurements are used to test several phenomenological and
%QCD models applicable in this low $Q^2$ and low $x$ kinematic domain.
\end{abstract}

\vspace*{1.5cm}

\begin{center}
%{\slshape To be Submitted to JHEP}
\end{center}

\end{titlepage}

%\begin{flushleft}
%  \input{h1auts}
%\end{flushleft}
 
%\newpage
\tableofcontents
\newpage

\section{Introduction}
%\normalsize
In the era of the Higgs discovery and extensive searches
for signals of new physics at the LHC it is crucial
to have accurate Standard Model (SM) predictions for
hard scattering processes at the LHC.
The most common approach to calculate the SM cross sections for  
such reactions is to use collinear factorisation in perturbative QCD (pQCD):
%is with perturbative QCD using a (collinear) factorisation approach: 
%{\small
\begin{equation}
\small
\begin{array}{lcl}
\sigma^{pp\rightarrow H + X}(\alpha_s,\mu_r,\mu_f) & = &
\sum\limits_{a,b}\,  \int\limits_{0}\limits^{1} dx_1 \int\limits_{0}\limits^{1} dx_2\, f_a(x_1,\alpha_s,\mu_F) 
 f_b(x_2,\alpha_s,\mu_F)\\ 
& \times & \, \hat{\sigma}^{ab \rightarrow H + X}(x_1,x_2;\alpha_s,\mu_R,\mu_F).
%\sigma^{pp\rightarrow H + X}(\alpha_s,\mu_r,\mu_f) = 
%\sum_{a,b} \int_{0}^{1} dx_1 \int_{0}^{1} dx_2 f_a(x_1) 
% f_b(x_1,\alpha_s,\mu_F) \times \hat{\sigma}^{ab \rightarrow H + X}(x_1,x_2;\alpha_s,\mu_R,\mu_F)
\label{eq:fact}
\end{array}
\end{equation}
%}
Here the cross section $\sigma^{pp\rightarrow H + X}$ for inclusive
Higgs production is expressed
as a convolution of Parton Distribution Functions (PDF) $f_a$ and $f_b$
with the partonic cross section
% that describe
%the 
$\hat{\sigma}^{ab \rightarrow H + X}$.
%
The PDFs describe 
the probability of finding a specific parton $a$ ($b$) in the first (second) proton carrying a fraction $x_1$ ($x_2$) of its momentum.
%
The sum over indices $a$ and $b$ in Eq.~\ref{eq:fact} indicates the various 
kinds of partons,
i.e. gluons and quarks and antiquarks of different flavours, 
that are considered
as the constituents of the proton.
%
Both the PDFs and the partonic cross section depend on the strong coupling
constant $\alpha_s$, and the factorisation and renormalisation scales,
$\mu_F$ and $\mu_R$, respectively.
%
The partonic cross sections are calculable in pQCD, but
the PDFs cannot yet be predicted in QCD, they must rather be 
determined from measuement. They are assumed 
to be universal such that different scattering reactions can be used 
to constrain them; in particular one can use specific reaction data 
for determining the PDFs and then use these PDFs for
predicting other processes.
% via Eq.~\ref{eq:fact}.
%

The Deep Inelastic Scattering (DIS) data from the $ep$ collider HERA provides crucial information for determining the PDFs.
%
For instance, the gluon density relevant
for calculating the dominant gluon-gluon fusion contribution to the Higgs production
at the LHC can be accurately determined from the HERA data alone.
%
%Despite being often plagued by larger perturbative uncertainties,
Specific data from the Tevatron $p\bar{p}$ and the LHC $pp$ collider
can help to further constrain the PDFs.
%
The most sensitive processes at the  colliders are
Drell Yan production, W and Z asymmetries, associated production of W or Z boson 
and heavy quarks, top quark production and jet production.
%

\fitter represents a QCD analysis framework that aims at 
determining precise PDFs by integrating all the PDF sensitive information
from HERA, the Tevatron and the LHC.
%
The processes that are currently included in \fitter framework are listed in Tab.~\ref{tab:proc}.
%
\begin{table}
\small
%\tiny
\scriptsize

\begin{tabular}{|l|l|l|l|}
\hline
Data &Type &  Reaction & Theory      \\
        &     &     & calculation \\
\hline

HERA &DIS NC   &$ep\to ep$ & QCDNUM, RT, ACOT \\
HERA &DIS CC   &$ep\to \nu_e p$ & QCDNUM, RT, ACOT\\
HERA &DIS jets &$ep\to eX$ & FastNLO (NLOJet++)\\
HERA &DIS heavy quark & $ep\to ep $& ZM (QCDNUM), RT, ACOT, \\
     &                     &            & FFNS (ABM,QCDNUM) \\
\hline
Fixed Target &DIS NC   &$ep\to ep$ & ZM (QCDNUM), RT, ACOT \\
\hline
Tevatron, LHC &Drell Yan &$pp(\bar p)$ & APPLGRID (MCFM) \\
Tevatron, LHC &W charge asym &$pp(\bar p)$ & APPLGRID (MCFM) \\
Tevatron, LHC &top &$pp(\bar p)$ & APPLGRID (MCFM),  \\
              &    &             & HATHOR \\
Tevatron, LHC &jets &$pp(\bar p)$ & APPLGRID (NLOJet++) \\
                &  & & FastNLO (NLOJet++) \\
LHC&  DY+heavy quark &$pp(\bar p)$ & APPLGRID (MCFM) \\
\hline
\end{tabular}
\caption{The list of processes available in the \fitter package. {\bf you need references for the theory predictions in this caption}}
\label{tab:proc}
\end{table}
%
\normalsize
The basic functionality of HERAFitter is shown in Fig.~\ref{fig:flow} and consists of four parts: %{\bf needs to update figure!}
\begin{figure}[!ht]
   \centering
   \includegraphics[width=8cm]{flow.pdf}
   \caption{Schematic structure of the \fitter program.} 
 \label{fig:flow}
\end{figure}
\begin{description}
\item 
\bf {Input data:} \rm  All relevant cross section data from the various reactions
are stored internally in \fitter with the full information on their uncorrelated and correlated
uncertainties.
\item
\bf{Theory predictions:} \rm Predictions are obtained relying on the factorisation approach (Eq.~\ref{eq:fact}). PDFs are parametrised at a starting scale $Q_0$  by a chosen functional form with a set of free parameters $\vec{p}$. They are then evolved from $Q_0$ to the scale of the measurement using the 
Dokshitzer-Gribov-Lipatov-Altarelli-Parisi 
(DGLAP)~\cite{Gribov:1972ri, Gribov:1972rt, Lipatov:1974qm,
Dokshitzer:1977sg, Altarelli:1977zs} evolution equations 
as implemented in QCDNUM~\cite{qcdnum}, 
and then convoluted (Eq.~\ref{eq:fact}) with the hard parton cross sections calculated by
a specific theory program (as listed in Tab.~\ref{tab:proc}).
\item
\bf{Minimization:} \rm  PDFs are extracted from a least square fit by constructing a 
$\chi^2$ from the input data and the theory prediction.
The $\chi^2$ is  minimized iteratively 
with respect to the PDF parameters using the MINUIT\cite{minuit} program.
%
%Fitted values of $\vec{p}$ and estimated uncertainties are obtained.
%The fitted parameters $\vec(p)$ and obtained from the uncertainties of the parameters are determined (from chi2+1???)
%
\item
\bf{Results:} \rm  The fitted parameters $\vec{p}$ and their estimated uncertainties are produced.
The resulting PDFs are provided in a format ready to be used by the LHAPDF 
library. Tools are supplied which allow the PDFs to be
graphically 
displayed at arbitrary scales with their one sigma uncertainty bands.
To demonstrate the fit consistency, plots 
which compare the input data to the fitted theory predictions can be made using
tools supplied with the package. 
This is illustrated in the Fig.~\ref{fig:data} showing  
HERA~I data (the default data set in \fitter) compared to predictions based on 
HERAPDF1.0\cite{h1zeus:2009wt}. This figure also illustrates this comparison 
taking into account the systematic uncertainty shift parameters which are 
applied to the predictions in the nuisance parameter method of accounting for 
correlated systematic uncertainties (see section~\ref{sec:chi2representation}) and the pulls 
{\bf you need to define exactly what you mean by a pull here, don't just leave it in the figure caption}.
\begin{figure}[!ht]
   \centering
   \includegraphics[width=8cm]{datatheory.pdf}
   \caption{An illustration of the \fitter drawing tools comparing the measurements (in this case HERA I) to the predictions of the fit. In addition, ratio plots are also provided together with the pull distribution (right panel).} 
 \label{fig:data}
\end{figure}

\end{description}
%
%This paper provides a comprehensive description of  \\
%the \fitter\ package.
%which is designed for analysis of the High Energy Physics data.
%The package has been developed by members of the H1 and ZEUS collaborations
%with an exclusive support of different theoretical groups.
%The main purpose of the \fitter\ package is analysis of the 
%data from the $e^{\pm}p$, $p\bar{p}$, and $pp$ collider experiments
%information obtained from the deep inelastic scattering experiments
%and the determination of the parton density functions (PDFs).
%The broad range of data taken from the $e^{\pm}p$, $p\bar{p}$, and $pp$ collider experiments can be
%studied by the package. 

%Based on the concept of factorisable nature of the cross sections into universal parton distribution functions (PDFs) and process dependent partonic scattering cross sections, 

The \fitter\ program facilitates the determination of the PDFs from many 
cross section measurements at $ep$, $p\bar{p}$ or $pp$ colliders.  
 It includes various options for theoretical calculations and various choices 
of how to 
account for the experimental uncertainties. Therefore, this project represents 
an ideal environment for benchmarking studies and a unique platform for the QCD interpretation of analyses within the LHC experiments,
as already demonstrated by several publicly available results using the \fitter\ 
framework~\cite{atlas:strange,atlas:jets,atlas:hm,cms:strange,cms:jets,h1:2012kk,h1zeus:charm}.  

The outline of this paper is as follows.
%
Section~\ref{sec:theory} discusses the various processes 
and corresponding theoretical calculations performed in the DGLAP~\cite{Gribov:1972ri,Gribov:1972rt,Lipatov:1974qm,
Dokshitzer:1977sg,Altarelli:1977zs} formalism that are available in \fitter.
Alternative approaches to the DGLAP formalism are presented in section~\ref{sec:alternative}.
%
In section~\ref{sec:techniques} various different choices made in the 
theory calculations are described.
Section~\ref{sec:method} elucidates the 
methodology of determining PDFs through fits based on various
% {\bf (what do you mean here
%by approaches?)} 
 $\chi^2$ definitions used in the
minimisation procedure. 
%
Specific applications of the package are given in
section ~\ref{sec:examples}. 
%
%{\bf add something more here?.}

%
\section{Theoretical Input}
\label{sec:theory}
%
\subsection{Evolution}

\label{sec:evolution}
The \fitter\ program uses DGLAP~\cite{Gribov:1972ri,Gribov:1972rt,Lipatov:1974qm,Dokshitzer:1977sg,Altarelli:1977zs}
 evolution 
equations as implemented in the QCDNUM~\cite{qcdnum} program. The fit 
procedure begins with parameterising the input PDFs at the starting 
scale $Q^2_0$ which should be chosen to be below the charm mass threshold
$m_C^2$.
The PDFs are then evolved using the DGLAP evolution equations  
at NLO~\cite{Curci:1980uw,Furmanski:1980cm} in the $\overline{MS}$ scheme.
The renormalisation and factorisation scales are set to $Q^2$. The \fitter\ program
also allows for LO and NNLO evolution. 

The cross-section predictions are obtain by convoluting the PDFS with the 
coefficient functions. For the DIS processes, those are calculated 
using the general mass variable-flavour scheme. 
The program implements the  zero mass scheme from QCDNUM as well as
the RT scheme~\cite{Thorne:1997ga,Thorne:2006qt}. The jet cross sections
are calculated using APPLGRID and FastNLO. The program has two implementations
for $pp$  DY processes. The first implementation uses
calculations at LO which can be extended to NLO using k-factors,
the second uses the APPLGRID interface.
 
%\subsection{Coupling}
%\input{coupling}
\subsection{PDF parametrization}

\label{sec:pdf}
The parton density functions are parametrised at the starting scale by two formulae.
The first one is the standart HERA parametrization
\begin{equation}
xf(x)=A x^{B} (1-x)^{C} (1+Dx+Ex^{2}+Fx^{3}) - A_{\mathbf{p}} x^{B_{\mathbf{p}}} (1-x)^{C_{\mathbf{p}}}
\label{pdf:herapara}
\end{equation}
and the second one is the parametrization used by CTEQ collaboration
\begin{equation}
xf(x)=A x^{B} (1-x)^{C} e^{Dx} (1+e^{E}x+e^{F}x^{2}).
\label{pdf:ctpara}
\end{equation}
These formulas are used for parametrization of the gluon distribution $xg$,
the valence quark distributions $xu_{v}$,$xd_{v}$,
and the $u$-type and $d$-type anti-quark distributions $x\bar{U}$, $x\bar{D}$.
Here $x\bar{U}=x\bar{u}$, $x\bar{D}=x\bar{d}+x\bar{s}$ at the starting scale.

By default, HERA parametrization fits include 10 or 13 free parameters.
The resulting parametrization with 10 parameters is presented by the formulae
\begin{eqnarray}
\begin{array}{c}
xg(x)=A_{g} x^{B_{g}} (1-x)^{C_{g}}, \\
xu_{v}(x)=A_{u_{v}} x^{B_{u_{v}}} (1-x)^{C_{u_{v}}} (1+E_{u_{v}}x), \\
xd_{v}(x)=A_{d_{v}} x^{B_{d_{v}}} (1-x)^{C_{d_{v}}}, \\
x\bar{U}(x)=A_{\bar{U}} x^{B_{\bar{U}}} (1-x)^{C_{\bar{U}}}, \\
x\bar{D}(x)=A_{\bar{D}} x^{B_{\bar{D}}} (1-x)^{C_{\bar{D}}}.
\end{array}
\end{eqnarray}
Here the parameters
\begin{eqnarray}
\begin{array}{c}
B_{u_{v}}=B_{d_{v}}, \\
B_{\bar{U}}=B_{\bar{D}},\\
A_{\bar{U}}=A_{\bar{D}}\frac{1-f_{s}}{1-f_{charm}},
\end{array}
\end{eqnarray}
and coefficients $A$ are defined by the quark number and momentum sum-rules
\begin{eqnarray}
\begin{array}{c}
\int\limits_{0}^{1} u_{v}(x) dx =2, \\
\int\limits_{0}^{1} d_{v}(x) dx =1, \\
\int\limits_{0}^{1} x \left( g(x)+u(x)+\bar{u}(x)+d(x)+\bar{d}(x) \right) dx =1. \\
\end{array}
\end{eqnarray}
The strange quark distribution is fixed by $x$-independent fraction, $f_{s}$,
of the d-type sea, $x\bar{s}=f_{s}x\bar{D}$ at $Q_{0}^{2}$.

In case of 13 free parameters, the gluon formula is changed to
\begin{equation}
xg(x)=A_{g} x^{B_{g}} (1-x)^{C_{g}} - A_{\mathbf{p}g} x^{B_{\mathbf{p}g}} (1-x)^{C_{\mathbf{p}g}},
\label{pdf:herag13par}
\end{equation}
and parameters $A_{\mathbf{p}g}$, $B_{\mathbf{p}g}$, and $B_{d_{v}}$ are freed.

\subsection{$\chi^2$ Definition}
%%%%%%%%%%%%%%%%%%%%%%%%%%%%%%
\section{$\chi^2$ Definitions}
\label{sec:chi2}
%%%%%%%%%%%

For a single data set with diagonal statistical uncertainties, 
 the $\chi^2$ function can be defined as~\cite{H1:2009bp}
%
\begin{equation}
 \chi^2_{\rm exp}\left(\boldsymbol{m},\boldsymbol{b}\right) = %\\
%~~~=
 \sum_i
 \frac{\left[m^i
- \sum_j \gamma^i_j m^i b_j  - {\mu^i} \right]^2}
{ \textstyle \delta^2_{i,{\rm stat}}\left(m^i -  \sum_j \gamma^i_j m^i b_j\right)+
\left(\delta_{i,{\rm uncor}}\,  m^i\right)^2}
 + \sum_j b^2_j.
\label{eq:ave}\end{equation}
%
Here ${\mu^i}$ is the  measured central value  at a point $i$ 
with  relative statistical $\delta_{i,stat}$ 
and relative uncorrelated systematic uncertainty $\delta_{i,unc}$.
Further, $b_j$ denotes a nuisance parameter for
 a correlated systematic error  source of type $j$ with an uncertainty
 while
$\gamma^i_j$ 
quantifies the sensitivity of the
measurement ${\mu^i}$ at the point $i$ to the systematic source $j$. 
The function $\chi^2_{\rm exp}$ depends on the predicted values $m^i$ 
(denoted as the vector $\boldsymbol{m}$) and 
 the set of systematic uncertainties $b_j$ ($\boldsymbol{b}$).
The predicted values $m^i$ depend on the PDFs as well as other input
paremeter (e.g value of $\alpha_S$),  $m^i( \boldsymbol{p})$. 
In the following, absolute (relative) values of uncertainties are given
by capital (small) Greek symbols, e.g. $\Delta^i_{\rm stat}$ ( $\delta^i_{\rm stat} )$. 

This definition of the $\chi^2$ function assumes that
systematic uncertainties are proportional to the central values 
(multiplicative errors), whereas the statistical errors scale 
with the square roots of the expected number of events. 
Other scaling properties for the statistical and uncorrelated
systematic uncertainties are discussed later.
% available as described in appendix~\ref{sec:herafitter}.
%%%%

In the case of off-diagonal statistical uncertainties, the $\chi^2$ function
is
\begin{equation} \label{eq:chi2gen}
\chi^2_{\rm exp} (\boldsymbol{m},\boldsymbol{b}) = \sum_{ij} \left ( m^i - \sum_l \Gamma^i_l(m^i)b_l - \mu^i \right)
  C^{-1}_{{\rm stat.}~ij}(m^i,m^j) \left(  m^j - \sum_l \Gamma^j_l(m^j)b_l - \mu^j \right) + 
\sum_l b^2_l \,.
\end{equation}
Here the scaling properties of the correlated systematic uncertainties 
$\Gamma^i_j$ and
of the covariance matrix $C_{{\rm stat.}~ij}$ are expresses as a dependence
on $m_i$ and the dependence of $\Delta_{\rm stat}$ on $b_j$ is ignored.

Eq.~\ref{eq:chi2gen} allows for two methods for fast determination
of the minimum, without need to include the formal nuisance parameters
corresponding to the systematic error sources into the minuit minimisation.
In the first method, the minimisation vs. $b_j$ is used to define covariance
matrix for the systematic uncertainties which is determined as
\begin{equation}
 C_{{\rm syst}~ij}= \sum_l \Gamma^i_l \Gamma^j_l \,.
\end{equation}
The total covariance matrix is given by the sum of the statistical and
systamtic covariance matrices
\begin{equation} 
C_{{\rm tot}~ij} = C_{{\rm stat}~ij} + C_{{\rm syst}~ij}\,,
\end{equation}
and the $\chi^2$ function takes a form
\begin{equation}
  \chi^2( \boldsymbol{m}) = \sum_{ij} ( m^i - \mu^i) C^{-1}_{{\rm tot}~ij} 
( m^j - \mu^j)\,.
\end{equation}

The second methods is used to determine optimal shifts of the nuisance
parameters at each iteration. The shifts are given by minimising 
Eq.~\ref{eq:chi2gen} vs. $b_j$ which leads to a system of  linear equations 
\begin{equation}
 \sum_k \sum_{ij} C^{-1}_{{\rm stat}~ij} \Gamma^i_l \Gamma^j_k \cdot b_k = \sum_{ij} C^{-1}_{{\rm stat}~ij} \Gamma^i_l (m_i - \mu_i)\,,
\end{equation}
where $1\le l \le N_{\rm syst}$, the total number of correlated systematic uncertainties.

Finally the nuisance parameters $\boldsymbol{b}$ can be excluded from the $\chi^2$ minimisation.  
In this case, which is referred to as an Offset method, the minimum is determined for their values set to zero
while uncertainties on the parameters $\boldsymbol{p}$ are determined by shifting each nuisance parameter $b_l$
by $\pm 1$. The total covariance matrix for parameters $p^i$ is determined as 
\begin{equation}
  C^{\rm offset}_{ {\rm par}~ ij} = \sum_{l=1}^{N_{syst}} \Delta p^i_l \Delta p^j_l \,,
\end{equation}
where $ \Delta p^i_l = 0.5 ( p^i( b_l = +1 ) - p^i(b_l = -1))$ and the quality of the fit is estimated by 
fixing $\boldsymbol{p}$ to the value at the minimum and minimising with respect to $\boldsymbol{b}$

Finally, all three approaches can be combined together. For example, only some of the systematic uncertainties
can be treated using the matrix method while others can be treated using the hessian method. In this case, the
covariance matrix  $C_{\rm syst}$ is build using the corresponding sub-set of systematic sources and $C_{\rm stat}$ 
is replaced by $C_{\rm stat}+C_{\rm syst}$ in Eq.~\ref{eq:chi2gen}. Similarly, some of the systematic uncertainties
can be treated using offset method and then $C^{\rm total}_{ {\rm par}} = C^{\rm hessian}_{\rm par} + C^{\rm offset}_{\rm par}$
where offset and hessian covariance matrices are calculated using corresponding systematic error sources.

\subsection{Bias corrections}

The correlated and uncorrelated systematic uncertainties can be treated as additive,  $\Gamma^i_l(m^i) = \gamma^i_l \mu^i$
or multiplicative, $\Gamma^i_l(m^i) = \gamma^i_l m^i$. The LogNormal treatment in which 
$ \mu^i + \sum_l \Gamma^i_j b_l$ is replaced by $ \mu^i \prod_l \exp( \gamma^i_j b_l) $ is forseen for the
next release of the {\tt HERAFitter}. 

The statistical uncertainties can be treated as additive, $\Delta^i(m^i) = \delta^i \mu^i$  and as Poisson,
$\Delta^i(m^i) = \delta^i \sqrt{\mu^i m^i}$. More complex scaling from Eq.~\ref{eq:ave}, 
which depends on shifts of $b_j$, is implemented using an iterative approach: for the first iteration $b_l =0$ 
 is used to determine values of $b_l$ which are then applied in the second iteration. Statistical covariance
matrix is scaled in a similar manner. In this case the correlation matrix is assumed to be fixed, the diagonal
ellements are updated using the prescription describe above and the covariance matrix is rescaled accordingly.

The modifications of the covariance matrix at each iteration of the minuit minimisation may lead to systematic
biases. There are two approaches to avoid these biases. In the first approach the covariance matrix is calculated
using the expected values at the first iteration of the minimisation and kept fixed to these values for further
iterations. This method requires several repetitions of the minimisation, to ensure that values close to optimal
are obtained already at the first iteration. The second method modifies the $\chi^2$ function by adding a term
corresponding to non-constant value of the covariance matrix:
\begin{equation}
 \chi^2_{\rm log} = 2 \log \frac{\Delta^i(m^i)}{\Delta^i(\mu^i)} 
\end{equation}  

\subsection{HERAFitter implementation}

%%%%%%%%%%%
% \subsection{Using Covariance Matrix}
%%%%%%%%%%%%%%%%%%%%%%%%%%%%%
 
%
%
\section{Data}
\label{sec:data}
%
\subsection{Data Sets}

\label{sec:data}
The following datasets are used for fits...

%\subsection{Overview of Systematic Uncertainties}
%\input{systematics}
%\section{Fits}
%\label{sec:fits}
%\subsection{DGLAP Fit}
%\label{sec:qcdfit}
%\input{qcd}
%\subsection{Dipole Model Fits}
%\input{dipole}
%% no discussion ...
%% \subsection{Discussion of Fit Results}
%% \input{discussion}
%%\subsubsection{Framework and Parameterisations}
%%\input{qcdset}
%%\subsubsection{Results and Kinematic Cut Variations}
%%\input{qcdresult}
%%
%%\subsection{Discussion}
%%
\section{Summary}
\label{sec:conclusion}

\label{sec:summary}
\fitter is an open-source platform designed for studies of the structure of the proton.
It provides a unique and flexible framework with a wide variety of QCD tools to 
facilitate analyses of the experimental data and theoretical calculations. 
\fitter allows for direct comparisons of various theoretical approaches under the same settings. 
Different methodologies in treating the experimental and model uncertainties and can be used for benchmarking studies.
The progress of \fitter is driven by the latest QCD advances in theoretical calculations and in the precision of experimental data.



%
\section*{Acknowledgements}
\refstepcounter{pdfadd} \pdfbookmark[0]{Acknowledgements}{s:acknowledge}
%\begin{theacknowledgments}    
%
We are grateful to the HERA machine group whose outstanding
efforts have made this experiment possible.
We thank the engineers and technicians for their work in constructing
and maintaining the H1 detector, our funding agencies for
financial support, the DESY technical staff for continual assistance
and the DESY directorate for support and for the hospitality
which they extend to the non DESY members of the collaboration.
%\end{theacknowledgments}


\bibliography{fitter.bib}  
\clearpage

%\input{tables}

%\clearpage

%\input{figures}


\end{document}

