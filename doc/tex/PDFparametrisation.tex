
\label{sec:pdfparam}
%%%%%%%%%%%
\subsection{Standard Functional form}
%%%%
Through standard functional form it is undertstood a simple polynomial 
that interpolates between the low and high $x$ regions:
\begin{equation}
 xf(x) = A x^{B} (1-x)^{C} P_i(x),
\label{eqn:pdf_std}
\end{equation}
We identify few standard forms commonly used by PDF groups.

%%%%
\subsubsection{CTEQ style}
%%%%
The notation used throughout this text reflects the 
notation used in the code.

\begin{equation}
 xf(x) = a_0 x^{(a_1+n)} (1-x)^{a_2} e^{a_3x} (1 + e^{a_4 x} + e^{a_5 x^2}),
\label{eqn:pdf_cteq}
\end{equation}
%
%%%%
\subsubsection{HERAPDF style}
%%%%
 The parametrised PDFs at HERA are the valence distributions
 $xu_v$ and  $xd_v$,  the gluon distribution $xg$, and the $u$-type and $d$-type 
$x\bar{U}$, $x\bar{D}$, where $x\bar{U} = x\bar{u}$, 
$x\bar{D} = x\bar{d} +x\bar{s}$. 
The following standard functional form is used to parametrise them
\begin{equation}
 xf(x) = A x^{B} (1-x)^{C} (1 + D x + E x^2),
\label{eqn:pdf}
\end{equation}
%
where the normalisation parameters, $A_{uv}, A_{dv}, A_g$,  are constrained by  
the QCD sum-rules, such that the counting  and  momentum conservation are preserved.
The $B$ parameters  $B_{\bar{U}}$ and $B_{\bar{D}}$ are set equal,
 $B_{\bar{U}}=B_{\bar{D}}$, such that 
there is a single $B$ parameter for the sea distributions. 
%
The strange quark distribution 
is already present at the starting scale and 
%
it is  assumed here that 
$x\bar{s}= f_s  x\bar{D}$ at $Q^2_0$. 
The  strange fraction is chosen to be $f_s=0.31$ which is
consistent with determinations 
of this fraction using neutrino induced di-muon production. 
%
In addition, to ensure that $x\bar{u} \to x\bar{d}$ 
as $x \to 0$,  
$A_{\bar{U}}=A_{\bar{D}} (1-f_s)$.
%
The $D$ and $E$ are introduced one by one until no further improvement in $\chi^2$ is found.
For the case when adding more precision data in the fit, as when adding HERA II data, this allows then for use of a more flexible parametrisation for the gluon and valence especially.
The best fit  results in a total of 10 free parameters when performing fits to solely HERA I data (fits are refered then ro as HERAPDF1.0), and of 13 free parameters when adding preliminary HERA II data on top (fits are refered then to as HERAPDF1.5).
\subsubsection{Flexible style}
%%%%
\subsection{Chebyshev Polynomial}

A flexible Chebyshev polynomials based parameterisation is used for the gluon and sea densities. The polynomials
use $\log x$ as an argument to emphasise the low $x$ behaviour. 
The parameterisation is valid for $x>x_{min} = 1.7\times 10^{-5}$. The PDFs are multiplied
by $1-x$ to ensure that they vanish as $x\to 1$. The resulting parameterisation form is 
\begin{eqnarray}
x g(x) &=& A_g \left(1-x\right) \sum_{i=0}^{N_g-1} A_{g_i} T_i \left(-\frac{\textstyle 2\log x - \log x_{min} } {\textstyle \log x_{min} } \right)\,, \label{eq:glu} \\
x S(x) &=& \left(1-x\right) \sum_{i=0}^{N_S-1} A_{S_i} T_i \left(-\frac{\textstyle 2\log x - \log x_{min} } {\textstyle \log x_{min} } \right)\,. \label{eq:sea} 
\end{eqnarray}
Here the sum over $i$ runs up to $N_{g,S}=15$ order Chebyshev polynomials of the first type $T_i$ for
the gluon, $g$, and sea-quark, $S$, density, respectively. 
The normalisation $A_g$ is given by the momentum sum rule.
The advantages of the parameterisation given by equations~\ref{eq:glu},\ref{eq:sea} is that momentum
sum rule can be evaluated analytically and  already for $N \ge 5$ the fit quality
is similar to a standard Regge-inspired parameterisation with a similar number of parameters.



