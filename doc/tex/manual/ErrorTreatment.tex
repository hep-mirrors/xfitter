%%%%%%%%%%%
%\subsection{Hessian Method}

%%%%%%%%%%%
%\subsection{Offset Method}

\newcommand{\rs}{s}
\newcommand{\ce}{b}
\DeclareRobustCommand\Vstat{\ensuremath{V^{\mathrm{(unc)}}}}
\DeclareRobustCommand\Vsys{\ensuremath{V^{\mathrm{(cor)}}}}
\DeclareRobustCommand\mb[1]{\ensuremath{\mbox{\mathversion{bold}{$#1$}}}}
\DeclareRobustCommand\mbs[1]{\ensuremath{\mbox{\mathversion{bold}{\scriptsize $#1$}}}}

\subsubsection {Correlated errors and \texorpdfstring{$\chi^2$}{chi2}}
\label{sec:cor-chi2}

% \cite{Chekanov:2002pv,Pascaud:1995qs}
% A typical formula for $\chi^2$ depending on the correlated systematic errors' fluctuations reads
Results of a measurement can be modelled as
(see eg.~\cite{Stump:2001gu,Botje:2001fx})
\begin{equation}
m_n = t_n(a) + r_n \sigma_n + \sum_{\mu=1}^K \rs_\mu \ce_{n\mu}
\;,\quad n=1,\dots,N
\end{equation}
where\\
$m_n$ is the value measured for the $n$-th data point,\\
$t_n(\mb a)$ is true (theoretical) value depending on parameters $\mb a = (a_1,\dots, a_M)$,\\
$\sigma_n$ is the uncorrelated error,\\
$\ce_{n\mu}$ are the errors from the $\mu$-th correlated error source,\\
$r_n$ and $\rs_\mu$ are random variables fluctuating around 0 with unit dispersion.

First, we assume that all $r_n$ are uncorrelated with $\rs_\mu$,
mutually independent and normally distributed,
%  with the statistical errors.
\begin{equation}
\rho(r) = \frac{e^{-r^2/2}}{\sqrt{2\pi}}
\,.
\end{equation}

In the following we will use scaled variables
\begin{subequations}
\begin{eqnarray}
x_i &\equiv& \frac{m_i-t_i}{\sigma_i}
\,,
\\
\beta_{i\mu} &\equiv& \frac{\ce_{i\mu}}{\sigma_i}
\,.
\end{eqnarray}
\end{subequations}

Keeping $\mb \rs$ fixed we get the probability density of measurements,
\begin{equation}
dp(\mb{m}| \mb s) =
 (2\pi)^{-N/2}\, e^{-\chi_1^2(\mbs\rs)/2}\, d^Nx
\,,
\end{equation}
where
\begin{equation}
\label{eq:chi1}
\chi_1^2(\mb\rs) = \sum_{n=1}^N
\left( x_n - \sum_\mu \beta_{n\mu} \rs_\mu \right)^2
\equiv (\mb{x - \beta s})^2
\,,
\end{equation}

Further, taking into account the probability distribution of the correlated error sources,
$p(\mb s)\, d^Ks$, we have
\begin{equation}
p(\mb{m},\mb s) = p(\mb s)\, p(\mb{m} | \mb s)
\,.
\end{equation}
Assuming again the uncorrelated normal distribution,
\begin{equation}
p(\mb s) = \prod_{\mu=1}^K \frac{e^{-s_\mu^2/2}}{\sqrt{2\pi}}
\,,
\end{equation}
we get
\begin{equation}
\label{eq:p_ms}
dp(\mb{m},\mb s) =
  (2\pi)^{-(N+K)/2}\, e^{-\chi_{\mathrm c}^2(\mbs\rs)/2}
  \,d^Nx\, d^Ks
\,,
\end{equation}
with
\begin{equation}
\label{eq:chi_c}
\chi_{\mathrm c}^2(\mb\rs) = 
% (\mb{x - \beta s})^{\mathrm{T}} (\mb{x - \beta s}) + \mb s^{\mathrm{T}} \mb s
(\mb{x - \beta s})^2 + \mb s^2
\,.
\end{equation}

This quadratic form in $\mb s$ allows for analytical integration of Eq~\ref{eq:p_ms}
resulting in

\begin{equation}
p(\mb{m}) \propto e^{-\chi^2/2}
\,,
\end{equation}
where
\begin{equation}
\label{eq:chi_int}
\chi^2 = \mb x^{\mathrm{T}} \mb{A\,x}
\end{equation}
% with constant $\mb A$ determined by $\mb\beta$ and $\mb\sigma$
with $\mb A$ depending on $\mb\beta$ only
(see eg. \cite{Stump:2001gu} Appendix B).

% If we assume that $\rs_\mu$ are random variables with normal distribution, we have
% \begin{equation}
% \tilde\chi^2(\rs) = \sum_{\mu=1}^K \rs_\mu^2
% \end{equation}
% and we can integrate them out analytically.
This is e.g. the CTEQ approach described in \cite{Stump:2001gu}.
It is worth noting that the solution Eq.~\ref{eq:chi_int} for $\chi^2$
can be obtained by minimizing $\chi_{\mathrm c}^2(\mb\rs)$ of Eq.~\ref{eq:chi_c} wrt. $\mb\rs$.

% -----------------------------------
\subsubsection{The Offset method}

In the Offset method presented here we assume that $\mb\rs$ is fixed, 
and we find the best theoretical model by minimizing
$\chi_1^2$ wrt. to $\mb a$. 
Hence the fitted parameters become functions
of $\mb\rs$. %, $\mb a = \mb a(\mb\rs)$. 
We do not impose any particular statistical properties on $\mb\rs$ 
and we take $\mb a(\mb\rs=0)$ as the ultimate fit result for the theory parameters. 
The dependence on $\mb\rs$ is, however, used to determine
the full error matrix of $\mb a$ (cf. \cite{Pascaud:1995qs}).

The full covariance matrix $V$ reads
\begin{equation}
\label{eq:Cv-full}
V = \Vstat + \Vsys
\end{equation}
% with $\Vsys$ coming from the $\rs$ fluctuations.
For each $\mb\rs$ we find the parameters $\mb a(\mb \rs)$ by minimising $\chi_1^2(\mb \rs)$, which results in
% $a = a(\rs)$ and 
$\Vstat(\mb \rs) = M^{-1}(\mb \rs)$ where
\begin{equation}
M_{jk}(\mb \rs)
= \left.
{\frac12} \frac{\partial^2\chi_1(\mb \rs)^2}{\partial a_j \partial a_k}\right\vert_{\mb a = \mb a(\mb \rs)}
\,.
\end{equation}
The dependence of $M$ on $\mb \rs$ is considered to be a higher order correction
and we take $\Vstat = M^{-1}(0)$.
% Nb. $\Vstat$ is the covariance matrix returned by Minuit for the fit with $\mb \rs =0$.

Within linear approximation to the error propagation
\begin{equation}
\label{eq:Vsysp}
\Vsys_{jk} = \sum_\mu \frac{da_j}{d\rs_\mu} \frac{da_k}{d\rs_\mu}
% \,.
\end{equation} 
and we calculate the derivatives as
\begin{equation}
\frac{da_j}{d\rs_\mu} \approx
\frac{a_j(\rs_\mu=\epsilon) - a_j(\rs_\mu=-\epsilon)}{2\epsilon}
% \,,
\end{equation}
with $\mb a(\rs_\mu=\epsilon)$ resulting from
fits to the data shifted by $\epsilon\ce_{n\mu}$.

In the code we use $\epsilon=1$, i.e. one standard deviation of
the correlated error source which, in the ideal statistical limit, corresponds to 
$\Delta\chi_1^2 = 1$.
On the other hand, within the leading approximation, the value of $\epsilon$ is irrelevant.
% in accordance to the standard Minuit normalization of $\Vstat$.

If another error definition, $\Delta\chi_1^2 = \lambda$, 
is adopted\footnote{E.g. Jon Pumplin uses $\lambda=5$, cf. \texttt{minuit/iterate.F}} then
the full covariance matrix, $V$, must be scaled by $\lambda$. 

\begin{description}
\item \bf{Implementation via \fitter\ } \rm

The Offset method is turned on by setting \verb:CHI2Style = 'Offset':
By default all fits are run in a single job, each fit driven by initial parameters and Minuit commands read from \verb'minuit.in.txt'.
Two optional parameters can be set in the \verb'CSOffset' NAMELIST, e.g.
\vspace*{-2.5ex}
\begin{verbatim}
&CSOffset
 CorSysIndex  =  0
 UsePrevFit = 1
&End
\end{verbatim}
\vspace*{-1ex}
Defaults are set in \verb'read_steer.f'
and the \verb'CSOffset NAMELIST' is read only when the Offset method is active.
Setting \verb'CorSysIndex' to any value $\in [-K, K]$ 
restricts the job to a single fit to data shifted (down or up) by a corresponding correlated error source.
\verb'CorSysIndex' = 0 corresponds to the central fit.
If \verb'CorSysIndex > NSYSMAX' then all the fits are performed.
The best way to perform all fits in a single run is 
to not specify \verb'CorSysIndex' at all 
(Default: \verb'CorSysIndex = NSYSMAX+1' )
\vspace{0.4cm}

The parameter {\tt UsePrevFit} determines how to use results of previous fits,
if such results are present in the \verb'output' folder.
\begin{enumerate}
\item [0 ---]
Do not use any previous fit results (Default)
\item [1 ---]
Use previously obtained parameters as starting values for the current fit.
% For a non-offset fit read initial parameters from \verb'minuit.save.txt';
% for an offset fit 
Read initial parameters from \verb'minuit.save_<CSI>.txt'
 --- e.g.  \verb'minuit.save_001m.txt' for \verb'CorSysIndex' $= -1$.
If the file does not exist and \verb'CorSysIndex' $\neq 0$ try to read
\verb'minuit.save_0.txt'.
\item [2 ---]
Do not perform the fit if a corresponding \verb'Results_<CSI>.txt' file exists,
otherwise switch to mode 1.
\end{enumerate}
\end{description}

%%%%%%%%%%%
\subsubsection{Monte Carlo Method}
\label{sec:ToyMC}

The PDF uncertainties can be estimated using a Monte Carlo technique \cite{Giele:1998gw, mcmethod2}.
The method consists in preparing replicas of data sets by allowing the central values of the cross sections to 
fluctuate within their systematic and statistical uncertainties taking into account all point-to-point correlations.
The preparation of the data is repeated for a large $N$ ($>100$ times) and for each of these replicas a NLO QCD fit is performed to 
extract the PDF set. The PDF central values and uncertainties are estimated using the mean values and RMS 
over the replicas. 
\subsubsection{Implementation in \fitter\ }
The steering flags to activate the MC method are located in the \tt steering.txt \rm via:

\begin{verbatim}
&MCErrors

  lRAND   = False  
  lRANDDATA = True
  ISeedMC = 123456 
  ! --- Choose what distribution for the random number generator 
  ! STATYPE (SYS_TYPE)  =   1  gauss
  ! STATYPE (SYS_TYPE)  =   2  uniform
  ! STATYPE (SYS_TYPE)  =   3  lognormal
  ! STATYPE (SYS_TYPE)  =   4  poisson (only for lRANDDATA = False !)
  STATYPE =  1
  SYSTYPE =  1
&End
\end{verbatim}
To activate the MC method for error estimation set \tt lRand = True \rm.
To use data (true, default) or theory (false) for the central values of the MC replica
the the flag \tt lRANDDATA \rm is used. The seed for random number generation is selected 
via \tt ISeedMC \rm . The smearing of the uncertainties can be treated differently for correlated 
or uncorrelated source and four distributions are supported for random number generators: Gauss, uniform, lognormal, and Poisson. If the flags are set to $0$ then no smearing is produced.




%%%%%%%%%%%
\subsubsection{Regularisation methods}

Regularisation methods are aiming to study the parametrisation assumptions of PDFs. 
When more flexible parametrisation styles is used the shape of the PDFs must be constrained and various methods are used.
The \fitter\ framework provides the means to study and compare various methods.

%The methods described in this section work for large data sets. 

\begin{description}

\item \bf{Data Driven Regularisation}\rm

This method was first applied by the NNPDF group.
It uses a redundant paremeters and introduces a stopping criterion based on data. This method splits data randomly
into ``fit'' and ``control''  samples. The ``fit'' sample is used to determine the PDF parameters. 
The $\chi^2$ of this sample is observed to decrease semi-monotonically. 
The ``control'' sample is used to protect against over-fitting and for this sample the $\chi^2$ will first decrease 
and then will start to increase due to fluctuation of the data.


\item \bf{External Regularisation based on a penalty term in $\chi^2$} \rm

Another method to constrain the PDF shape is to simply apply a penalty term
to the $\chi^2$ function. 
One method is the so called ``length penalty'' which selects PDF solutions with a smoother shape in $W\approx Q\sqrt{\frac{1-x}{x}}$:
\begin{equation}
L=\int_{W_{min}}^{W_{max}} \sqrt{1+\left(\frac{dxf(W)}{dW}\right)^2}dW
\end{equation}
This method can be applied when using Chebyshev polynomials to parametrise PDFs.
For more details, the reader is invited to consult  reference \cite{Chebyshev}.
This method is implemented in \fitter\ via steering flags under Namelist \tt \&Cheb \rm as follows:
\begin{verbatim}
&Cheb
  ! Set following > 0 to turn on:
   NCHEBGLU = 0   ! number of parameters for the gluon (max 15)
   NCHEBSEA = 0   ! number of parameters for the sea   (max 15)

  ! Cheb. polynomial type: multiply by (1-x) (1) or not (0)  
   ichebtypeGlu = 1 
   ichebtypeSea = 1 

  ! Starting point in x:
   chebxmin = 1.E-5

   ILENPDF  = 0   ! use pdf length constraint

  ! PDF length constraint strength for different PDFs:
   PDFLenWeight = 1., 1., 1., 1., 1.     

  ! Range in W where length constraint is applied:
   WMNLen =  20.
   WMXLen = 320.
&end
\end{verbatim}

Alternatively, when using the flexible parametrisation style a $\chi^2$
penalty term can be applied to account for th deviation from a simpler parametric form, for example
\begin{equation}
\chi^2_{reg}= T\sum_f\left(\left(\frac{D_f}{\Delta_D}\right)^2+ \left(\frac{E_f}{\Delta_E}\right)^2\right),
\end{equation}

with $\Delta D=\Delta E = 100$, such that for large $D$ and $E$ the ratio will approach $1$. 
$T$ is the regularisation parameters, such that for $T=0$ there is no penalty term. 
For large $T$ there is strong penalty.
This method of regularisation is accessed via Namelist: \tt \&ExtraMinimisationParameters \rm
with the name \tt 'Temperature' \rm which is $T$ in the above description.

\end{description}
