% Please make sure you insert your
% data according to the instructions in PoSauthmanual.pdf
\documentclass{article}

\usepackage{graphicx,subfigure}
\usepackage{enumerate}
\usepackage{amsmath} %% Math symbols
\usepackage{microtype} %% Fancier typesetting
\usepackage[top=3cm, bottom=3cm, left=3cm, right=3cm]{geometry}
\usepackage{titling} % Customizing the title section
\usepackage{setspace}
\setstretch{1.25}
\usepackage{float}
\usepackage{amsmath,amssymb,amsfonts}
\usepackage{mathtools}
\usepackage{caption}
\usepackage{footnote}
\usepackage[usenames,dvipsnames]{xcolor}
\usepackage{units}
\usepackage[squaren]{SIunits}
\usepackage{bm}
\usepackage{bbm}
\usepackage{hyperref}
\usepackage{authblk} % formating of authors and affiliation
% Package for dashed lines in tables
\usepackage{arydshln}
% new command for large Chi
\newcommand{\Chi}{\large{\chi}}

\title{Short User Guide for nPDFs in xFitter}
      
\author[1]{Marina Walt \thanks{marina.walt@uni-tuebingen.de}}
\affil[1]{Institute for Theoretical Physics, University of T\"ubingen, Auf der Morgenstelle 14, 72076 T\"ubingen, Germany}

\date{\today} % Leave empty to omit a date 


\begin{document}

\maketitle
\tableofcontents


\section{Introduction}

A new QCD analysis for nuclear parton distribution functions (nPDFs) at next-to-leading order (NLO) and next-to-next-to-leading order (NNLO) was published in \cite{Walt:2019slu}. The framework of the analysis, including the form of the parameterization as well as the included DIS datasets, are discussed there. Also the results of that QCD analysis are compared to the existing nPDF sets and to the fitted data in reference \cite{Walt:2019slu}. The presented framework is based on \textsc{xFitter} \cite{Zenaiev:2016jnq,Bertone:2017tig} which has been modified to be applicable also for a nuclear PDF analysis. The purpose of this documentation is to provide a short user guide for nPDFs in \textsc{xFitter}. For the general xFitter manual please refer to~\cite{xfitter-manual}. A summary of the required modifications can be found in~\cite{Walt:2019slu}.

\clearpage
\section{nPDF Parameterization} 

Usually, the information on the mass number~$A$ and the proton number~$Z$ is individual per dataset since different nuclei can be involved in the measured scattering reaction. Therefore, it is recommended to provide $A$ and $Z$ for the particular nucleus inside the dataset file. This option is described in section~\ref{sec-nucldata}.\\ 
\\
In order to activate the nPDF analysis one needs to set the following parameters in the \textbf{steering.txt} file:
\begin{enumerate}
\item PDFType = 'nucleus'
\item Anucleus = 1.0 [Mass number $A$, =1 for proton]
\item Znucleus = 1.0 [Proton number $Z$, =1 for proton]
\end{enumerate}
\vskip 0.2in
\noindent Please note: there are several nuclei, and not a single one, involved in a global analysis. Therefore, it is strongly recommended to provide the information on the individual combinations of $A$ and $Z$ inside the data files. Here, in the steering.txt file, the parameters Anucleus, Znucleus are used only in case that this information is \textit{not} provided in the data files.\\
\\
As per the current implementation, the nPDFs are available only for the following combination of parameterization:
\begin{enumerate}
\item[4.] PDFStyle = 'CTEQ'
\item[5.] nTUJU = True [Flag to activate constraints, like e.g. sum rules, in the TUJU19 style \cite{Walt:2019slu}]
\end{enumerate}

\noindent The extension 'n' in the name of the flag 'nTUJU' symbolizes that 'TUJU' framework has been developed for \textit{nuclear} PDFs. However, the flag 'nTUJU' can be used for both PDFTypes proton and nucleus. If the user does not set the flag 'nTUJU = True' the program will run, but the results might be potentially inconsistent. \\
\\
By selecting PDFStyle = 'CTEQ' the following parameterization is applied:
\begin{equation}
xf^{p/A}_i\left(x,Q_0^2 \right) = c_0\,x^{c_1} (1-x)^{c_2} \left(1+c_3\,x + c_4\,x^2 \right)
\label{eq:pdf-parameterization}
\end{equation}
with $i=g,\,u_v\,d_v\,\bar{u},\,\bar{d},\,s$ as per TUJU19 framework. A similar ansatz has been used to derive ~\cite{Abramowicz:2015mha}. Note: For 'nTUJU=True' the number of parameters per parton flavour is limited to 5 ($c_i$ with $i=0,...,4$).\\
\\
The same form of the parameterization (\ref{eq:pdf-parameterization}) is valid for both, proton and nuclear PDFs. The difference appears in regards to the parameters $c_i$ ($i=0,...,4$). For nuclear PDFs the coefficients in equation (\ref{eq:pdf-parameterization}) are further parameterized to be dependent on the nuclear mass number $A$ as
\begin{equation}
c_k\,\rightarrow c_k(A) = c_{k,0}+c_{k,1}\left( 1 - A^{-c_{k,2}} \right)
\label{eq:coeff-A}
\end{equation}
with $k={0,\dots,4}$. This form of $A$-dependent coefficients was used in the nCTEQ15 analysis \cite{Kovarik:2015cma}. This $A$-dependent parameterization has the advantage that in case of a free proton $(A=1)$ the term $\left( 1 - A^{-c_{k,2}} \right)$ in equation (\ref{eq:coeff-A}) becomes zero and the functional form of a free proton is automatically retained. Here, again $k={0,\dots,4}$ is valid only for 'nTUJU=True' setting.\\
\\
As per the current implementation, nuclear, i.e. $A$-dependent coefficients are implemented only for the PDFStyle='CTEQ' as described above. For each coefficient in equation~(\ref{eq:pdf-parameterization}) one can have two additional $A$-dependent parameters as per equation~(\ref{eq:coeff-A}).\\
\\
In order to provide the initial parameters the \textbf{minuit.in.txt} file is used. The input format of coefficients $c_k \equiv c_{k,0}$ remains unchanged, i.e.:\\ 

\begin{tabular}[h!]{ll}
Parameters & Parton \\ \hline
$1-9$ & $g$ \\
$11-19$ & $u_v$\\
$21-29$ & $d_v$\\
$31-39$ & $\bar{u}$\\
$41-49$ & $\bar{d}$\\
$81-89$ &$s$\\
$90-100$ & others
\end{tabular}
\vskip 0.2in

\noindent These parameters are what one could call coefficients of the PDF for the \textit{free} proton.
What is new for nPDFs are the $A$-dependent coefficients $c_{k,1}$ and $c_{k,2}$. For those the following parameter numbers have been reserved in the \textbf{minuit.in.txt} file:\\

\begin{tabular}[h!]{lll}
Parameter $\#$ & Parameter name & Parton \\ \hline
111, 112 & $c_{0,1}$, $c_{0,2}$ & $g$ \\
113, 114 & $c_{1,1}$, $c_{1,2}$ &  \\
115, 116 & $c_{2,1}$, $c_{2,2}$ &  \\
117, 118 & $c_{3,1}$, $c_{3,2}$ &  \\
119, 120 & $c_{4,1}$, $c_{4,2}$ &  \\ \hdashline[.4pt/2pt]
131, 132 & $c_{0,1}$, $c_{0,2}$ & $u_v$ \\
133, 134 & $c_{1,1}$, $c_{1,2}$ &  \\
135 - 140 & \dots &  \\\hdashline[.4pt/2pt]
151, 152 & $c_{0,1}$, $c_{0,2}$ & $d_v$ \\
153, 154 & $c_{1,1}$, $c_{1,2}$ &  \\
155 - 160 & \dots &  \\ \hdashline[.4pt/2pt]
171, 172 & $c_{0,1}$, $c_{0,2}$ & $\bar{u}=\bar{d}$ \\
173, 174 & $c_{1,1}$, $c_{1,2}$ &  \\
175 - 180 & \dots & 
\end{tabular}
\vskip 0.2in
\noindent The $c_{k,1}$ and $c_{k,2}$ parameters for $\bar{d}$ are determined by the constrain $\bar{u}=\bar{d}$. Due to the limited maximum number of parameter allowed in \textbf{minuit.in.txt} file, the $c_{k,1}$ and $c_{k,2}$ coefficients for $s$ have been assigned to the following parameter space:
\vskip 0.2in
\begin{tabular}[h!]{lll}
Parameter $\#$ & Parameter name & Parton \\ \hline
129, 130 & $c_{0,1}$, $c_{0,2}$ & $s=\bar{s}$ \\
149, 150 & $c_{1,1}$, $c_{1,2}$ &  \\
169, 170 & $c_{2,1}$, $c_{2,2}$ &  \\
184, 185 & $c_{3,1}$, $c_{3,2}$ &  \\
186, 187 & $c_{4,1}$, $c_{4,2}$ &  \\ 
\end{tabular}
\vskip 0.2in

\noindent Please note, that the implemented routine has been validated only for 'nTUJU=True' and '{PDFSTyle~=CTEQ}' with $c_{k,\mathrm{max}}=4$. In the general case ('nTUJU=False') the according assignment might change or overlap (especially for $s$ quarks).\\
\\
Furthermore, please note, that with 'nTUJU=True' the number sum rule and the momentum sum rule are used to constrain the normalizations of $d_v$, $u_v$ and $\bar{u}$. Furthermore, the constraint $\bar{u}=\bar{d}=s=\bar{s}$ is applied.\\
\\
The nuclear parton distribution function $f_i^{\,N/A}$ for a \textit{bound} nucleon inside a nucleus with mass number $A$ is constructed from the \textit{bound} proton's PDF $f_i^{\,p/A}$ (not from a free proton's PDF $f^p$). In particular for the distribution of partons in a bound nucleon we write
\begin{equation}
f_i^{\,N/A} \left( x,Q^{\,2} \right) = \frac{Z\cdot f_i^{\,p/A}+ (A-Z)\cdot f_i^{\,n/A}}{A}\,,
\label{eq:nucleon}
\end{equation} 
where $Z$ is the number of protons in the nucleus. The PDF of the \textit{bound} neutron~$f_i^{\,n/A}$ is determined from the fitted proton's PDF using the isospin symmetry.\\
\\
This nucleon decomposition has been implemented in xFitter for the calculation of DIS cross sections by using the following heavy-quark schemes only: '\textbf{ZMVFNS}' scheme and '\textbf{FONLL}' scheme. For the other schemes available in xFitter, the nucleon decomposition has not been implemented/validated.

\section{Nuclear Data}
\label{sec-nucldata}

For nuclear data, the experimental measurements are often published for a ratio of a cross section measured on one nucleus with mass number $A_1$ to the cross section of the other nuclear target $A_2$, i.e. $\sigma(A_1)/\sigma(A_2)$ for cross sections or $F_2(A_1)/F_2(A_2)$ for structure functions. In such a case, inside the data file one would not provide a single observable, but a ratio of two observables. As part of the fitting routine, the quantities inside the data files are compared to the calculated theoretical values. In order to have a consistent comparison one needs to identify inside the data file if the bin 'Sigma' is a ratio or not. For a ratio it needs to be set:\\
\hspace*{0.5cm}CInfo = 'ratio'\\
\hspace*{0.5cm}DataInfo = 1.0 [and 0.0 for an absolute cross section].\\
\\
The information on $Z_1,\,A_1\,$ and $Z_2,\,A_2$ (if applicable) is also provided inside the data file:\\
\hspace*{0.5cm}CInfo = 'A1', 'Z1', 'A2', 'Z2'\\
\hspace*{0.5cm}DataInfo = 56., 26.0, 12.0, 6.0 [an example for $\sigma(Fe)/\sigma(C)$].\\
\\
Besides that, some experiments apply isoscalar corrections to the measured data and publish only the modified information. Thus, the analysis procedure has been adapted so that the theoretically calculated quantities are consistent with the iso-corrected experimental data. For this purpose, different flags were introduced in \textsc{xFitter} for the different forms of isoscalar corrections, which are specific to the corresponding experiments (CInfo$=$'NMC', 'EMC', 'SLAC'). For the explicit form of the isoscalar correction please refer to the TUJU19 publication \cite{Walt:2019slu}. The information if or if not an isoscalar flag needs to be applied is provided by the particular experimental publication. In order to set the corresponding flag CInfo and DataInfo are used:\\
\hspace*{0.5cm}CInfo = 'NMC' [or 'EMC' or 'SLAC' respectively]\\
\hspace*{0.5cm}DataInfo = 1.0 [if True, and 0.0 if False].\\
\\
Eventually, another modification on \textsc{xFitter} was necessary for the treatment of charged current DIS processes measured in neutrino-nucleus scattering reactions. As part of this framework, the differential cross sections $\mathrm{d}\sigma^2/\mathrm{d}y\mathrm{d}Q$ were used for the analysis. In order to identify the corresponding reaction one needs to specify inside the data file:\\
\hspace*{0.5cm}Reaction = 'antineutrino+p CC' [or Reaction = 'neutrino+p CC'].\\
\\
Please note that these additional processes have been implemented only for the two heavy-quark schemes 'ZMVFNS' and 'FONLL'. When using another scheme available in xFitter an error message might probably occur.


\section{Output and Error Analysis}

When running the analysis routine for nPDFs in xFitter the file minuit.out.txt will automatically contain all final parameters (proton and nuclear parameters). Files containing PDFs, originally named 'pdfs$\_$q2val$\_$0i.txt', are named 'A-XXX$\_$pdfs$\_$q2val$\_$0i.txt' where XXX is the nuclear mass number (e.g. 2 for D, and 208 for Pb). During the Hessian error analysis, there are also PDF files generated, those do not carry an $A$-dependent extension, but are valid for the nucleus used last in the series of data files. This is due to the reason, that one would need an additinal loop over all nuclei in the error analysis routine which is not implemented yet. A workaround exists by running the error analysis part ('DoBands=True') changing the series of data files, so that every time another nucleus is listed last. The same is valid for the generated output in the LHAPDF format. The central PDF members and the *.info files are generated for all nuclei and are stored in folders named 'A-XXX$\_$xfitter$\_$pdf', whereas the error set members are created only for the last nucleus on the list. This is valid for the Hessian error analysis only. If the Monte Carlo (MC) error analysis method is used, all nuclei are covered simultaneously. The text files containing PDFs and the grids in LHAPDF format are generated for the partons in a nucleon (\textit{not} proton). \\
\\
For nuclear PDFs the Hessian error analysis is usually performed with $\Delta \Chi^2 >1$. The error bands routine has been modified accordingly so that scaling based on the quadratic approximation is applied (please refer to \cite{Walt:2019slu} for more details ). In order to set the $\Delta \Chi^2$ parameter the command \\
\hspace*{0.5cm} set errdef 10.0 [for example] \\
can be used in the minuit.in.txt file.


%\newpage
\section{Change Log}

The modifications described here have been applied on xFitter versoin 2.0.1.\\

Modified header and include files:
\begin{itemize}
\item include/c$\_$interface.inc
\item include/dimensions.h
\item include/indata.inc
\item include/pdfparam.inc
\item include/qcdnumhelper.inc
\item include/steering.inc
\item include/theo.inc
\item include/theorexpr.inc
\item include/xfitter$\_$cpp.h
\end{itemize}

\vskip 0.2in
\noindent Modified source code files (bold items have been modified largely):
\begin{itemize}
\item interfaces/src/hf$\_$pdf$\_$calls.f
\item src/c$\_$interface.f
\item src/chi2scan.cc
\item src/dataset$\_$tools.f
\item \textbf{src/dis$\_$sigma.f}
\item src/error$\_$bands$\_$pumplin.f
\item src/evolution.f
\item \textbf{src/fcn.f}
\item src/lhapdf6$\_$output.c
\item src/lhapdferrors.cc
\item \textbf{src/pdf$\_$param.f}
\item src/read$\_$data.f
\item src/read$\_$steer.f
\item \textbf{src/sumrules.f}
\item src/theory$\_$dispatcher.f
\end{itemize}

\vskip 0.2in
\noindent New source code files:
\begin{itemize}
\item src/nucl$\_$pdf.f
\item src/nucl$\_$pdfcc.cc (optional, requires gsl libraries)
\end{itemize}

\noindent Additionally, files Makefile.in and Makefile.am in the src/ folder have been modified in order to add new source code files.


\begin{thebibliography}{99}

\bibitem{Walt:2019slu} M. Walt, I. Helenius, W. Vogelsang, arXiv:1908.03355.

\bibitem{Zenaiev:2016jnq} xFitter team (O. Zenaiev for the collaboration). xFitter project, PoS DIS2016 (2016) 033.

\bibitem{Bertone:2017tig} xFitter Developers' Team (V. Bertone et al.), xFitter 2.0.0: An Open Source QCD Fit Framework, PoS DIS2017 (2018) 203, arXiv:1709.01151.

\bibitem{xfitter-manual} \url{https://www.xfitter.org/xFitter/xFitter/DownloadPage?action=AttachFile&do=view&target=manual.pdf}

\bibitem{Abramowicz:2015mha} H. Abramowicz et al., Eur.Phys.J. C75 (2015) no.12, 580.

\bibitem{Kovarik:2015cma} K. Kovarik et al., Phys.Rev. D93 (2016) no.8, 085037.


\end{thebibliography}

\end{document}

