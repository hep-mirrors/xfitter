\documentclass[11pt,a4paper]{article}
\usepackage{graphicx,epsfig}
\usepackage{hhline}

\usepackage{amsmath,amssymb}
\usepackage{times}
\usepackage[varg]{txfonts}
\DeclareMathAlphabet{\mathbold}{OML}{txr}{b}{it}

\usepackage{array,multirow,dcolumn}
%\usepackage[mathlines,displaymath]{lineno}
\usepackage{rotating}

 
% we use natbib instead of cite to work with hyperref
%\usepackage{cite}
%\usepackage[numbers,square,comma,sort&compress]{natbib}
%\usepackage{hypernat}
\usepackage{textcomp}

\bibliographystyle{alpha}
\renewcommand{\topfraction}{1.0}
\renewcommand{\bottomfraction}{1.0}
\renewcommand{\textfraction}{0.0}

% \renewcommand{\arraystretch}{1.2}
\newlength{\dinwidth}
\newlength{\dinmargin}
\setlength{\dinwidth}{21.0cm}
\textheight24cm \textwidth16.0cm
\setlength{\dinmargin}{\dinwidth}
\setlength{\unitlength}{1mm}
\addtolength{\dinmargin}{-\textwidth}
\setlength{\dinmargin}{0.5\dinmargin}
\oddsidemargin -1.0in
\addtolength{\oddsidemargin}{\dinmargin}
\setlength{\evensidemargin}{\oddsidemargin}
\setlength{\marginparwidth}{0.9\dinmargin}
\marginparsep 8pt \marginparpush 5pt
\topmargin -42pt
\headheight 12pt
\headsep 30pt \footskip 32pt
\parskip 3mm plus 2mm minus 2mm


\newcommand\fitter{ \mbox{\tt HERAFitter} }
\title{\fitter\ - PDF Fitting package}
\author{HERAFitter developers}
\begin{document}
\maketitle
\begin{abstract}
\end{abstract}
\tableofcontents
\newpage
%%%%%%%%%%%%%%%%%%%%%%%%%%%%%%
\section{Introduction}
\label{section:introduction}
%%%%%%%%%%%%%%%%%%%%%%%%%%%%%%
This manual provides a short description of the \fitter\ program 
which can be used to determine unpolarised parton density functions 
(PDFs) using deep inelastic scattering (DIS) data and other processes such as 
Drell-Yan, jet or ttbar processes.
The parton density functions are needed to calculate cross sections
for the $ep$ and $pp$ colliders and thus required for interpetation
of the data collected at the LHC.
% The \fitter\ program were used to determine the HERA1.0 PDF set~\cite{h1zeus:2009wt}.

The manual begins with program installation instructions for different scenarios (section~\ref{sec:install}), followed by a brief discussion of the theoretical calculation
used in the program (section~\ref{sec:theory}) continued by a description of the
PDF parameterisation (section~\ref{sec:pdfparam}) and various $\chi^2$ functions used in the
minimisation (section~\ref{sec:chi2}). A description of the program steering cards and
the output options is given in section~\ref{sec:man}.

  
%%%%%%%%%%%%%%%%%%%%%%%%%%%%%%
\section{Program Installation Instructions} 

\label{sec:install}
%%%%%%%%%%%

The Installation Instructions are dependent on which modules are activated via the configuration option. 
\subsection{Pre-requirements}

The following packages are needed in order to build \fitter\ package:
\begin{itemize}
\item QCDNUM~\cite{qcdnum} version at least {\tt qcdnum-17-00/04}, can be found at \\
  {\tt http://mbotje.web.cern.ch/mbotje/qcdnum/Site/QCDNUM17.html}
\item {\tt CERNLIB} libraries. Note that for {\tt CERNLIB} one can use {\tt /afs/} installation from CERN:
  {\tt /afs/cern.ch/sw/lcg/external/cernlib/}
%\item Link to recent Root libraries (e.g. version 5.26)
%\item Optional: {\tt APPLGRID}
\end{itemize}
The \fitter\ program has been tested on various platforms: 
   SL4, SL5 (32 and 64 bit),  Ubuntu 10.10.
%%%%%%%%%%%
\subsection{Default Installation}
\begin{itemize}
\item
 Specify {\tt CERN\_ROOT} 
     and {\tt QCDNUM\_ROOT} variables such that 
     {\tt \$CERN\_ROOT/lib}  and {\tt \$QCDNUM\_ROOT/lib}
 point to the corresponding libraries
\item Run:
\begin{verbatim}
%    autoreconf --install
    ./configure
    make 
    make install
\end{verbatim}
After these commands are finished, the executable {\tt bin/FitPDF} 
file should be installed
\item  Run a check:
\begin{verbatim}
    bin/FitPDF 
\end{verbatim}
\end{itemize}
%%%%%%%%%%%
\subsection{Installation with {\tt APPLGRID}}
\begin{itemize}
\item
 Specify {\tt CERN\_ROOT} and {QCDNUM\_ROOT} variables such that 
     {\tt \$CERN\_ROOT/lib}  and {\tt \$QCDNUM\_ROOT/lib}
 point to the corresponding libraries
\item Make sure that {\tt \$PATH} and {\tt \$LD\_LIBRARY\_PATH} 
variables point to the {\tt APPLGRID} environment.
\item Run:
\begin{verbatim}
    autoreconf --install
    ./configure --enable-applgrid
    make 
    make install
\end{verbatim}
After these commands are finished, the executable {\tt bin/FitPDF} 
file should be installed
\item  Run a check:
\begin{verbatim}
    bin/FitPDF 
\end{verbatim}
\end{itemize}
%%%%%%%%%%%
\subsection{Installation with {\tt LHAPDF}}\label{sec:install_lhapdf}

Installation with LHAPDF requires the {\tt LHAPDF} package, available online at:\\
{\tt http://lhapdf.hepforge.org/install}.
Then
\begin{verbatim}
tar -xvzf lhapdf-v.r.p.tar.gz
cd lhapdf-v.r.p
./configure --prefix=/path/to/directory (and/or --enable-low-memory)
make
make install
cd /path/to/directory/share/lhapdf
mkdir PDFsets
\end{verbatim}

Once installed, then create the path that will be linked to the {\tt HERAFitter} package.
 Specify {\tt LD\_LIBRARY\_PATH} 
     and {\tt LHAPATH} variables such that they
 point to the corresponding libraries, and PDF sets location (where lhapdf tables are stored)
\begin{verbatim}
export LD_LIBRARY_PATH=/path/to/directory/lhapdf-v.r.p/lib:\$LD_LIBRARY_PATH
export LHAPATH=/path/to/directory/share/lhapdf/PDFsets
\end{verbatim}



%%%%%%%%%%%
\subsection{Installation with {\tt PDF reweighting}}\label{sec:install_nnpdfrweight}

Note: For installation allowing for PDF reweighting, the latest version of {\tt LHAPDF}, lhapdf-5.8.7b2, should be installed.

\begin{itemize}
\item Make sure that {\tt \$LD\_LIBRARY\_PATH} includes the LHAPDF libraries.
\item Run:
\begin{verbatim}
    autoreconf --install
    ./configure --enable-lhapdf  --enable-nnpdfWeight
    make 
    make install
\end{verbatim}
After these commands are finished, the executable {\tt bin/FitPDF} 
file should be installed
\item Set {\tt FLAGRW = True} in the steering file and change also the other parameters of the {\tt \&reweighting} namelist if needed.
\item  Run a check:
\begin{verbatim}
    bin/FitPDF 
\end{verbatim}
\end{itemize}


%%%%%%%%%%%
\subsection{Installation with {\tt HATHOR}}

 \begin{itemize}
  \item Download Hathor from 
\begin{verbatim}
http://www-zeuthen.desy.de/~moch/hathor/
\end{verbatim}
     and install it according to the instructions given there
     (requires \begin{verbatim}LHAPDF \end{verbatim}\ library)

  \item Define a variable HATHOR\_ROOT  such that HATHOR\_ROOT  points to the
     directory of your Hathor installation

  \item Install the H1Fitter as described above but configuring it
     with the option "--enable-hathor" before building it
 \end{itemize}


%%%%%%%%%%%
%\subsection{Installation with {\tt CASCADE}}
\subsection{Installation for TMD (uPDF) in high-energy factorisation (using  {\tt CASCADE})}

\begin{itemize}

\item  set environment variables (with {SYSNAME=i586\_rhel50} or similar)
    
\begin{verbatim}
    export CERN\_ROOT=/cern/pro  
    export QCDNUM\_ROOT=/h1wgs/h1desy11/x04/usr/glazov/openfitter/qcdnum-17-00-03
    export CASCADE\_ROOT=/afs/desy.de/group/alliance/mcg/public/MCGenerators/cascade/2.2.04/\$SYSNAME 
    export PYTHIA\_ROOT=/afs/desy.de/group/alliance/mcg/public/MCGenerators/pythia6/425/\$SYSNAME}
    \end{verbatim}	

\item use steering and minuit input files from "input\_steering": 

   \begin{verbatim} 
   cp input-steering/steering.txt.kt-factorisation steering.txt 
   cp input-steering/minuit.in.txt.kt-factorisation minuit.in.txt 
   cp input-steering/steer-ep-CASCADE steer-ep 
   \end{verbatim}

\item  edit steering.txt: 
 \begin{verbatim}
   \&CCFMFiles: give name for output grid file for uPDF.\&H1Fitter 
   \&H1Fitter \\ 
   \!  ITheory = 101   ! =101 fit with kernel ccfm-grid.dat file 
   \!  ITheory = 102   ! =102 fit evolved uPDF, fit just normalisation 
    ITheory = 103   ! =103 fit using precalculated grid of $sigma\_hat$
       all other parameters are standard
  \end{verbatim}

\item run the program: bin/FitPDF 
   
\item plotting F2 fit results:
   DrawResults \! will draw F2 results to plot uPDF one can use the "updfplotter" web interface $\to$ copy updf grid (with name from "CCFMFiles" under 2 to somewhere .... this description needs improvements
\end{itemize}


 
%%%%%%%%%%%%%%%%%%%%%%%%%%%%%%
\section{Theoretical Input}
\label{sec:theory}
The \fitter\ program uses currently as standard the DGLAP~\cite{Gribov:1972ri,Gribov:1972rt,Lipatov:1974qm,Dokshitzer:1977sg,Altarelli:1977zs}
 evolution equations as implemented in the QCDNUM~\cite{qcdnum} program. The fit 
procedure begins with parameterising the input PDFs at the starting 
scale $Q^2_0$ which should be chosen to be below the charm mass threshold
$m_C^2$.
The PDFs are then evolved using the DGLAP evolution equations  
at NLO~\cite{Curci:1980uw,Furmanski:1980cm} in the $\overline{\text{MS}}$ scheme.
%The renormalisation and factorisation scales are set to $Q^2$. 
The LO and NNLO evolutions are also provided by the QCDNUM and can be 
selecteed via the steering parameters in \fitter\ . 

The cross-section predictions are obtain by convoluting the PDFs with the 
hard scattering coefficient functions. For the DIS processes, those are calculated 
using the Fixed-Flavour number (FFN)~\cite{Laenen:1992,Laenen:1993,Riem:1995} or 
the general mass Variable-Flavour number (VFN) schemes. 
The program implements the zero mass variable flavour number (ZMVFN) scheme from QCDNUM.

%The various treatments for the heavy quark thresholds are implemented as provided by the MSTW group
The VFN schemes with various treatments for the heavy quark thresholds are implemented in \fitter\ 
as provided by the MSTW group
- the RT scheme with its variants at NLO and NNLO ~\cite{Thorne:1997ga,Thorne:2006qt}, 
as provided by the CTEQ group - the ACOT scheme with its variants at LO and NLO, 
as provided by the ABM group - the BMSN scheme at NLO and NNLO~\footnote{The BMSN scheme as provided by ABM group 
currently is not yet fully implemented in \fitter\ .}. The fixed-flavour number scheme
is available via the QCDNUM implementation and via the {\tt OPENQCDRAD~\cite{openqcdrad:page} } interface.
Each of these schemes is briefly discussed in further details.

The jet cross sections
are calculated using APPLGRID or FastNLO. The program has two implementations
for $pp$  Drell Yan processes: The first implementation uses
calculations at LO which can be extended to NLO using k-factors,
the second uses the APPLGRID interface.
For thorough details of the theoretical modules we direct the user to read the provided references of these packages.


%%%%%%%%%%%
\subsection{DIS and Schemes}
%There are different approaches to the treatment of the heavy quark production. 
%These include the fixed-flavour (FFN) and variable flavour number (VFN) schemes.
%%%%
\subsubsection{ZMVFNS}
The evolution program QCDNUM~\cite{qcdnum} used in \fitter\ provides 
the calculation of the deep inelastic structure functions in the zero-mass, 
generalised mass and the fixed flavour number schemes. 
In the zero-mass variable flavour number scheme (ZM-VFNS) heavy quark densities are
included into proton above quark masses but they are treated as massless in both,
the initial and final states.
This scheme is accurate in the region where $Q^2$ is so much greater than $m_h^2$
but becomes unreliable for $Q^2 \sim m_h^2$. \\
The un-polarised DIS structure functions in ZM-VFN scheme are computed as a 
convolution of the parton densities with zero-mass coefficient functions and 
in \fitter\ are activated via namelist {\tt HF$\_$SCHEME} in the {\tt steering.txt}.
%%%%
\subsubsection{TR}

The Thorne-Roberts (TR) scheme is a general-mass variable flavour number scheme (GM-VFNS) used as default for the MTSW PDF sets. GM-VFNS smoothly connect the two regions: scale below the heavy quark threshold and the scale above the heavy quark threshold. However, the definition is not unique.
A GM-VFNS can be defined by demanding equivalence of the nf = n (FFNS) and nf = n+1 flavour (ZM-VFNS) descriptions above the transition point for the new parton distributions
(they are by definition identical below this point), at all orders. However, the equivalency of swapping the $O(m_{H}^2/Q^2)$ 
terms without violating the definition of a GM-VFNS is what mainly distinguish the ACOT from TR schemes. 
One major issue in a complete GM-VFNS, is that of the ordering of the
perturbative expansion. This ambiguity comes about because the ordering in alphas is different for the number of active flavours.


 
%%%%
\subsubsection{ACOT}

The Aivazis-Collins-Olness-Tung scheme  belongs to the group of VFN factorization schemes that uses the renormalization method of Collins-Wilczek-Zee (CWZ).This scheme involves a mixture of the $\overline{\text{MS}}$ scheme for light partons (and for heavy partons if the factorisation scale is larger than the heavy mass)and the zero-momentum subtraction renormalization scheme for graphs with heavy quark lines (if factorisation scale is smaller than the mass of the heavy quark threshold).

There are different variants of this scheme which are all incorporated in the \fitter\  framework and can be selected 
via namelist {\tt HF$\_$SCHEME} in the {\tt steering.txt} .

%%%%
\subsubsection{FFNS}

The fixed-flavour number scheme for DIS structure functions in \fitter\ can be selected via {\tt HF$\_$SCHEME}
using QCDNUM or ABM~\cite{openqcdrad:page} implementation.
In the FFN scheme only gluon and the light quarks are considered
as partons within the proton, massive quarks are produced perturbatively in the final state.
\\
In addition, the recent variant of the fixed-flavour number scheme in which the running mass definition 
is used in the $\overline{\text{MS}}$ scheme~\cite{Alekhin:runm} can be selected
in \fitter\ . This variant is realised via the interface to the open-source code 
OPENQCDRAD~\cite{openqcdrad:page}.
This scheme has the advantage of reducing the sensitivity of the DIS cross sections to
higher order corrections, and improving the theoretical precision of the mass definition. 
\\           
In the QCDNUM, the calculation of the heavy quark contributions to DIS structure functions
are avalable at NLO and only electromagnetic exchange contributions are taken into account.
In the ABM implementation, the QCD corrections to the massive Wilson coefficients 
up to the NNLO for the neutral-current (NC) heavy-quark production~\cite{} and up to NLO
for the charged-current (CC) case are available.
%
%The fixed-flavour number scheme for  in \fitter\ can be also accessed via interface to
%the open-source code OPENQCDRAD~\cite{openqcdrad:page} in \fitter\ framework 
%provides an access to 
%fixed flavour number scheme (FFNS)~\cite{Laenen:1992,Laenen:1993,Riem:1995}
%
%By default, in FFNS the number of light quark 
%flavours $n_{f}$ (here $n_{f}=3$) are considered in the PDF evolution and heavy (massive) 
%quarks appear only in the final state. 
%The QCD corrections to the massive Wilson coefficients which are known up to the NNLO
%for the neutral-current (NC) heavy-quark production~\cite{} are implemented in OPENQCDRAD.
%In the case of charged-current (CC), the massive NLO QCD corrections~\cite{} are available.
%In addition, the treatment of the heavy-quark contributions in DIS are provided 
%in both, the pole-mass and the running-mass definition in $\overline{\text{MS}}$ 
%scheme~\cite{Alekhin:runm}. \\
%In case the FFN scheme is chosen as the fitting option (see corresponding instructions in the section 1),
%the heavy quark contributions to DIS structure functions $F_2$ and $F_L$ (and $F_3$ in the charged 
%current case) are calculated in the FFNS and together with the light-flavor contributions are 
%provided for the theory prediction calculation (theory$\_$dispatcher.f).
%The interpolation to PDFs and $\alpha_s$ evolution from QCDNUM are set up in the interface to OPENQCDRAD. \\
%The variation of the renormalisation and factorisation scales for heavy quarks is 
%possible (see available options in steering.txt file).
%
%       
%%%%%%%%%%%
\subsubsection{\texorpdfstring{$ep$}{ep} Electroweak corrections}
%%%%%%%%%%%
 
To properly compare the experimental data with theoretical predictions, 
QED corrections are necessary. In the HERAFitter, the electroweak corrections 
for the DIS process are based on the EPRC package provided by Hubert Spiesberger \cite{HS}.
The measured cross sections as presented by HERA are not corrected for the weak corrections
in order to retain sensitivity to higher order EW effects.

The calculations of higher-order electroweak corrections to DIS scattering at HERA are performed
in the on-shell scheme where the gauge bosons masses $M_W$ abd $M_Z$ are treated symmetrically
as basic parameters together with the top and Higgs masses, besides the fine structure constant $\alpha$ and other fermion masses.

The code provides the running of $\alpha$ using the most recent parametrisation
of the hadronic contribution to $\Delta_\alpha$ \cite{Jegerlehner}, as well as an older 
one from Burkhard.

For the Drell Yan process there are independent treatments (such as SANC, FEWZ).

 


\subsection{Drell Yan process}
%%%%

The calculatiosn of the Drell Yan processes are known for many observables up to  the NNLO order. For example, there are packages such as FEWZ, DYNNLO for NNLO, or MCFM for NLO calculations. However, due to the complicated nature of these calculation involving an increased number of diagrams with additional higher order, these calculations are too slow to be used iteratively in a fit.
There are various methods to overcome this shortage: using the "k-factors" approximation from lower to higher order,  or using the grid technique when available.
  

The leading order Drell-Yan~\cite{Drell:1970wh,Yamada:1981mw} cross section 
for the neutral current, triple differential in
invariant mass \(M\), boson rapidity \(y\) and CMS
lepton scattering angle \(\cos\theta\), can be written as
\begin{align}
\frac{\mathrm{d}^3\sigma}{\mathrm{d}M\mathrm{d}y\mathrm{d}\cos\theta} &=
  \frac{\pi\alpha^2}{3MS}\sum_{q}P_q
  \left[F_q(x_1,Q^2)F_{\bar{q}}(x_2,Q^2) + (q\leftrightarrow\bar{q})\right],
\end{align}
where \(S\) is a squared CMS beam energy, \(x_{1,2} = \frac{M}{\sqrt{S}}\exp(\pm y)\) and 
\begin{align}
  P_q &=  e_l^2e_q^2(1+\cos^2\theta) \nonumber \\
      &+  e_le_q\frac{2M^2(M^2-M_Z^2)}{\sin^2\theta_W\cos^2\theta_W
          \big[(M^2-M_Z^2)^2+\Gamma_Z^2M_Z^2\big]}
          \big[aA_q(1+\cos^2\theta)+2bB_q\cos\theta\big] \nonumber \\
      &+  \frac{M^4}{\sin^4\theta_W\cos^4\theta_W
          \big[(M^2-M_Z^2)^2+\Gamma_Z^2M_Z^2\big]}
          \big[(a^2+b^2)(A_q^2+B_q^2)(1+\cos^2\theta)+8abA_qB_q\cos\theta\big].
\end{align}
Here \(\theta_W\) is the Weinberg angle, \(M_Z\) and \(\Gamma_Z\) are Z boson mass and 
width, and
\begin{align}
 a & = -\frac{1}{4} + \sin^2\theta_W,  \nonumber \\
 b & = -\frac{1}{4},  \nonumber \\
 A_q & = \frac{1}{2}I_q^3-e_q\sin^2\theta_W, \nonumber \\
 B_q & = \frac{1}{2}I_q^3,  \nonumber \\
 I_u^3 & = -I_d^3 = \frac{1}{2},  \nonumber \\
 e_l & = -1, e_u = \frac{2}{3}, e_d = -\frac{1}{3}.
\end{align}

The expression for charged current has simpler form:
\begin{align}
\frac{\mathrm{d}^3\sigma}{\mathrm{d}M\mathrm{d}y\mathrm{d}\cos\theta} &=
 \frac{\pi\alpha^2}{48S\sin^4\theta_W}
 \frac{M^3(1-\cos\theta)^2}{(M^2-M_W^2)+\Gamma_W^2M_W^2}
 \sum_{q_1,q_2}V_{q_1q_2}^2F_{q_1}(x_1,Q^2)F_{q_2}(x_2,Q^2),
\end{align}
where \(V_{q_1q_2}\) is the CKM quark mixing matrix and \(M_W\) and \(\Gamma_W\)
are \(W\) boson mass and decay width.

The simple form of these expressions allows to calculate integrated
cross sections without utilization of Monte-Carlo techniques.
This is particularly useful for PDF fitting purposes because
the statistical fluctuations are avoided in this case. In both 
neutral and charge current expressions the parton density functions
factorize as a function dependent only on boson rapidity \(y\) and
invariant mass \(M\) leaving \(\cos\theta\) dependence aside.
The integral in \(\cos\theta\) can be computed analytically and
integrations in \(y\) and \(M\) can be performed with Simpson
method. The \(\cos\theta\) parts are kept in the equation 
explicitly because their integration is asymmetric for
data in lepton \(\eta\) bins and also is being performed when applying 
the lepton \(p_{\perp}\) cuts.

The fact that PDF functions factorize with the rest part of 
expression allows to significantly boost calculations when 
performing parameter fits over lepton rapidity data. In this case
the factorized part of expression independent on PDFs can be
calculated only once for all minimization iterations.
The leading order code in HERAFitter package implements this 
optimization and uses fast convolution routines provided by
QCDNUM. Currently the full width LO calculations are optimized 
for lepton pseudorapidity and boson rapidity distributions with
possibility to apply lepton \(p_{\perp}\) cuts.

The calculated leading order cross sections are multiplied by
NLO or NNLO K-factors provided for corresponding data distributions.
%%%%


On the other hand, one can obtain directly the NLO predictions by using APPLGRID or FASTNLO techniques, which rely on factorisation theorem by decoupling the hard scattering coefficients from PDFs. Therefore, the calculated hard scattering coefficients can be stored into a grid for a given kinematic bin, speeding up the convolution process with the PDFs allowing to be used for QCD fits. 
These method are described in more details in sections \ref{sec:theory:jets}

%%%%%%%%%%%
\subsection{\texorpdfstring{$t\bar{t}$}{t-tbar} Cross Sections via {\tt HATHOR}}
Top-quark pairs ($t\bar{t}$) are mainly produced via $gg$ fusion and
$q \bar q$ annihilation. Furthermore, there are the $q q'$ and
$q g$ production modes.
The program HATHOR~\cite{Aliev:2010zk} allows calculating
the expected total $t \bar t$ cross section at hadron colliders
($p \bar p$ and $p p$) up to approximate NNLO accuracy.
Version 1.3 of HATHOR includes the exact NNLO for $q \bar q \to t \bar t$ \cite{Baernreuther:2012ws}
as well as a new high-energy constraint on the approximate NNLO obtained from
soft-gluon resummation \cite{Moch:2012mk}.
The default choice for renormalization and factorization scale in $t \bar t$ production is the top-quark mass, $m_t$.
The pole mass scheme is typically employed for $m_t$ but HATHOR also supports calculations in
the $\overline{\text{MS}}$ scheme.
\subsection{Jets}\label{sec:theory:jets}
The calculation of higher order jet cross sections is very demanding
in means of computing power. The reasons are the large number of contributing
Feynman diagrams and also the large number of infrared divergencies.
For an accurate cancellation of these singularities, typically, the 
dipole subtraction method is applied in such calculations.
During the necessary Monte Carlo integration a very fine phase
space sampling has to be performed in order to account for the
accurate cancellation of the counter terms.

In order to enable the inclusion of jet-cross section 
measurements in PDF and $\alpha_s$ fits, these perturbative
coefficients have to be pre-computed in a PDF and $\alpha_s$ 
independent way. For this purpose, two quite similar tools are
interfaced to the HERAFitter.

\subsubsection{FastNLO}
The fastNLO project~\cite{Kluge:2006xs,Wobisch:2011ij,Britzger:2012bs}
enables the inclusion of jet data in PDF and $\alpha_s$ fits.
This tool uses multi-dimensional interpolation
techniques to convert the convolutions of perturbative 
coeffcients with parton distribution functions and 
the strong coupling into simple products.
Although the concept is process independent, the perturbative 
coefficients are usually calculated by the \texttt{NLOJET++}
program~\cite{Nagy:1998bb} where calculations for jet-production
in DIS~\cite{Nagy:2001xb}  as well as in hadron-hadron 
collisions~\cite{Nagy:2003tz,Nagy:2001fj} are available.
Also threshold-corrections of $\mathcal{O}$(NNLO) for 
inclusive jet cross sections in hadron-hadron collisions are
available~\cite{Kidonakis:2000gi}.

The fastNLO libraries are standardized included in the HERAFitter
package and no further requirements or compilation options
are needed. In order to include a new measurement into the PDF-fit,
the fastNLO table have to be specified. These tables include all
necessary informations of the perturbative coefficients and the
calculated process for all bins of a certain dataset. 
Tables for almost all published jet measurements
are available through the project website {\tt http://fastnlo.hepforge.org},
or have otherwise to be calculated by using the full fastNLO package.

Features of the fastNLO concept are the very quick convolution of the
perturbative coefficients with the PDFs of
$\mathcal{O}(100 ms)$ and the very high accuracy
of the interpolation procedure. 
The fastNLO tables are conventionally calculated
for multiple factors of the factorization scale, 
and the renormalization scale factor can be choosen freely.
Some of the fastNLO tables already involve a scale-independent
concept~\cite{Britzger:2012bs}, which allows for 
the free choice of the renormalization and the factorization
scale as a function of two pre-defined observables.
The evaluation of the strong coupling constant, which enters
the cross section calculation, is taken consistently from the 
QCDNUM evolution code.


\subsubsection{APPLGRID}
The APPLGRID~\cite{Carli:2010rw} package allows to compute a fast estimate
of NLO cross section for particular processes for arbitrary set of 
proton parton density functions. The package implements
calculation of cross section of electroweak boson (\(Z,W\))
production as well as jet production in proton-(anti)proton
collisions and DIS processes. 

The approach is based on storing perturbative coefficients
of NLO QCD calculations of final-state observables measured
in hadron colliders in look-up tables. The PDFs and the 
strong couplings are included during the final calculations,
e.g. during the PDF fits procedure. The method allows 
variation of factorization and renormalization scales in
calculations.

The look-up tables (grids) can be generated with modified versions of
of MCFM~\cite{Campbell:1999ah,Campbell:2010ff} or 
NLOjet++~\cite{Nagy:2001fj} software distributed
with the full version of APPLGRID package. NLO calculations
for the current analysis are performed with the help of APPLGRID
generated grids based on MCFM calculations. 

Run parameters
and electroweak parameters are set in MCFM in a standard way
via the input file and user's part of the code. 
Binning and definitions of the observables for which the
differential cross sections are needed are set in the 
APPLGRID code. 
The grid parameters \(x_1, x_2\) and \(Q^2\) binning
and interpolation orders are also defined in the code.

APPLGRID performs construction of the look-up tables in two 
steps: {\it (i)} exploration of the phase space in order
to optimize the memory storage and {\it (ii)} actual grid
construction in the phase space corresponding to the 
requested observables.

Afterwards the NLO cross sections are restored from the grids
with providing PDFs, \(\alpha_S\), factorization and 
renormalization scales and with QCD NNLO k-factors applied
if stated.

%%%%%%%%%%%
\subsection{DIPOLE models}

%At low $x$ and low $Q^{2}$, virtual photon-proton scattering is described using the colour
%dipole model formalism~\cite{NNZ:91}. Within this formalism, the scattering process is calculated as a fluctuation of the
%photon into a quark-antiquark pair (dipole), with a lifetime $\propto\enskip 1/x$, which interacts with the proton.
%
%Several approaches have been developed to phenomenologically describe the dipole-proton interaction
%cross section, three of which are implemented in the HERAFitter. These are
%the original model version (GBW)~\cite{Golec-Biernat:1998js}, a model based on the colour glass condensate approach
%to the high parton density regime (IIM)~\cite{Iancu:2003ge}, and a modified GBW model by adding effects of the 
Dipole picture provides an attractive approach to the virtual photon-proton scattering in the low $x$ region because it allows to describe inclusive and diffractive processes together. In this approach the virtual photon fluctuates into a $q\bar q$ (or $q\bar q g$ ....)  dipole which interacts with the proton~\cite{NNZ:91}.  The dipoles can be viewed as quasi-stable quantum mechanical states, which have very long life time $\propto 1/m_p x\;$ and a size which is not changed by scattering. 

Several dipole models have been developed to describe various DIS reactions. They vary due to different assumption made about the behavior of the dipole cross sections.   In the HERAFitter  three representative models  are implemented.
\begin{itemize}
\item
the original (GBW)~\cite{Golec-Biernat:1998js} dipole saturation model,
\item
  the colour glass condensate approach
to the high parton density regime (IIM)~\cite{Iancu:2003ge},
\item
  a modified GBW model which takes into account the effects of  
DGLAP evolution (BGK)~\cite{Bartels:2002cj}.
\end{itemize}
%%%%
\subsubsection{GBW model}
In the GBW model the dipole-proton cross section $\sigma_{\text{dip}}$ is given by
\begin{equation}
\label{eGBW}
   \sigma_{\text{dip}}(x,r^{2}) = \sigma_{0} \left(1 - \exp \left[-\frac{r^{2}}{4R_{0}^{2}(x)} \right]\right),
\end{equation}
where $r$ corresponds to the transverse separation between the quark and the antiquark, and $R_{0}^{2}$ is 
%an $x$ dependent scale parameter, having the form $R_{0}^{2}(x)=\left(x/x_{0}\right)^{\lambda}$.
an $x$ dependent scale parameter which has a meaning of saturation radius,  $R_{0}^{2}(x)=\left(x/x_{0}\right)^{\lambda}$.
The free fitted parameters are the cross-section normalisation $\sigma_{0}$ as well as $x_{0}$ and $\lambda$.
%%%%
\subsubsection{IIM model}
The IIM model assumes an improved expression for dipole cross section which is based on the 
Balitsky-Kovchegov equation~\cite{Balitsky:1995ub}. The explicit formula for $\sigma_{\text{dip}}$ 
can be find in~\cite{Iancu:2003ge}. The free fitted parameters are the alternative scale parameter $\tilde{R}$, $x_{0}$ and $\lambda$.
%%%%
\subsubsection{BGK model}
%The BGK model modifies the equation (\ref{eGBW}) by incorporating the LO and NLO DGLAP evolution
%of the gluon distribution. This leads to the expression for the dipole cross section
The BGK model modifies the GBW model by taking into account the  DGLAP evolution
of the gluon density. The dipole cross section is given by
\begin{equation*}
\label{eBGK}
   \sigma_{\text{dip}}(x,r^{2}) = \sigma_{0} \left(1 - \exp \left[-\frac{\pi^{2} r^{2} \alpha_{s}(\mu^{2}) xg(x,\mu^{2})}{3 \sigma_{0}} \right]\right).
\end{equation*}
The factorization scale $\mu^{2}$ has the form $\mu^{2} = C_{bgk}/r^{2}+\mu^{2}_{0}$.
%This model uses the following gluon density at the starting scale $Q_{0}^{2}=1\mbox{ GeV}^{2}$
In this model the gluon density, which  is parametrized  at some starting scale $Q_{0}^{2}$ by
\begin{equation*}
\label{eqTH730}
   xg(x,Q^{2}_{0}) = A_{g} x^{-\lambda_{g}}(1-x)^{C_{g}}.
\end{equation*}
is evolved to larger $Q^2$'s using LO and NLO DGLAP evolution.
The free fitted parameters for this model are $\sigma_{0}$, $\mu^{2}_{0}$ and 3 parameters for gluon $A_{g}$, $\lambda_{g}$, $C_{g}$. The parameter $C_{bgk}$ is kept fixed: $C_{bgk} = 4.0$. 
%%%%
%\newpage
%\subsubsection{Mixed with DGLAP model}
\subsubsection{BGK model with valence quarks}
The dipole models are valid in the low-$x$ region only, where the valence quark contribution is small, of the order of 5\%. The new HERA $F_2$ data have a precision which better than 2 \%. Therefore in the HERAfitter the contribution of the valence quarks is taken from the pdf fits and added to the original 
 BGK model. The quality of the fits of the BGK dipole model with valence quarks and without valence quarks are the same.
The default  initial parameters for the fit without valence quark are : 
\begin{table}[h]
\begin{center}
\begin{tabular}{|c||c||c||c|c||c|c|c||c|c|} 
\hline 
$\sigma_0$ & $A_g$ & $\lambda_g$ & $C_g$ & $cBGK$& $eBGK$\\
\hline
37.490 & 3.3446 & 0.0298 & 2.6302 & 4.0 & 15.362 \\
\hline
\end{tabular}
\end{center}
\end{table}
\\
For the BGK dipole model fits with valence quarks the initial parameters and the obtained $\chi^2$ are:
\begin{table}[ht]
\begin{center}
\begin{tabular}{|c||c||c||c|c||c|c|c||c|c|c||c|} 
\hline 
No& 
$Q^2$&&
$\sigma_0$ & $A_g$ & $\lambda_g$ & $C_g$ & $cBGK$& $eBGK$& $Np$& $\chi^2$& $\chi^2/Np$\\
\hline
1 &
$Q^2 \ge 3.5$ & NLO & 35.980 & 1.964 &-0.147& 3.068& 4.0 & 15.171 & 196& 245.74& 1.254 \\
\hline
2 &
$Q^2 \ge 8.5$ & NLO & 27.820& 3.660 &-0.076& 8.405& 4.0 & 18.188 & 157& 128.92&  0.821 \\
\hline
\end{tabular}
\end{center}
\end{table}
%%%%%%%%%%%
%\subsection{Unintegrated PDFs using CASCADE}
\subsection{TMD (unintegrated PDF) with CCFM}
\def\kt{\ensuremath{k_t}}
\newcommand{\Pmax}{p}
\newcommand{\CCFM}{CCFMa,CCFMb,Catani:1989sg,CCFMd}

In high energy factorization \cite{Catani:1990eg} the cross section is written as a convolution of the partonic cross section $\hat{\sigma}(� \kt)$ which depends on the transverse momentum $\kt$ of the incoming parton with the $\kt$-dependent parton density function ${\cal \tilde A}\left(x,\kt,\Pmax\right)$ (transverse momentum dependent (TMD) or unintegrated uPDF):
\begin{equation}
 \sigma  = \int 
\frac{dz}{z} d^2k_t \hat{\sigma}(\frac{x}{z},k_t)  {\cal \tilde A}\left(x,\kt,\Pmax\right)\label{kt-factorisation}
\end{equation}
The evolution of ${\cal \tilde A}\left(x,\kt,\Pmax\right)$ 
can proceed via the BFKL, DGLAP or via the CCFM evolution equations. Here, an extension of the CCFM \cite{\CCFM} evolution is applied. Since the evolution cannot be easily obtianed in  a closed form, 
 first a kernel $ {\cal \tilde A}\left(x'',\kt,\Pmax\right) $ is determined from the MC solution of the CCFM evolution equation, and then is folded with the non-perturbative starting distribution ${\cal A}_0 (x)$ \cite{Jung:2012hy}:
\begin{eqnarray}
x {\cal A}(x,\kt,\Pmax) &= &x\int dx' \int dx'' {\cal A}_0 (x) {\cal \tilde A}\left(x'',\kt,\Pmax\right)  \delta(x' \cdot x'' - x) \\
&= &\int dx' \int dx'' {\cal A}_0 (x) {\cal \tilde A}\left(x'',\kt,\Pmax\right) \frac{x}{x'} \delta(x'' - \frac{x}{x'}) \\
& = & \int dx' {{\cal A}_0 (x') }  
\cdot \frac{x}{x'}{ {\cal \tilde A}\left(\frac{x}{x'},\kt,\Pmax\right) } 
\end{eqnarray}
%An intrinsic $\kt$ dependence is included in the kernel ${\cal \tilde A}$
%\begin{eqnarray}
%{\cal \tilde A} & = & {\cal \tilde A'} \cdot f(k_{t\;0}) = {\cal \tilde A'} \cdot  \exp\left[ 
%-\frac{(\mu-k_{t\;0})^2}{\sigma^2}\right]
%\end{eqnarray}
The kernel  ${\cal \tilde A}$ includes all the dynamics of the evolution, Sudakov form factors and splitting functions and is determined in a grid of $50\otimes50\otimes50$ bins in $x,\kt,\Pmax$.  

The calculation of the cross section according to eq.(\ref{kt-factorisation}) involves a multidimensional Monte Carlo integration which is time consuming and suffers from numerical fluctuations, and cannot be used directly in a fit procedure involving the calculation of numerical derivates in the search for the minimum. Instead the following procedure is applied:
\begin{eqnarray}
\sigma_r(x,Q^2) & = & \int_x^1 d x_g {\cal A}(x_g,\kt,\Pmax) \hat{ \sigma}(x,x_g,Q^2) \\
  & = & \int_x^1 dx' {\cal A}_0 (x') \cdot \tilde{ \sigma}(x/x',Q^2) \label{final-convolution}
 \end{eqnarray}

The kernel ${\cal \tilde A}$ has to be provided separately and is not calculable within this program. The starting distribution  ${\cal A}_0$  at the starting scale $Q_0$ of the following form is used:
\begin{eqnarray}
x{\cal A}_0(x,\kt) &=& N x^{-B_g} \cdot (1 -x)^{C_g}\left( 1 -D_g x\right) 
\label{a0}
\end{eqnarray}
with free parameters $N,\, B_g,\, C_g,\, D_g$. 

In the present version, only the transverse momentum dependent gluon distribution can be obtained from the fit. 

The calculation of the $ep$ cross section follows eq.(\ref{kt-factorisation}), with the off-shell matrix element including quarks masses taken from \cite{Catani:1990eg} in its implementation in {\tt CASCADE} \cite{Jung:2010si} .In addition to the boson gluon fusion process, also valence quark initiated $\gamma q\to q$ processes are included, with the valence quarks taken from~\cite{Deak:2010gk}.

Please note that in the present version only DIS $ep$ processes can be used to determine the transverse momentum dependent (uPDF) gluon density distribution.

%%%%%%%%%%%
\subsection{Diffractive PDFs}
Diffractive DIS data are fitted within the 'proton vertex factorisation' approach where 
the diffractive DIS is mediated by the exchange of hard Pomeron and a secondary Reggeon.
The model supplied by the DiffDIS package provides values of the 'reduced cross section',
$\sigma_r = F_2 - y^2/(1+(1-y)^2) F_L$
which is expected to be the experimentally measured quantity.

\makeatletter
\def\comsp{\@ifnextchar,\relax{\@ifnextchar\ \relax{\@ifnextchar:\relax{\@ifnextchar.\relax\ }}}}
\makeatother
\DeclareRobustCommand{\ie}{{\it i.e.}\comsp}
\DeclareRobustCommand{\eg}{{\it e.g.}\comsp}
\DeclareRobustCommand{\cf}{{\it cf.}\comsp}
\DeclareRobustCommand{\etal}{{\it et al.}\comsp}
\DeclareRobustCommand\bs{\ensuremath{\backslash}}
\newcommand\ssp{\ifmmode\relax\else\comsp\fi}

%\newcommand\Eq[1]{Eq.~(\ref{#1})}
\newcommand\Eq[1]{(\ref{#1})}
\newcommand\Fig[1]{Fig.~\ref{#1}}
\DeclareRobustCommand\NF{\ensuremath{N_{\rm f}}\ssp}
% \DeclareRobustCommand\NC{\ensuremath{N_{\rm c}}\ssp}
\DeclareRobustCommand\GeV{\ensuremath{{\rm GeV}}\ssp}
% \DeclareRobustCommand\Vstat{\ensuremath{V^{\rm (stat)}}\ssp}
% \DeclareRobustCommand\Vsys{\ensuremath{V^{\rm (sys)}}\ssp}
\DeclareRobustCommand\FL{\ensuremath{F_{\mathrm{L}}}\ssp}
\DeclareRobustCommand\FT{\ensuremath{F_{\mathrm{T}}}\ssp}
\newcommand\AP {{\cal P}}

\def \beq{\begin{equation}}
\def \eeq{\end{equation}}
\def \beqa{\begin{eqnarray}}
\def \eeqa{\end{eqnarray}}
\def \beqal{\begin{subequations}\begin{eqnarray}}
\def \eeqal{\end{eqnarray}\end{subequations}}

\let\optspace=\ssp
% \newcommand{\xh}{\hat x}
\DeclareRobustCommand{\as}[1]{\ensuremath{\alpha_{\rm s}(#1^2)}\optspace}
\newcommand{\asotp}{\ensuremath{\frac{\alpha_{\rm s}}{2\pi}}\optspace}
\newcommand{\Sgl}[1]{\ensuremath{\tilde f_{#1+}}\optspace}
\newcommand{\Pom}{{I\!P}}
\newcommand{\Reg}{{I\!R}}
% \newcommand{\xP}{x_\Pom}
\newcommand{\xP}{\xi}
\newcommand\sigRed{\ensuremath{\overline\sigma}}
\newcommand\DX{\ensuremath{\mathcal{X}}}

%\parindent=0pt
%\parskip=4pt

%\graphicspath{{figs/}}

%==========================================
\subsubsection {Cross-section}

\beq
  \frac{d\sigma}{d\beta\,dQ^2\,d\xP\,dt}
=
  \frac{2\pi\alpha^2}{\beta Q^4}\,
    \left( 1 +  (1-y)^2 \right) \sigRed^{D(4)}(\beta,Q^2,\xP,t)
\label{Dxs}
\eeq
where the `reduced cross-section', \sigRed, is defined as
\beq
\label{eq:sigred}
\sigRed
 = F_2 - \frac{y^2}{1 +  (1-y)^2}\, \FL
 = \FT + \frac{2(1-y)}{1 +  (1-y)^2}\, \FL
\eeq
Nb. $\xi$ is denoted by $x_\Pom$ in the H1 and ZEUS papers.

The dimension of 
\(
F_k^{D(4)}(\beta,Q^2,\xP,t)\)
is $\GeV^{-2}$
and
thus the quantities integrated over $t$
\beq
F_k^{D(3)}(\beta,Q^2,\xP)
\equiv
\int_{t_{\rm min}}^{t_{\rm max}} dt
F_k^{D(4)}(\beta,Q^2,\xP,t)
\eeq
are dimensionless.

The maximum kinematically allowed value of $t$ is given by
\begin{equation}
t_{\rm MAX} 
=
-\frac{\xP^2 m_p^2 + p_\perp^2}{1-\xP}
\approx 
-\frac{\xP^2}{1-\xP} m_p^2
\end{equation}
where $m_p$ is the proton mass.

As $x = \xP\beta$ we can normalize to the standard DIS formula
\begin{equation}
\frac{d\sigma}{d\beta\,dQ^2\,d\xP\,dt} =
  \frac{2\pi\alpha^2}{x\, Q^4}\,
    \left( 1 +  (1-y)^2 \right) \xP\sigRed^{D(4)}(\beta,Q^2,\xP,t)
\end{equation}
which upon integration over $t$ reads
\begin{equation}
\label{Dxs3}
  \frac{d\sigma}{d\beta\,dQ^2\,d\xP}
=  
  \frac{2\pi\alpha^2}{x Q^4}\,
    \left( 1 +  (1-y)^2 \right) \,\xi\sigRed^{D(3)}(\beta,Q^2,\xP)
\end{equation}


The H1 and ZEUS data files typically contain $\xP\sigRed^{D(3)}$.

%==========================================
\subsubsection {Regge factorization}

For better data description we include a contribution from a secondary Reggeon, $\Reg$,
\beq
F_k^{D(4)}(\beta,Q^2,\xP,t) = 
\sum_{\mathcal{X} =\Pom,\Reg}
\phi_\mathcal{X}(\xP,t)\, F^\mathcal{X}_k(\beta,Q^2)
\eeq

or
\beq
\label{eq:FD3}
F_k^{D(3)}(\beta,Q^2,\xP) = 
\sum_{\mathcal{X} =\Pom,\Reg}
\Phi_\mathcal{X}(\xP)\, F^\mathcal{X}_k(\beta,Q^2)
\eeq
where
\begin{equation}
\label{eq:intFlux}
\Phi_{\mathcal{X}}(\xP) =
\int\limits_{t_{\rm min}}^{t_{\rm max}} dt\, \phi_\mathcal{X}(\xP,t)
\,.
\end{equation}

Parametrization of the fluxes
\begin{subequations}
\label{eq:flux}
\begin{equation}
\phi_\mathcal{X}(\xP,t) = 
\frac {A_\mathcal{X}\, e^{b_\mathcal{X} t}} {\xP^{2\alpha_\mathcal{X}(t) -1}}
\end{equation}
where
\begin{equation}
\alpha_\mathcal{X}(t) = \alpha_\mathcal{X}(0) + \alpha_\mathcal{X}' t
\,.
\end{equation}
\end{subequations}

$F^\Reg_k(\beta,Q^2)$ are taken as those of the pion.
%  with the normalization factor being absorbed in $\phi_\Reg(\xP,t)$.


%==========================================
%\subsubsection {Pomeron parametrization}

%The Pomeron is parametrized at the initial
%$Q_0^2$ in terms of two singlet distributions,
%$f_{g}$ and $f_{+}$.
%\begin{subequations}
%\label{singlet}
%\begin{eqnarray}
%\frac{d}{dt}f_{+} &=&
%\asotp\left[
%\AP_{\rm FF} f_{+} +\AP_{\rm FG}f_g
%\right]
%\\
%\frac{d}{dt}f_{g} &=&
%\asotp\left[
%\AP_{\rm GF} f_{+} +\AP_{\rm GG}f_g
%\right]
%\end{eqnarray}
%\end{subequations}

%As $\Pom$ is neutral, $f_{q} = f_{\bar q}$ for each flavour $q$.
%Assuming that all light quark PDFs are equal
%\begin{equation}
%f_d = f_u = f_s
%\,,
%\end{equation}
%we have
%\begin{subequations}
%\label{eq:pm}
%\begin{eqnarray}
%f_{q-} &\equiv& 0
%\\
%f_{q+} &\equiv& 2 f_q
%\end{eqnarray}
%\end{subequations}
%
%At \NF = 3
%\begin{equation}
%\label{eq:fq3}
%f_{q+} = f_{+}/3,\; q = d,u,s
%\,.
%\end{equation}
%% \ifFullVer
%\ie
%% \begin{equation}
%% \Sgl q = 0,\; q = d,u,s
%% \,,
%% \end{equation}
%% where
%\begin{equation}
%\Sgl q \equiv f_{q+} - \frac{1}{\NF}f_{+}
% = 0,\;\mbox{for}\; q = d,u,s
%\,.
%\end{equation}
% \fi
%
%This gives all PDFs for the FFNS, while for VFNS 
%$f_{h+}$ for $h=c,b,t$ are generated dynamically above the respective
%transition scales $Q_h^2$.
%Hence at $\NF > 3$ the singlet has contributions from the heavy quarks
%and we get non-trivial nonsinglet distributions $\Sgl{h}$ satisfying
%\begin{equation}
%\label{eq:nsevol}
%\frac{d}{dt}\Sgl h = \asotp\, \AP_{(+)} \Sgl h
%\end{equation}
%
%%\subsection {Parametrization at \texorpdfstring{$Q_0^2$}{Q0}}
%%\subsubsection {Parametrization at {$Q_0^2$}}
%{\bf Parametrization at {$Q_0^2$}} \\
%%-----------------------------------------------------------
%\label{sec:Par}
%
%Full PDFs are given in analogy to \Eq{eq:FD3}
%\begin{equation}
%f_k^{D(3)}(\beta,Q^2,\xP) =
%\hat\Phi_\Pom(\xP)\, f^\Pom_k(\beta,Q^2)
%+
%\Phi_\Reg(\xP)\, f^\Reg_k(\beta,Q^2)
%\end{equation}
%where $\hat\Phi_\Pom \equiv \Phi_\Pom/A_\Pom$,
%with the fluxes given by \Eq{eq:intFlux} and \Eq{eq:flux}.
%
%The Pomeron PDFs are parametrized as
%\def\Cini#1#2{A^{(#1)}_#2}
%\begin{equation}
%\label{eq:fP0}
%f^\Pom_N = \Cini N1  x^{\Cini N2} (1-x)^{\Cini N3}
%%   \left(1 + \Cini N4 x \right)
%  \; \exp\left(-\frac{d}{1.00001-x}\right)
%\,,
%\end{equation}
%where the `damping factor' $d$ is taken as 0.01 or 0.001.
%$N = \mathrm G$ for gluon and $N = \mathrm S$ for `singlet': $f_{\rm S} \equiv f_+(\NF=3)$,
%\cf \Eq{eq:fq3}.

% The Reggeon PDFs $f^\Reg_k$ are taken from pion.

%%\subsubsection {HERAFitter parameters}
%{\bf HERAFitter parameters} \\
%%-----------------------------------------------------------
%\label{sec:HFitterPar}
%
%% minuit.in.txt
%% ExtraMinimisationParameters
%
%\begin{tabular}{l|l|l}
%Parameter & HERAFitter name & input file\\
%% \hline
%$\Cini {\mathrm G}1$ & Ag & minuit.in.txt \\
%$\Cini {\mathrm G}2$ & Bg & minuit.in.txt \\
%$\Cini {\mathrm G}3$ & Cg & minuit.in.txt \\
%$\Cini {\mathrm S}1$ & Auv & minuit.in.txt \\
%$\Cini {\mathrm S}2$ & Buv & minuit.in.txt \\
%$\Cini {\mathrm S}3$ & Cuv & minuit.in.txt \\
%$\alpha_\Pom(0)$ & Pomeron\_a0 & steering.txt \\
%$A_\Reg$ & Reggeon\_factor & steering.txt \\
%$\alpha_\Reg(0)$ & Reggeon\_a0 & steering.txt \\
%\end{tabular}

\endinput


        



%%%%%%%%%%%%%%%%%%%%%%%%%%%%%
\section{Bayesian Reweighting Technique}
In this section a different approach to PDF studies based on the reweighing techniques is described. 
Bayesian reweighting of PDF sets is a way to include new data into an existing PDF set without actually carrying out a full-blown fitting procedure. 
It was first suggested by Giele and Keller~\cite{Giele:1998gw} and first pursued in practice by the NNPDF Collaboration~\cite{Ball:2011gg,Ball:2010gb}. 
Watt and Thorne~\cite{Watt:2012tq} have also proposed a scheme for how to implement the Bayesian reweighting technique for PDF predictions based on central values with errors determined using the Hessian Eigenvector Method. 

allows these methods to be used to update any PDF that is available either as a probability distribution... or as a PDf eigenvector set

The \fitter package allows these methods to be used to update any PDF that is available either as a probability distribution
(i.e. a lhapdf .LHgrid file in NNPDF format) or as a PDf eigenvector set 
(i.e. any PDF set in lhapdf .LHgrid file format with errors determined using the Hessian Eigenvector Method).
This enables the user to assess the impact of new data not only on the {\tt HERAPDF} using the full-blown fit procedure 
but also on other standard PDF sets. This one can investigate how the data impact different PDF sets.

The Bayesian Reweighting technique essentially uses PDF probability distributions as input, applies weights to these distributions based on how well the new data is described and outputs an updated PDF probability distribution. In the following paragraphs, firstly the construction of these PDF probability distributions is described, then the calculation of the weights to update the PDF probability distribution is introduced and lastly, the configuration of the module within the \fitter framework is explained.

\subsection{PDF probability distributions}

PDF probability distributions are constructed as finite ensembles of $N_{\mathrm{rep}}$ parton distribution functions $\mathrm{PDF}_k$, $\mathcal{E} = \{PDF_k, k = 1, . . . ,N_{\mathrm{rep}}\}$. Observables $\mathcal{O}(\mathrm{PDF})$ are conventionally calculated from the average of the predictions obtained from the ensemble:

\begin{equation}
 \langle\mathcal{O}(\mathrm{PDF})\rangle = \frac{1}{N_{\mathrm{rep}}} \sum_{k=1}^{N_{\mathrm{rep}}} \mathcal{O}(\mathrm{PDF}_k)
\label{eq:meanReplicas}
\end{equation}
 
Their uncertainties are calculated as the standard deviation, defined as:

\begin{equation}
\sigma_{\mathcal{O}(\mathrm{PDF})} = \sqrt{  \frac{1}{N_{\mathrm{rep}} - 1 }  \sum_{k=1}^{N_{\mathrm{rep}}} 
( \mathcal{O}(\mathrm{PDF}_k) - \langle \mathcal{O}(\mathrm{PDF})  \rangle   )^2     
     }
\end{equation}

While the standard PDF sets from the NNPDF collaboration are already available as ensembles of parton distribution functions, the PDF predictions of other PDF fitting groups need to be converted to PDF probability distributions. This is possible provided that the PDF sets have associated uncertainties that can be used to create replicas of the central PDF set with random variations that lie within the uncertainties. 

In the case of uncertainties provided by standard Hessian eigenvector error sets, this can be easily achieved 
by creating the $k$-th random replica by introducing introducing random fluctuations around the central PDf set, $\mathrm{PDF}_0$.

If the PDF eigenvectors are asymmetric, that is they come in pairs of negative and positive PDF error sets, 
corresponding to negative and positive deviations from the central value, these random fluctuations can be created by 
drawing a random number $R_{jk}$ and adding, depending on the sign of the random number, 
the difference of the positive or respectively negative PDF of the $j$-th PDF eigenvector pair from the central value, 
scaled by the absolute value of the random number:

\begin{equation}
 \mathrm{PDF}_k = \mathrm{PDF}_0  + \sum_{j=0}^{n} \left[ \mathrm{PDF}^{\pm}_j - \mathrm{PDF}_0 \right] |R_{jk}|
\end{equation}
 
Here, $k$ denotes the number of the random replica and runs from $k=1, ... , N_\mathrm{rep}$; $j$ denotes the eigenvector pair and runs from $j=1, ..., n$, where $n$ is the number of eigenvectors, e.g. $n=20$ for MSTW08. 

In case that the Hessian eigenvectors have been symmetrised and only one error set is given per eigenvector, 
the above prescription simplifies to:

\begin{equation}
 \mathrm{PDF}_k = \mathrm{PDF}_0  + \sum_{j=0}^{n} \left[ \mathrm{PDF}_j - \mathrm{PDF}_0 \right] R_{jk}
\end{equation}

\subsection{Bayesian Reweighting of PDF sets}

Once PDF probability distributions are available as inputs, they can be updated to incorporate the new data. This is achieved by applying weights to the PDF probability distributions such that the prediction for observable $\langle\mathcal{O}(\mathrm{PDF})\rangle$ from equation \ref{eq:meanReplicas} changes to:

\begin{equation}
 \langle\mathcal{O}^{\mathrm{new}}(\mathrm{PDF})\rangle = \frac{1}{N_{\mathrm{rep}}} \sum_{k=1}^{N_{\mathrm{rep}}} w_k \mathcal{O}(\mathrm{PDF}_k)
\end{equation}

The weights $w_k$ calculated are here according to:

\begin{equation}
 w_k = \frac{(\chi^2_k)^{\frac{1}{2} (N_{\mathrm{data}}-1) } \exp^{-\frac{1}{2}\chi^2_k}}{ \frac{1}{N_{\mathrm{rep}}} \sum^{N_{\mathrm{rep}}}_{k=1}(\chi^2_k)^{\frac{1}{2}(N_{\mathrm{data}}-1)} \exp^{-\frac{1}{2}\chi^2_k}  },
\end{equation}

where $N_{\mathrm{data}}$ is the number of new data points, $k$ denotes the specific replica for which the weight is calculated and $\chi^2_k$ is between a given data point $y_i$ and its theoretical prediction obtained with the $k$-th PDF replica:

\begin{equation}
 \chi^2 (y,\mathrm{PDF}_k) = \sum_{i,j=0}^{N_{\mathrm{data}}} (y_i - y_i(\mathrm{PDF}_k)) \sigma^{-1}_{ij} (y_j-y_j(\mathrm{PDF}_k))  
\end{equation}

The weighted PDF probability distribution can be turned into a new ensemble of PDF replicas, based on which predictions for any observable can be calculated. This new, reweighted PDF probability distribution commonly is chosen to be based upon a smaller number of PDF sets compared to the input PDF probability distribution, because replicas that are incompatible with the data are discarded 
in order to create a more stream-lined PDF set.

\subsection{Usage of the PDF reweighting in the \fitter framework}
 
The \fitter ramework can be used to apply PDF reweighting for
NNPDF-style PDF probability distributions as well as for PDF sets with Hessian PDF eigenvector error sets. 

This requires that the {\tt NNPDF reweight} and the {\tt LHAPDF} modules are installed, see sections \ref{sec:install_nnpdfrweight} and \ref{sec:install_lhapdf}. In the \fitter steering files, the PDF reweighting needs to be switched on and the relevant parameters have to be set: 

\begin{itemize}
 \item \textbf{FLAGRW}: (En/dis) able reweighting
 \item \textbf{RWPDFSET}: Name of the PDF set to be reweighted
 \item \textbf{RWDATA}: Arbitrary name for the data to be updated, used to create the names of the output PDF set and the directory
 \item \textbf{RWMETHOD}: Do the reweighting based on chi2 (method 1, where you read in \fitter data files and theory predictions and calculate the chi2 based on them) or on data (method 2, where you have to provide an input text file with theoretical predictions {--} the input format of this input file will be explained below)
 \item \textbf{DORWONLY}: Disable the usual PDF fit, such that only the reweighting is done 
 \item \textbf{RWREPLICAS}: Number of input replicas used for the PDF probability distributions (not applicable for NNPDF sets, since they come with a fixed number of replicas)
 \item \textbf{RWOUTREPLICAS}: Number of replicas in the output PDF set.
\end{itemize}

The setup of the module is such that it is parsing the \fitter steering file and from the specified settings it creates a special reweighting steering file in the directory {\tt input\_steering} with the pattern {\tt <RWPDFSET>\_<RWDATA>\_<RWMETHOD: chi2 or data>.in}. 

In the output directory, a sub-directory is created for the output of the reweighting procedure. Its name pattern is: {\tt output/<RWPDFSET>\_<RWDATA>\_<RWMETHOD: chi2 or data>/} and it will contain the following files:

\begin{itemize}
 \item {\tt <RWPDFSET>\_<RWDATA>\_<RWMETHOD: chi2 or data>\_nRep<RWOUTREPLICAS>.LHgrid}: The output PDF probability distribution in form of an .LHgrid file, which allows easy usage.
 \item {\tt whist-rw.eps}: Plot with the distributions of weights calculated for each replica. The meaning of this plot is further described in~\cite{Ball:2011gg,Ball:2010gb}.
 \item {\tt palpha-rw.eps}: Plot with the probability for each replica to describe the data. The meaning of this plot is further described in~\cite{Ball:2011gg,Ball:2010gb}.
 \item {\tt <RWPDFSET>\_<RWREPLICAS>InputReplicas.LHgrid}: This is the PDF probability function that has been produced from the eigenvector PDF sets produced by the Hessian method (not applicable for NNPDF sets).
\end{itemize}

 
%%%%%%%%%%%%%%%%%%%%%%%%%%%%%
\section{PDF Parameterisation}

\label{sec:pdfparam}
%%%%%%%%%%%
\subsection{Standard Functional form}
%%%%
Through standard functional form it is undertstood a simple polynomial 
that interpolates between the low and high $x$ regions:
\begin{equation}
 xf(x) = A x^{B} (1-x)^{C} P_i(x),
\label{eqn:pdf_std}
\end{equation}
We identify few standard forms commonly used by PDF groups.

%%%%
\subsubsection{CTEQ style}
%%%%
The notation used throughout this text reflects the 
notation used in the code.

\begin{equation}
 xf(x) = a_0 x^{(a_1+n)} (1-x)^{a_2} e^{a_3x} (1 + e^{a_4 x} + e^{a_5 x^2}),
\label{eqn:pdf_cteq}
\end{equation}
%
%%%%
\subsubsection{HERAPDF style}
%%%%
 The parametrised PDFs at HERA are the valence distributions
 $xu_v$ and  $xd_v$,  the gluon distribution $xg$, and the $u$-type and $d$-type 
$x\bar{U}$, $x\bar{D}$, where $x\bar{U} = x\bar{u}$, 
$x\bar{D} = x\bar{d} +x\bar{s}$. 
The following standard functional form is used to parametrise them
\begin{equation}
 xf(x) = A x^{B} (1-x)^{C} (1 + D x + E x^2),
\label{eqn:pdf}
\end{equation}
%
where the normalisation parameters, $A_{uv}, A_{dv}, A_g$,  are constrained by  
the QCD sum-rules, such that the counting  and  momentum conservation are preserved.
The $B$ parameters  $B_{\bar{U}}$ and $B_{\bar{D}}$ are set equal,
 $B_{\bar{U}}=B_{\bar{D}}$, such that 
there is a single $B$ parameter for the sea distributions. 
%
The strange quark distribution 
is already present at the starting scale and 
%
it is  assumed here that 
$x\bar{s}= f_s  x\bar{D}$ at $Q^2_0$. 
The  strange fraction is chosen to be $f_s=0.31$ which is
consistent with determinations 
of this fraction using neutrino induced di-muon production. 
%
In addition, to ensure that $x\bar{u} \to x\bar{d}$ 
as $x \to 0$,  
$A_{\bar{U}}=A_{\bar{D}} (1-f_s)$.
%
The $D$ and $E$ are introduced one by one until no further improvement in $\chi^2$ is found.
For the case when adding more precision data in the fit, as when adding HERA II data, this allows then for use of a more flexible parametrisation for the gluon and valence especially.
The best fit  results in a total of 10 free parameters when performing fits to solely HERA I data (fits are refered then ro as HERAPDF1.0), and of 13 free parameters when adding preliminary HERA II data on top (fits are refered then to as HERAPDF1.5).
\subsubsection{Flexible style}
%%%%
\subsection{Chebyshev Polynomial}

A flexible Chebyshev polynomials based parameterisation is used for the gluon and sea densities. The polynomials
use $\log x$ as an argument to emphasise the low $x$ behaviour. 
The parameterisation is valid for $x>x_{min} = 1.7\times 10^{-5}$. The PDFs are multiplied
by $1-x$ to ensure that they vanish as $x\to 1$. The resulting parameterisation form is 
\begin{eqnarray}
x g(x) &=& A_g \left(1-x\right) \sum_{i=0}^{N_g-1} A_{g_i} T_i \left(-\frac{\textstyle 2\log x - \log x_{min} } {\textstyle \log x_{min} } \right)\,, \label{eq:glu} \\
x S(x) &=& \left(1-x\right) \sum_{i=0}^{N_S-1} A_{S_i} T_i \left(-\frac{\textstyle 2\log x - \log x_{min} } {\textstyle \log x_{min} } \right)\,. \label{eq:sea} 
\end{eqnarray}
Here the sum over $i$ runs up to $N_{g,S}=15$ order Chebyshev polynomials of the first type $T_i$ for
the gluon, $g$, and sea-quark, $S$, density, respectively. 
The normalisation $A_g$ is given by the momentum sum rule.
The advantages of the parameterisation given by equations~\ref{eq:glu},\ref{eq:sea} is that momentum
sum rule can be evaluated analytically and  already for $N \ge 5$ the fit quality
is similar to a standard Regge-inspired parameterisation with a similar number of parameters.





A flexible Chebyshev polynomials based parameterisation is used for the gluon and sea densities. The polynomials
use $\log x$ as an argument to emphasise the low $x$ behaviour. 
The parameterisation is valid for $x>x_{min} = 1.7\times 10^{-5}$. The PDFs are multiplied
by $1-x$ to ensure that they vanish as $x\to 1$. The resulting parameterisation form is 
\begin{eqnarray}
x g(x) &=& A_g \left(1-x\right) \sum_{i=0}^{N_g-1} A_{g_i} T_i \left(-\frac{\textstyle 2\log x - \log x_{min} } {\textstyle \log x_{min} } \right)\,, \label{eq:glu} \\
x S(x) &=& \left(1-x\right) \sum_{i=0}^{N_S-1} A_{S_i} T_i \left(-\frac{\textstyle 2\log x - \log x_{min} } {\textstyle \log x_{min} } \right)\,. \label{eq:sea} 
\end{eqnarray}
Here the sum over $i$ runs up to $N_{g,S}=15$ order Chebyshev polynomials of the first type $T_i$ for
the gluon, $g$, and sea-quark, $S$, density, respectively. 
The normalisation $A_g$ is given by the momentum sum rule.
The advantages of the parameterisation given by equations~\ref{eq:glu},\ref{eq:sea} is that momentum
sum rule can be evaluated analytically and  already for $N \ge 5$ the fit quality
is similar to a standard Regge-inspired parameterisation with a similar number of parameters.


%%%%%%%%%%%%%%%%%%%%%%%%%%%%%%
\section{$\chi^2$ Definitions}
\label{sec:chi2}
%%%%%%%%%%%

For a single data set with diagonal statistical uncertainties, 
 the $\chi^2$ function can be defined as~\cite{H1:2009bp}
%
\begin{equation}
 \chi^2_{\rm exp}\left(\boldsymbol{m},\boldsymbol{b}\right) = %\\
%~~~=
 \sum_i
 \frac{\left[m^i
- \sum_j \gamma^i_j m^i b_j  - {\mu^i} \right]^2}
{ \textstyle \delta^2_{i,{\rm stat}}\left(m^i -  \sum_j \gamma^i_j m^i b_j\right)+
\left(\delta_{i,{\rm uncor}}\,  m^i\right)^2}
 + \sum_j b^2_j.
\label{eq:ave}\end{equation}
%
Here ${\mu^i}$ is the  measured central value  at a point $i$ 
with  relative statistical $\delta_{i,stat}$ 
and relative uncorrelated systematic uncertainty $\delta_{i,unc}$.
Further, $b_j$ denotes a nuisance parameter for
 a correlated systematic error  source of type $j$ with an uncertainty
 while
$\gamma^i_j$ 
quantifies the sensitivity of the
measurement ${\mu^i}$ at the point $i$ to the systematic source $j$. 
The function $\chi^2_{\rm exp}$ depends on the predicted values $m^i$ 
(denoted as the vector $\boldsymbol{m}$) and 
 the set of systematic uncertainties $b_j$ ($\boldsymbol{b}$).
The predicted values $m^i$ depend on the PDFs as well as other input
paremeter (e.g value of $\alpha_S$),  $m^i( \boldsymbol{p})$. 
In the following, absolute (relative) values of uncertainties are given
by capital (small) Greek symbols, e.g. $\Delta^i_{\rm stat}$ ( $\delta^i_{\rm stat} )$. 

This definition of the $\chi^2$ function assumes that
systematic uncertainties are proportional to the central values 
(multiplicative errors), whereas the statistical errors scale 
with the square roots of the expected number of events. 
Other scaling properties for the statistical and uncorrelated
systematic uncertainties are discussed later.
% available as described in appendix~\ref{sec:herafitter}.
%%%%

In the case of off-diagonal statistical uncertainties, the $\chi^2$ function
is
\begin{equation} \label{eq:chi2gen}
\chi^2_{\rm exp} (\boldsymbol{m},\boldsymbol{b}) = \sum_{ij} \left ( m^i - \sum_l \Gamma^i_l(m^i)b_l - \mu^i \right)
  C^{-1}_{{\rm stat.}~ij}(m^i,m^j) \left(  m^j - \sum_l \Gamma^j_l(m^j)b_l - \mu^j \right) + 
\sum_l b^2_l \,.
\end{equation}
Here the scaling properties of the correlated systematic uncertainties 
$\Gamma^i_j$ and
of the covariance matrix $C_{{\rm stat.}~ij}$ are expresses as a dependence
on $m_i$ and the dependence of $\Delta_{\rm stat}$ on $b_j$ is ignored.

Eq.~\ref{eq:chi2gen} allows for two methods for fast determination
of the minimum, without need to include the formal nuisance parameters
corresponding to the systematic error sources into the minuit minimisation.
In the first method, the minimisation vs. $b_j$ is used to define covariance
matrix for the systematic uncertainties which is determined as
\begin{equation}
 C_{{\rm syst}~ij}= \sum_l \Gamma^i_l \Gamma^j_l \,.
\end{equation}
The total covariance matrix is given by the sum of the statistical and
systamtic covariance matrices
\begin{equation} 
C_{{\rm tot}~ij} = C_{{\rm stat}~ij} + C_{{\rm syst}~ij}\,,
\end{equation}
and the $\chi^2$ function takes a form
\begin{equation}
  \chi^2( \boldsymbol{m}) = \sum_{ij} ( m^i - \mu^i) C^{-1}_{{\rm tot}~ij} 
( m^j - \mu^j)\,.
\end{equation}

The second methods is used to determine optimal shifts of the nuisance
parameters at each iteration. The shifts are given by minimising 
Eq.~\ref{eq:chi2gen} vs. $b_j$ which leads to a system of  linear equations 
\begin{equation}
 \sum_k \sum_{ij} C^{-1}_{{\rm stat}~ij} \Gamma^i_l \Gamma^j_k \cdot b_k = \sum_{ij} C^{-1}_{{\rm stat}~ij} \Gamma^i_l (m_i - \mu_i)\,,
\end{equation}
where $1\le l \le N_{\rm syst}$, the total number of correlated systematic uncertainties.

Finally the nuisance parameters $\boldsymbol{b}$ can be excluded from the $\chi^2$ minimisation.  
In this case, which is referred to as an Offset method, the minimum is determined for their values set to zero
while uncertainties on the parameters $\boldsymbol{p}$ are determined by shifting each nuisance parameter $b_l$
by $\pm 1$. The total covariance matrix for parameters $p^i$ is determined as 
\begin{equation}
  C^{\rm offset}_{ {\rm par}~ ij} = \sum_{l=1}^{N_{syst}} \Delta p^i_l \Delta p^j_l \,,
\end{equation}
where $ \Delta p^i_l = 0.5 ( p^i( b_l = +1 ) - p^i(b_l = -1))$ and the quality of the fit is estimated by 
fixing $\boldsymbol{p}$ to the value at the minimum and minimising with respect to $\boldsymbol{b}$

Finally, all three approaches can be combined together. For example, only some of the systematic uncertainties
can be treated using the matrix method while others can be treated using the hessian method. In this case, the
covariance matrix  $C_{\rm syst}$ is build using the corresponding sub-set of systematic sources and $C_{\rm stat}$ 
is replaced by $C_{\rm stat}+C_{\rm syst}$ in Eq.~\ref{eq:chi2gen}. Similarly, some of the systematic uncertainties
can be treated using offset method and then $C^{\rm total}_{ {\rm par}} = C^{\rm hessian}_{\rm par} + C^{\rm offset}_{\rm par}$
where offset and hessian covariance matrices are calculated using corresponding systematic error sources.

\subsection{Bias corrections}

The correlated and uncorrelated systematic uncertainties can be treated as additive,  $\Gamma^i_l(m^i) = \gamma^i_l \mu^i$
or multiplicative, $\Gamma^i_l(m^i) = \gamma^i_l m^i$. The LogNormal treatment in which 
$ \mu^i + \sum_l \Gamma^i_j b_l$ is replaced by $ \mu^i \prod_l \exp( \gamma^i_j b_l) $ is forseen for the
next release of the {\tt HERAFitter}. 

The statistical uncertainties can be treated as additive, $\Delta^i(m^i) = \delta^i \mu^i$  and as Poisson,
$\Delta^i(m^i) = \delta^i \sqrt{\mu^i m^i}$. More complex scaling from Eq.~\ref{eq:ave}, 
which depends on shifts of $b_j$, is implemented using an iterative approach: for the first iteration $b_l =0$ 
 is used to determine values of $b_l$ which are then applied in the second iteration. Statistical covariance
matrix is scaled in a similar manner. In this case the correlation matrix is assumed to be fixed, the diagonal
ellements are updated using the prescription describe above and the covariance matrix is rescaled accordingly.

The modifications of the covariance matrix at each iteration of the minuit minimisation may lead to systematic
biases. There are two approaches to avoid these biases. In the first approach the covariance matrix is calculated
using the expected values at the first iteration of the minimisation and kept fixed to these values for further
iterations. This method requires several repetitions of the minimisation, to ensure that values close to optimal
are obtained already at the first iteration. The second method modifies the $\chi^2$ function by adding a term
corresponding to non-constant value of the covariance matrix:
\begin{equation}
 \chi^2_{\rm log} = 2 \log \frac{\Delta^i(m^i)}{\Delta^i(\mu^i)} 
\end{equation}  

\subsection{HERAFitter implementation}

%%%%%%%%%%%
% \subsection{Using Covariance Matrix}
%%%%%%%%%%%%%%%%%%%%%%%%%%%%%


\section{Treatment of the Experimental Uncertainties}
%%%%%%%%%%%
\subsection{Hessian Method}

%%%%%%%%%%%
\subsection{Monte Carlo Method}


The PDF uncertainties can be estimated using a Monte Carlo technique \cite{mcmethod}.
The method consists in preparing replicas of data sets by allowing the central values of the cross sections to 
fluctuate within their systematic and statistical uncertainties taking into account all point-to-point correlations.
The preparation of the data is repeated for a large $N$ ($>100$ times) and for each of these replicas a NLO QCD fit is performed to 
extract the PDF set. The PDF central values and uncertainties are estimated using the means values and RMS 
over the replicas. 



%%%%%%%%%%%
\subsection{Regularisation methods}



%%%%%%%%%%%%%%%%%%%%%%%%%%%%%%

\section{Program Manual}
\label{sec:man}
%%%%%%%%%%%
In this section a user manual is presented. The section starts with a general overview of the code
organisation and it follows with a more detailed explanation for the most frequently used functions.

%%%%%%%%%%%%%%%%%%%%%%%%%%%%%%%%%%%%%%%%
\subsubsection{Code Organisation}
A general diagram of  available modules is illustrated in figure~\ref{fig:org}.
The flow is depicted such that it follows the structure of the \fitter\ .
\begin{figure}
\begin{center}
\includegraphics[width=0.75\linewidth]{figures/organisation.pdf}
\end{center}
\caption{Schematic structure of the \fitter\ program organisation in different modules.}
\label{fig:org}
\end{figure}

In addition, an inventory list with short description of existing subroutines is  
presented in Table~\ref{tab:list}. 
Here we choose to enlist only the routines from the common target module to guide 
the user of available functionalities.

\begin{center}
\begin{table}
\begin{tabular}{lp{4cm}p{10cm}}
\hline
\hline
\small\bf{steerings} & $\bullet$ steering.txt:& free PDF parameters to be varied by MINUIT \\
 & $\bullet$ minui.in.txt:& main steering card \\
 & $\bullet$ ewparam.txt:& settings of electroweak parameters, as well as masses \\
\toprule
\bf{src} & $\bullet$ main.f:& main program \\
& $\bullet$ read\_steer.f: &access steer parameters from steering card \\
& $\bullet$ read\_data.f: & reading the datatables and storing data information \\
& $\bullet$ init\_theory.f: & initialising theory modules \\
& $\bullet$ dataset\_tools.f:&  allocating bin indices \\
& $\bullet$ error\_logging.f: &  error logging information\\
& $\bullet$ minuit\_ini.f: & initialise minuit module \\
& $\bullet$ fcn.f: & passes to minuit the $\chi^2$ to be minimised \\
& $\bullet$ pdf\_param.f:  &  parametrisation of the PDFs at starting scale\\
& $\bullet$ sumrules.f: & PDF constraints at starting scale, such as QCD sum rules. \\
& $\bullet$ evolution.f: & evolution of PDFs \\
& $\bullet$ theory\_dispatcher.f: & distribution of theory prediction calculations for a given dataset  \\
& $\bullet$ dis\_sigma.f & calulates the DIS cross sections \\
& $\bullet$ GetChisquare.f & calculates the $\chi^2$  \\
& $\bullet$ GetCovChisquare.f & calculates the $\chi^2$ using covariance matrix \\
& $\bullet$ GetPointScaledErrors.f & calculates the rescaled statistical, uncorrelated and constant errors \\
& $\bullet$ prep\_corr.f & prepare systematic correlation matrix \\
& $\bullet$ systematics.f & build the matrix for systematic uncertainties and invert it \\
& $\bullet$ error\_bands\_pumplin.f  & Hessian error calculations  \\
& $\bullet$ mc\_errors.f & MC method for creating replicas of data through smearing. \\\cmidrule{2-3}
& $\bullet$ GetDiffDisXsection.f  & calulates the diffractive DIS cross sections \\
& $\bullet$ FixModelParams.f  & used for diffractive DIS cross sections \\\cmidrule{2-3}
& $\bullet$ lhapdf\_dum.f   & \tiny (used only with ENABLE\_LHPDF)\\
& $\bullet$ reweighting.f & main subroutine for PDF rewighting \tiny (used only with ENABLE\_NNPDF) \\
& $\bullet$ nnpdfreweighting.f & main subroutine for NNPDF rewighting \tiny (used only with ENABLE\_NNPDF) \\\cmidrule{2-3}
& $\bullet$ dy\_cc\_sigma.f  & calulates the DY cross sections \\
& $\bullet$ applgrids\_dum.f    & protective file against miss-use of flags in steering\\
& $\bullet$ fappl\_grid.cxx  &  \tiny (used only with ENABLE\_APPLGRID) \\
& $\bullet$ applgrids.f    &  passing PDFs to APPLGRID \tiny (used only with ENABLE\_APPLGRID) \\
& $\bullet$ pp\_jets\_applgrid.f   & calulates $pp$ jets cross sections  \\
& $\bullet$ ep\_jets\_fastnlo.f  &  calulates $ep$ jets cross sections \\\cmidrule{2-3} 
& $\bullet$ getncxskt.f  & access the NC cross sections grids for uPDFs\\
& $\bullet$ Getgridkt.f & acess the grids for uPDFs \\\cmidrule{2-3}
& $\bullet$  ttbar\_hathor\_dum.f  & protective file against miss-use of flags in steering\\
& $\bullet$  ttbar\_hathor.f  & \tiny ( used only with ENABLE\_HATHOR) \\\cmidrule{2-3}
& $\bullet$ offset\_fns.f & collects results from Offset method and stores them\\
& $\bullet$  g\_offset.cc  & file used for Offest method\\
& $\bullet$  matrix.cc  & inversion of matrix as used for Offset method \\
& $\bullet$  FitPars\_base.cc  & file used for Offest method\\
& $\bullet$  FTNFitPars.cc   & file used for Offest method\\
& $\bullet$  Xstring.cc   & file used for Offest method\\
& $\bullet$  decor.cc   & file used for Offest method\\\cmidrule{2-3}
& $\bullet$ store\_output.f & write the output  \\
& $\bullet$ store\_h1qcdfunc.f & store structure functions   \\
\bottomrule
\end{tabular}
\caption{A list of main subroutines are listed with a short description of their function.}
\label{tab:list}
\end{table}
\end{center}

%%%%%%%%%%%%%%%%%%%%%%%%%%%%%%%%%%%%%%%%
\subsubsection{Steering files}
 The software behavior is controlled by three files with steering commands.
 These files have predefined names:
    
\begin{itemize}
      \item {\tt steering.txt}  --   controls main "stable" (un-modified during 
                         minimisation) parameters. The file also contains
                         names of data files to be fitted and definitions 
                         of kinematic cuts                              
      \item {\tt minuit.in.txt}
                   --  controls minimisation parameters and minimisation 
                         strategy. Standard Minuit commands can be provided
                         in this file
      \item {\tt ewparam.txt}    --  controls electroweak parameters such
        as W and Z boson masses and CKM matrix parameters.
\end{itemize}


%%%%%%%%%%%%%%%%%%%%%%%%%%%%%%%%%%%%%%%%

\begin{description}
\item \bf{Steering.txt}\rm 

Different options are activated via steering flags in the main steering file.
%and the default steering file is displayed in figure  \ref{fig:steering}.
 
The format of the steering file follows standard "namelist" conventions.
Comments start with exclamation mark (similarly used for data file format).
The following namelist blocks are encountered:
\begin{itemize}
\item  {\tt InFiles}: Namelist to control input data
\item  {\tt InCorr}: Namelist to control statistical correlation files
\item  {\tt Scales} (Optional): Namelist to modify renormalisation/factorisation scale
\item  {\tt HeraFitter}: Main steering cards. 
%Further details can be found in the appendix \ref{sec:herafitter}. 
\item  {\tt ExtraMinimisationParameters}:  Namelist to add extra to minuit parameters.
\item  {\tt Output}: Namelist that outputs steering cards 
\item  {\tt Cuts}: Namelist for process dependent cuts
\item  {\tt MCErrors} (Optional):Namelist for MC errors steering cards
\item  {\tt Cheb} (Optional): Chebyshev study namelist
\item  {\tt Poly} (Optional): pure polynomial parameterisation for valence quarks
\item  {\tt HQScale} (Optional): choose the factorisation scale for HQs
\item  {\tt lhapdf} (Optional):LHAPDF steering card
\item  {\tt reweighting} (Optional): reweighting steering cards
\end{itemize}

These namelist blocks are described in greater details in the User's example \ref{section:example}. 


\begin{description}
\item \it\bf Theory type: \rm\\
 
here is a steering flag which defines the theory type via the chosen evolution.
The following types are supported:
\begin{itemize}
\item \tt{TheoryType = 'DGLAP' }\rm as used for collinear evolution theories. For this type another 
 flag is needed to specify the order of the perturbative series in $\alpha_S$:
\tt{Order}\rm which can be leading order (LO), next-to-leading order (NLO) and when available NNLO.
\item \tt{TheoryType = 'DIPOLE' }\rm as used for the dipole models;
\item \tt{TheoryType = 'uPDF' }\rm as used for the un-integrated PDFs (with 4 variants)
\end{itemize}

\item \it\bf Starting scale: \rm\\
The evolution starting scale is set via flag  \tt{Q02}\rm, commonly set below charm threshold, as imposed by QCDNUM.

\item \it\bf Scheme type: \rm\\
For the DIS process, several schemes are available for heavy quark treatments via  \tt{HF\_SCHEME}\rm flag.
\begin{itemize}
  \item VFNS (Variable Flavour Number Schemes):
    \begin{itemize}
     \item RT-VFNS  schemes                               [from Robert Thorne], \tt{HF\_SCHEME = RT, RT OPT}\rm, as well as the fast variants based on k-factors \tt{RT FAST, RT OPT FAST}
     \item Zero Mass VFNS                                 [qcdnum], \tt{ZM-VFNS}\rm   
     \item  ACOT (ACOT-Full, ACOT-ZM, S-ACOT-Chi) schemes  [from Fred Olness], \tt{HF\_SCHEME = ACOT Full, ACOT Chi, ACOT ZM}\rm, they are all based on k-factors. 
    \end{itemize}
  \item FFNS (Fixed Flavour Number Scheme)
    \begin{itemize}
    \item via QCDNUM, \tt{HF\_SCHEME = FF}\rm 
    \item via ABM (openqcdrad-1.6)   [from Sergey Alekhin], \tt{HF\_SCHEME = FF ABM}\rm
    \end{itemize}
\end{itemize}
IMPORTANT to note if running with FFNS (nf=3): 
\begin{itemize}
  \item only neutral current DIS data should be used in FF scheme due to missing NLO 
    coefficient functions in charged current process, in this cases valence quark parameters  
    need to be fixed in minuit.in.txt file.
  \item In FF ABM implementation the charged current coefficients are available
    therefore valence parameters do not need to be fixed.
  \item $\alpha_s(Q^2)$ in FFNS is 3-flavour and recommended to be set to the value of 0.105 
    such that is not too high at low energies
  \item the scale in FFNS is defined as $\mu^2 = Q^2 + 4m_h^2$ by default, can be 
    changed in HQScale in \tt{steering.txt}\rm (scale variation in ABM not yet implemented)
  \item  the pole mass definition for heavy quarks is set in ABM by default, 
    the running mass definition \cite{Alekhin:runm} can be switched in 
    by setting \tt{HF\_SCHEME = FF ABM RUNM} \rm in \tt{steering.txt} \rm. 
\end{itemize}

\item \it\bf PDF parametrisation style: \rm\\
There are various types of parametric functional form supported by \fitter\ .
They are accessed via the steering flag called \tt{PDFStyle}\rm. Available styles
are summarised as follows:

\begin{tabular}{ll}
 \tt  '10p HERAPDF'   & -- HERAPDF-like with extra assumption Buv = Bdv \\
 \tt   '13p HERAPDF' & -- HERAPDF-like with Buv and Bdv floated independently and two fur-\\
                     &  ther free parameters for the gluon PDF which allow it to be negative at \\
                     &  low scale\\
 \tt   '10p H12000'  & -- H12000-like (D,U,Dbar,Ubar+g) \\
 \tt   'CTEQ'        & -- CTEQ-like parameterisation \\
 \tt   'CHEB'        & -- CHEBYSHEV parameterisation based on glu, sea, uval, dval evolved\\
                     &  pdfs \\
 \tt  'LHAPDFQ0'    & -- use lhapdf library to define pdfs at starting scale and evolve with local \\
                    &  qcdnum parameters \\
 \tt  'LHAPDF'      & -- use lhapdf library to define pdfs at all scales\\
 \tt   ' DDIS'        & -- use Diffractive DIS \\
 \tt  'BiLog'       & -- bi-lognormal parametrisation \\
\end{tabular}


These styles were described in details in section~\ref{sec:pdfparam}.
The LHAPDF style can be used only with proper configuration settings,
 as explained in the section~\ref{sec:install}.
\item \it\bf  Definition of Chisquares:\rm\\
Currently two different formats of defining $\chi^2$ are supported.
The new format is explained in the section \ref{sec:chi2}.
%Currently two different styles are supported for a smoother 
%transition to the new style which is more flexible.
The old format corresponds to {\tt CHI2Style}  --- (string) choice of the $\chi^2$ function:\\
\begin{tabular}{ll}
    {\tt 'H12000'}& -- Pascaud-like, systematic shifts to theory, no scaling of statistical, uncorrelated errors.\\
    {\tt 'HERAPDF'}& -- Pascaud-like + "mixed error scaling"\\
    {\tt 'HERAPDF Sqrt'}&   -- Pascaud-like + "sqrt error scaling"\\
    {\tt 'HERAPDF Linear'}& -- Pascaud-like + "linear error scaling"\\
    {\tt 'Offset'}& -- Offset method activated\\
 \end{tabular}

\item \it\bf (logical) debug flag:\rm
 The debug flag will be turned on for more print outs via \tt LDEBUG. \rm

%%%%%%%%%%%%%%%%%%%%%%%%%%%%%%%%%%%%%%%%
%\subsubsection{Selection of the data}
\item \it\bf Selection of the data:\rm \\
  The namelist \&Cuts, located inside the {\tt steering.txt} file can be used to apply
  simple process dependent cuts. The cuts are limited to bin variables.
  Simple low and high limits are allowed. For example, a cut on $Q^2>3.5$~GeV$^2$ for
  NC ep scattering is specified as

\begin{verbatim}
  ! Rule #1: Q2 cuts
   ProcessName(1)     = 'NC e+-p'
   Variable(1)        = 'Q2'
   CutValueMin(1)     = 3.5 
   CutValueMax(1)     = 1000000.0
\end{verbatim}

Maximum 100 cuts can be used by default.
\end{description}


%%%%%%%%%%%%%%%%%%%%%%%%%%%%%%%%%%%%%%%%

The specific input files are stored in the {\tt $input\_steerings$} directory and 
it contains the following ready to use inputs (with corresponding minuit files):

\begin{itemize}
\item  {\tt steering.txt.ALLdata}: all data files 
\item  {\tt steering.txt.DIFFRACTION}: diffraction specific settings 
\item  {\tt steering.txt.kt-factorisation}: kt factorisation specific settings
\item  {\tt steering.txt.dipole}: dipole model specific settings  
\end{itemize}


%%%%
%\subsubsection{Options for Jets}
%%%%
%\subsubsection{Options for Diffractive fits}
%%%%
%\subsubsection{Options for DY fits}
%%%%%%%%%%%

%%%%%%%%%%%%%%%%%%%%%%%%%%%%%%%%%%%%%%%%
\item \bf {{\tt Minuit} steering cards}\rm


The minuit steering card is described below, a sample file is presented 
in Fig.~\ref{fig:minuit}
\begin{figure}
\begin{center}
\includegraphics[width=0.45\linewidth]{figures/minuit.pdf}
\end{center}
\caption{An example of a minuit steering card.}
\label{fig:minuit}
\end{figure}

The first three lines set the title and specify the list of MINUIT parameters which are to follow.      
The index of parameters is the first column and it is hardwired to the source code:\\

\begin{tabular}{ll}
1 -10 & gluon parameters \\    
11-20 & uval  parameters \\
21-30 & dval  parameters \\
31-40 & Ubar  parameters \\
41-50 & Dbar  parameters \\
51-60 & U     parameters \\
61-70 & D     parameters \\
71-80 & Sea   parameters \\
81-90 &Delta parameters \\
91-100 & other parameters: alphas (95), fs=Dbar/str (96), fc=Ubar/ch (97)\\
\end{tabular}

The second column represents just user defined names,
the third column is  the input starting value for the parameter.
The forth column sets the step size (usually chosen the same order as the error).
If the step size is zero this parameter is FIXED.
The fifth column sets the lower bound of the fit parameter, 
The sixth column sets the upper bound of the fit parameter
if these columns are not filled then there are no bounds.

Only parameters that have non-zero stepsize are varied 
in the fit (free parameters). Another way to fix the parameters is
simply by typing at the end of the list of parameters ``FIX parameter number''.  
(make sure there is one line free before the minuit list).
Examples of commands taken by minuit are:\\

\begin{tabular}{ll}
call fcn 3  &   fit is not performed, only 1 iteration, useful for testing\\
            &    Minuit parameters ARE NOT minimized. \\
migrad       & fit is performed (default number of calls 2000).\\
migrad 20000 & fit is performed up to 20000 calls, then terminates.\\
hesse        & Hessian estimate of the \tt{MINUIT}\rm parameters \\
& (more reliable than \tt{MINUIT}\rm)\\
\end{tabular}


The output of the fit is stored in the output/ directory as \tt{minuit.out.txt}\rm.
Statements in minuit.out.txt which are useful for interpreting the results of the fit:
\begin{itemize}
\item \tt{FCN=575.16}\rm  \; this is total chisquare
\item \tt{FROM MIGRAD   STATUS=CONVERGED}\rm \; this is desirable for a fit that converged
\item \tt{FROM HESSE     STATUS=OK}\rm       \; this is desirable for a fit that converged 
\item \tt{ERROR MATRIX ACCURATE}   \rm       \; errors estimated with HESSE method
\end{itemize}

{\bf HERAFitter parameters for diffractive fits} \\
%%-----------------------------------------------------------
\label{sec:HFitterPar}

% minuit.in.txt
% ExtraMinimisationParameters

\begin{tabular}{l|l|l}
Parameter & HERAFitter name & input file\\
% \hline
$\Cini {\mathrm G}1$ & Ag & minuit.in.txt \\
$\Cini {\mathrm G}2$ & Bg & minuit.in.txt \\
$\Cini {\mathrm G}3$ & Cg & minuit.in.txt \\
$\Cini {\mathrm S}1$ & Auv & minuit.in.txt \\
$\Cini {\mathrm S}2$ & Buv & minuit.in.txt \\
$\Cini {\mathrm S}3$ & Cuv & minuit.in.txt \\
$\alpha_\Pom(0)$ & Pomeron\_a0 & steering.txt \\
$A_\Reg$ & Reggeon\_factor & steering.txt \\
$\alpha_\Reg(0)$ & Reggeon\_a0 & steering.txt \\
\end{tabular}
\vspace{0.7cm}

{\bf HERAFitter parameters for dipole fits} \\
The default  initial parameters for the fit without valence quarks are : 
\begin{table}[h]
\tiny
\begin{center}
\begin{tabular}{|c||c||c||c|c||c|c|c||c|c|} 
\hline 
$\sigma_0$ & $A_g$ & $\lambda_g$ & $C_g$ & $cBGK$& $eBGK$\\
\hline
37.490 & 3.3446 & 0.0298 & 2.6302 & 4.0 & 15.362 \\
\hline
\end{tabular}
\end{center}
\end{table}
\\
For the BGK dipole model fits with valence quarks the initial parameters and the obtained $\chi^2$ are:

\begin{table}[ht]
\begin{center}
\begin{tabular}{|c||c||c||c|c||c|c|c||c|c|c||c|}
\hline
No&
$Q^2$&&
$\sigma_0$ & $A_g$ & $\lambda_g$ & $C_g$ & $C_{BGK}$& $\mu_{0}^2$& $Np$&
$\chi^2/Np$\\
\hline
1 &
$Q^2 \ge 3.5$ &  LO & 66.6 & 4.0 & -0.039 & 18.6& 4.0 & 5.3  & 196 &  0.930 \\
\hline
2 &
$Q^2 \ge 3.5$ & NLO & 79.4 & 3.2 &-0.021 & 13.7 & 4.0 & 6.7 & 196 & 0.927 \\
\hline
\end{tabular}
\end{center}
\end{table}
\end{description}


%%%%%%%%%%%%%%%%%%%%%%%%%%%%%%%%%%%%%%%%
\subsubsection{Data file format}
\label{sec:dataformat}
   Experimental data are provided by the standard {\tt ASCII} text files. The files
   contain a "header" which describes the data format and the "data" in terms
   of a  table. Each line of the data table corresponds to a
   data point, the meaning of the columns is specified in the file header.

   For example, a header for HERA-I combined H1-ZEUS data for e+p neutral 
   current scattering cross section is given in the file

\begin{verbatim}
       datafiles/H1ZEUS_NC_e-p_HERA1.0.dat
\end{verbatim}

   The format of the file follows standard "namelist" conventions. Comments 
   start with an exclamation mark.  Pre-defined variables are:
\begin{itemize}
     \item{\tt Name}        --- (string) provides the name of the data set
    \item{\tt  Reaction}    --- (string) reaction type of the data set. Reaction type is used 
                      to trigger the corresponding theory calculation. The following 
                      reaction types  are currently supported by the HERAFitter:
                      \begin{itemize}
                        \item {\tt 'NC e+-p'}  -- double differential NC ep scattering
                                      (ZM-VFNS and RT-VFNS schemes) 
                        \item {\tt 'CC e+-p'}  -- double differential CC ep scattering
                                      (ZM-VFNS scheme)
                        \item {\tt 'CC pp'}    -- single differential $d \sigma_{W^{\pm}}/d eta_{\ell^{\pm}}$
                                      production and W asymmetry at $pp$ and $p\bar{p}$ 
                                      colliders (LO+kfactors and APPLGRID interface)
                        \item {\tt 'NC pp'}    -- single differential $d \sigma_Z / d y_Z$ at $pp$ and
                                      $p\bar{p}$ colliders
                                      ({\tt LO} with k-factors and {\tt APPLGRID} interface)

                        \item 'pp jets APPLGRID' -- $pp\to$ inclusive jet production, using
                                     {\tt APPLGRID}

                        \item 'FastNLO jets' -- jet cross sections using {\tt FastNLO} interface.
                                     All $ep$, $pp$ and $p\bar{p}$ colliders are supported.

                        \item 'FastNLO ep jets normalised' -- jet cross sections in the $ep$ collisions 
                                     using {\tt FastNLO} interface and normalised to the inclusive DIS cross sections.

                      \end{itemize}                       
      \item {\tt NData}       --- (integer) specifies number of data points in the file. 
                     This corresponds to the number of table rows which 
                     follow after the header.
      \item {\tt NColumn}     --- (integer) number of columns in the data table.
      \item {\tt ColumnType}  --- (array of strings)
                      Defines layout of the data table. The following column types
                      are pre-defined: 'Bin', 'Sigma', 'Error' and 'Dummy'
                      The keywords are case sensitive. 'Bin' correspond to an
                      abstract bin definition, 'Sigma' corresponds to the data
                      measurement, 'Error' - to various type of uncertainties and
                      'Dummy' indicates that the column should be ignored.
      \item {\tt ColumnName}  --- (array of strings)
                      Defines names of the columns. The meaning of the name depends
                      on the ColumnType. For ColumnType 'Bin', ColumnName gives a
                      name of the abstract bin. The abstract bins can contain
                      any variable names, but some of them must be present for 
                      correct cross section calculation. For example, 'x', 'Q2' and
                      'y' are required for DIS NC cross-section calculation.
 
                      For ColumnType 'Sigma', ColumnName provides a label for 
                      the observable, which can be any string.
 
                      For ColumnType 'Error', the following names have special meaning:
                      \begin{itemize}
                       \item 'stat'  -- specifies column with statistical uncertainties, request Poisson re-scaling;
                       \item 'stat const'  -- specifies column with statistical uncertainties, request no re-scaling of the errors;
                       \item 'uncor' -- specifies column with uncorrelated uncertainties. Any name containing keyword ``uncor'' is treated as an uncorrelated
  error source, e.g. ``h1 uncor'';  
                       \item 'uncor const' -- specifies column with uncorrelated uncertainties, request no re-scaling of the errors;  
                       \item 'total' -- specifies column with total uncertainties. 
                                  Total uncertainties are not used in the fit,
                                  however there is an additional check is performed
                                  if 'total' column is specified: sum in quadrature
                                  of statistical, uncorrelated and correlated 
                                  systematic uncertainties is compared to the total
                                  and a warning is issued if they differ significantly.
                       \item'ignore' - specifies column to be ignored (for special studies).
                       \item Other names specifies columns of correlated systematic 
                      uncertainty. For a given data file, each column of the correlated
                      uncertainty must have unique name. To specify correlation across
                      data files, same name must be used for different files.  
                      \end{itemize}
      \item {\tt SystScales}  --- (array of float)
                      For special studies, systematic uncertainties can be scaled
                      The numbering of uncertainties starts from the first column
                      with the ColumnType 'Error'. For example, setting 
\begin{verbatim}
                  SystScale(1) = 2.  
\end{verbatim}
                      in {\tt datafiles/H1ZEUS\_NC\_e-p\_HERA1.0.dat} would scale systematic 
                      uncertainty by factor of two.                       
      \item {\tt Percent}     --- (array of bool) For each uncertainty specify if it is given in 
                      absolute ("false") or in percent ("true").  The numbering of 
                      uncertainties starts from the first column with the 
                      {\tt ColumnType} 'Error' (see example above).
      \item {\tt NInfo}       --- (integer) Calculation of the cross-section predictions may 
                      require  additional information about the data set. The number of 
                      information strings is given by NInfo
      \item {\tt CInfo}       --- (array of strings) Names of the information strings. 
                      Several of them are predefined for different cross-section 
                      calculations.
      \item {\tt DataInfo}    --- (array of float) Values, corresponding to {\tt CInfo} names.
      \item {\tt IndexDataset} -- (integer) Internal \fitter\ index of the data set. Provide unique
                      numbers to get extra info for $\chi^2/dof$ for each data set.      
      \item {\tt TheoryInfoFile} --- (string) Optional additional theory file with extra 
                     information for cross-section calculation. This could be k-factors,
                     {\tt APPLGRID} file or {\tt FastNLO} table.  
      \item {\tt TheoryType} --- (string) Theory file type ('kfactor', 'applgrid' or 'fastnlo').      
      \item {\tt NKFactor}   --- (integer) For kfactor files, number of columns in
                     {\tt TheoryInfoFile}.
      \item {\tt KFactorNames} --- (array of strings) For kfactor files, names of columns in 
                     {\tt TheoryInfoFile}.
\end{itemize}

Depending on the chosen process specific requirements for the header might be present. 
Dataset-wise options are provided by a {\tt CInfo} / {\tt DataInfo} variable set. In case the information
varies between data points (e.g. bin borders, hadronisation corrections etc.) it is
provided in the data file and recognised by the program using reserved column names.
In the following all these requirements are listed and briefly explained.

%%%%%%%%%%%%%%%%%%%%%%%%%%%%%%%%%%%%%%%%
\begin{description}
\item \bf{Data format requirements for DIS}\rm

In this subsection we describe specific requirements for files using 'NC e+-p' and 'CC e+-p'
reaction types. Examples of such input files are:

{\tt datafiles/H1ZEUS\_NC\_e-p\_HERA1.0.dat}

{\tt datafiles/H1ZEUS\_CC\_e-p\_HERA1.0.dat}.

The properly formatted DIS input files will have the following fields available
in the {\tt CInfo} variable list: 

\begin{itemize} 
    \item  {\tt 'sqrt(S)'} --- the ep collision centre-of-mass energy in GeV. In particular, for 
    HERA based results the the corresponding {\tt DataInfo} value should be $300.$ for measurements
    based on data collected prior to $1997$ (inclusive) and $318.$ for data collected after 1997.
    
    \item {\tt 'reduced'} --- a field indicating whether calculated cross section should be reduced (1.) or not (0.)
    (reference to proper equation somewhere in this manual).
    
    \item {\tt 'e charge'} --- electric charge of the colliding lepton beam. Supported {\tt DataInfo} values
    are '1.' for electron and '-1.' for positron.

    \item {\tt 'e polarity'} --- polarity of the lepton beam. The corresponding {\tt DataInfo} value 
    should be between $-1.0$ and $1.0$ (is this true?) with abs($1.0$) indicating fully polarised
    beam and $0.0$ fully unpolarised one. 

    In case of non-vanishing polarity the following additional fields are required:

    \item {\tt 'pol err unc'} --- explain

    \item {\tt 'pol err corLpol'} --- explain
 
    \item {\tt 'pol err corTpol'} --- explain

\end{itemize}

The inclusive DIS cross sections are calculated on an x-Q$^2$-y grid. Correspondingly,
the following columns need to present in the correctly formatted input file: 
{\tt 'x'}, {\tt 'Q2'} and {\tt 'y'}.


%%%%%%%%%%%%%%%%%%%%%%%%%%%%%%%%%%%%%%%%

\item \bf{Data format requirements for FastNLO} \rm

In this subsection we describe data format specific for the FastNLO implementation
accessed by choosing 'FastNLO jets' and 'FastNLO ep jets normalised' reaction types.
Examples of properly formatted files are:

   {\tt datafiles/HERA/ZEUS\_InclJets\_HighQ2\_98-00.dat}

   {\tt datafiles/HERA/H1\_NormInclJets\_HighQ2\_99-07.dat}.

{\tt TheoryType = 'FastNLO'} indicates usage of the FastNLO. The variable {\tt ThoryInfoFile} 
should contain the proper path to the FastNLO table in version 2.0 or higher.
\fitter\ supports both flexible and inflexible scales.
Older FastNLO tables can be still accessed through the APPLGRID interface.

The following fields are required to be present in the {\tt CInfo} list:

\begin{itemize}

    \item {\tt 'PublicationUnits'} --- The desired units in which the cross sections 
    are calculated by the FastNLO code. If the corresponding {\tt DataInfo} field is 
    set to '1.' the cross sections will be given in the same units as used in the
    relevant publication. In the case it is set to '0.', absolute cross section
    units will be used. 

    \item {\tt 'MurDef', 'MufDef'} --- The renormalisation and factorisation scale definitions
    used with variable scale FastNLO tables. If the chosen FastNLO table does not support 
    variable scales, these fields will be ignored and the scale embedded within the table will 
    be used instead. The values of the corresponding {\tt DataInfo} fields set 
    the renormalisation scale $\mu_r$ and factorisation scale $\mu_f$ following the FastNLO standard:
    \begin{align*} 
       \text{value} :&\quad \text{definition} \\
       0 :&\quad   \mu_{r/f}^2 = \mu_1^2 \\
       1 :&\quad   \mu_{r/f}^2 = \mu_2^2 \\
       2 :&\quad   \mu_{r/f}^2 = ( \mu_1^2 + \mu_2^2 )\\
       3 :&\quad   \mu_{r/f}^2 = ( \mu_1^2 + \mu_2^2 ) / 2 \\
       4 :&\quad   \mu_{r/f}^2 = ( \mu_1^2 + \mu_2^2 ) / 4 \\
       5 :&\quad   \mu_{r/f}^2 = (( \mu_1 + \mu_2 ) / 2 )^2\\
       6 :&\quad   \mu_{r/f}^2 = (( \mu_1 + \mu_2 ))^2\\
       7 :&\quad   \mu_{r/f}^2 = \text{max}( \mu_1^2, \mu_2^2)\\
       8 :&\quad   \mu_{r/f}^2 = \text{min}( \mu_1^2, \mu_2^2) \\
       9 :&\quad   \mu_{r/f}^2 = (\mu_1 * exp(0.3 * \mu_2)) ^2
   \end{align*}

   where $\mu_1$ and $\mu_2$ are specific scales chosen during production of the table. In particular
   for jet production at HERA traditionally 
   \begin{equation*}
          \mu_1^2 = Q^2 \quad \quad \quad \mu_2^2 = p_T^2
    \end{equation*}  

   \item {\tt sqrt(S)} --- Should be defined only for 'FastNLO ep jets normalised' reaction type. 
         The ep collision centre-of-mass energy in GeV. In particular, for 
         HERA based results the the corresponding {\tt DataInfo} value should be $300.$ for measurements
         based on data collected prior to $1997$ (inclusive) and $318.$ for data collected after 1997.

   \item {\tt 'lumi(e-)/lumi(tot)'} --- Should be defined only for 'FastNLO ep jets normalised'
         reaction type. The normalisation depends on the ratio of the luminosities of the positron and electron data 
         used for the cross section measurement. This ratio should  be
         given in a format (lumi($e^-$) / (lumi($e^-$) + lumi($e^+$)) and ashould take values between [0., 1.].

   \item {\tt 'UseZMVFNS'} --- Should be defined for 'FastNLO ep jets normalised' reaction type. The calculation
         of the integrated inclusive DIS cross sections could be time consuming.
         This option provides an opportunity to use a "Zero Mass Variable Flavour
         Number Scheme" approximation which is very fast and provides
         enough precision for normalisation purposes. ZMVNS is used if 
         the corresponding {\tt DataInfo} field is set to 1. Otherwise, the same scheme
         is used as defined globally with the variable 'HF\_SCHEME' defined in steering.txt file.
\end{itemize}


In addition there are some specific values within the {\tt ColumnName} field which allow
information specific to each data point to be passed. They are listed below:

\begin{itemize}
     \item{\tt 'Z0Corr'} --- (optional) The correction due to the $Z_0$ boson exchange.
                 If it is given, each point calculated by the FastNLO code will be
                 multiplied by the {\tt Z0Corr} value.

     \item{\tt 'NPCorr'} --- (optional) The non-perturbative correction.
                 If it is given, each point calculated by the FastNLO code will be
                 multiplied by the {\tt NPCorr} value. {\tt Z0Corr} and {\tt NPCorr} can be added 
                 simultaneously, and in this case the calculated cross sections
                 will be multiplied by the product {\tt Z0Corr} * {\tt NPCorr}.

    \item{\tt 'q2min', 'q2max', 'ymin', 'ymax', 'xmin', 'xmax'} --- Should be defined for 
         'FastNLO ep jets normalised' reaction type and are used to define 
         DIS phase space for the normalisation. Since these three ({\tt q2, y, x}) are 
         connected by the relation
         \begin{equation}
              Q^2 = x \cdot y \cdot s
         \end{equation}
         only two are required to be present to unambiguously define the DIS phase space for each data point.
        
\end{itemize}
\end{description}


%%%%%%%%%%%%%%%%%%%%%%%%%%%%%%%%%%%%%%%%
\subsubsection{Understanding the output}
  The results of the minimization are printed to the standard output and written
  to files in the {\tt output/} directory. 

  The quality of the fit can be judged based on the total $\chi^2$ per degrees of freedom.
  It is printed for each iteration as 
\begin{verbatim}
                      Iteration   Chi2   NDF       Chi2/NDF
   FitPDF f,ndf,f/ndf      3      588.64 579        1.02
\end{verbatim}
  The resulting $\chi^2$ is reported at the end of minimisation for each data set and for the correlated 
  systematic uncertainties separately. This information is printed and written
  to the {\tt output/Results.txt} file. The {\tt Results.txt} file contains additional 
  information about shifts of the correlated systematic uncertainties.

  The minimization information from the {\tt minuit} program is stored using the standard {\tt minuit} output format in the {\tt output/minuit.out.txt}
  file. The level of verbosity for this information can be changed by {\tt minuit} commands
  in the {\tt minuit.in.txt} file. 
  It is a good idea to check that minuit does not report any errors
  or warnings at the end of minimisation.
  
  Point by point comparison of the data and predictions after the minimization 
  is provided in the file stored in {\tt output/fittedresults.txt}. The file reports three columns
  corresponding to the three first bins of the input tables, data value, sum in 
  quadrature of statistical and uncorrelated systematic uncertainty, total
  uncertainty and then the predicted value, before and after applying correlated systematic shifts,
  the pull between the  data and theory and 
  the data set index. The pull $p$ is calculated as 
  \begin{equation}
      p = \frac{ \mu - m} {\sigma_{\rm uncor}}
  \end{equation}
  where $\mu$ is the data value, $m$ is the prediction and $\sigma_{\rm uncor}$ is the total
  uncorrelated uncertainty.
  Similar information is stored in the {\tt pulls.first.txt} and {\tt pulls.last.txt} files
  ( dataset index, first bin, second bin, third bin, theory, data, pull).
  Theory is  adjusted for systematic error shifts in this case.

  The output PDFs are stored in  {\tt output/pdfs\_q2val\_XX.txt} files.
  Each of the files reports values of gluon, and quark PDFs as a function of $x$
  for fixed $Q^2$ points. The $Q^2$ values and $x$ grid are specified by 
  {\tt \&Output} namelist in the {\tt steering.txt} file.
  
  The PDF information and data to theory comparisons can be plotted using 
  the {\tt bin/DrawResults} program.  Calling it without arguments plots results from the
  {\tt output/} directory. Giving the programme one argument specifies the sub-directory 
  where the information is read. Calling the {\tt bin/DrawResults} program with two
  arguments provides a comparison of the PDFs obtained in the two fits.
  
  In addition the \fitter\ package provides PDFs in the {\tt LHAPDF} format as {\tt output/lhapdf.block.txt} file. 
  To obtain the
  {\tt LHAPDF} grid file, run the {\tt tools/tolhapdf.cmd} script. The script provides a
  {\tt PDFs.LHgrid} file which can be read by the lhapdf version lhapdf-5.8.6.tar.gz
  or later.
%%%%%%%%%%%



%%%%%%%%%%%%%%%%%%%%%%%%%%%%%%

\bibliography{writeup.bib}
%%%%%%%%%%%%%%%%%%%%%%%%%%%%%%

\appendix
%%%%%%%%%%%%%%%%%%%%%%%%%%%%%%

\section{{\tt \&HERAFitter} namelist format}

\label{sec:herafitter}
\begin{itemize}
  \item {\tt ITheory} --- (integer) Currently only QCDNUM standard evolution
     is implemented for which {\tt ITheory} is set to 0.
  \item {\tt IOrder} --- (integer) For {\tt ITheory} =0 (collinear factorisation) : 
        LO fit (1) or NLO (2) or NNLO (3) 
  \item {\tt Q02} --- (float) Evolution starting scale.
  \item {\tt HF\_SCHEME} --- Specify heavy quark flavour treatment for neutral
 current $ep$ process. The following schemes are implemented: 
    \begin{itemize}
      \item {\tt 'ZMVFNS'}: Zero Mass Variable Flavour Number Scheme, as implemented
 in {\tt QCDNUM}.
      \item {\tt 'RT'}: Thorne-Roberts VFN scheme for $F_2^{\gamma}$. 
      \item {\tt 'RT FAST'}: Fast approximate RT VFN scheme using k-factor 
with respect ot QCDNUM ZMVFNS, calculated at the first iteration.
    \end{itemize}
\item {\tt PDFStyle} --- (string) PDF parameterisation style. Possible styles are currently available:
   \begin{itemize}
  \item{\tt '10p HERAPDF'} -- HERAPDF-like with an extra assumption 
                                 $B_{u_v} = B_{d_v}$;
  \item{\tt '13p HERAPDF'} -- HERAPDF-like with $B_{u_v}$ and $B_{d_v}$ 
                          floated independently;
  \item{\tt '10p H12000'}  -- H12000-like with independent PDFs being the
               $D,U,\bar{D},\bar{U}$ quarks and gluon.
  \item{\tt 'CTEQ'}        -- CTEQ-like parameterisation.
  \item{\tt 'CHEB'}        -- CHEBYSHEV parameterisation based on 
         gluon,sea, $u_{v}$, $d_{v}$ independent pdfs.
 \end{itemize}
\item {\tt CHI2Style}  --- (string) choice of the $\chi^2$ function:
   \begin{itemize}
   \item {\tt 'H12000'} -- Pascaud-like, systematic shifts to theory, no scaling of statistical, uncorrelated errors.
   \item {\tt 'HERAPDF'} -- Pascaud-like + "mixed error scaling"
   \item {\tt 'HERAPDF Sqrt'}   -- Pascaud-like + "sqrt error scaling"
   \item {\tt 'HERAPDF Linear'} -- Pascaud-like + "linear error scaling"
 \end{itemize}
  \item {\tt LDEBUG}  --- (logical) debug flag.
\end{itemize}
%%%%%%%%%%%%%%%%%%%%
\section{How do add new module}
\subsection{Structure of the module}
\subsection{Theory Interface Example}
\subsection{Steerings and Configurations}
%%%%%%%%%%%%%%%%%%%%%
\section{How to add new data}
Inclusion of the data files is controlled by {\tt \&InFiles} namelist in the 
{\tt steering.txt} file. For example, by default the following four HERA-I
    files are included:
\begin{verbatim}
&InFiles
    NInputFiles = 4
    InputFileNames(1) = 'datafiles/H1ZEUS_NC_e-p_HERA1.0.dat'
    InputFileNames(2) = 'datafiles/H1ZEUS_NC_e+p_HERA1.0.dat'
    InputFileNames(3) = 'datafiles/H1ZEUS_CC_e-p_HERA1.0.dat'
    InputFileNames(4) = 'datafiles/H1ZEUS_CC_e+p_HERA1.0.dat'
&End
\end{verbatim}

To include more files:
\begin{itemize}
 \item  Increase the {\tt NInputFiles} variable.
 \item  Specify the additional file by providing corresponding
  {\tt InputFileNames()} variable.
\end{itemize}
Details about data file format can be found in section~\ref{sec:dataformat}.



\end{document}

